% Options for packages loaded elsewhere
\PassOptionsToPackage{unicode}{hyperref}
\PassOptionsToPackage{hyphens}{url}
%
\documentclass[
]{article}
\title{Drake\_Zhou\_Part\_01}
\author{Drake Zhou}
\date{4/21/2022}

\usepackage{amsmath,amssymb}
\usepackage{lmodern}
\usepackage{iftex}
\ifPDFTeX
  \usepackage[T1]{fontenc}
  \usepackage[utf8]{inputenc}
  \usepackage{textcomp} % provide euro and other symbols
\else % if luatex or xetex
  \usepackage{unicode-math}
  \defaultfontfeatures{Scale=MatchLowercase}
  \defaultfontfeatures[\rmfamily]{Ligatures=TeX,Scale=1}
\fi
% Use upquote if available, for straight quotes in verbatim environments
\IfFileExists{upquote.sty}{\usepackage{upquote}}{}
\IfFileExists{microtype.sty}{% use microtype if available
  \usepackage[]{microtype}
  \UseMicrotypeSet[protrusion]{basicmath} % disable protrusion for tt fonts
}{}
\makeatletter
\@ifundefined{KOMAClassName}{% if non-KOMA class
  \IfFileExists{parskip.sty}{%
    \usepackage{parskip}
  }{% else
    \setlength{\parindent}{0pt}
    \setlength{\parskip}{6pt plus 2pt minus 1pt}}
}{% if KOMA class
  \KOMAoptions{parskip=half}}
\makeatother
\usepackage{xcolor}
\IfFileExists{xurl.sty}{\usepackage{xurl}}{} % add URL line breaks if available
\IfFileExists{bookmark.sty}{\usepackage{bookmark}}{\usepackage{hyperref}}
\hypersetup{
  pdftitle={Drake\_Zhou\_Part\_01},
  pdfauthor={Drake Zhou},
  hidelinks,
  pdfcreator={LaTeX via pandoc}}
\urlstyle{same} % disable monospaced font for URLs
\usepackage[margin=1in]{geometry}
\usepackage{color}
\usepackage{fancyvrb}
\newcommand{\VerbBar}{|}
\newcommand{\VERB}{\Verb[commandchars=\\\{\}]}
\DefineVerbatimEnvironment{Highlighting}{Verbatim}{commandchars=\\\{\}}
% Add ',fontsize=\small' for more characters per line
\usepackage{framed}
\definecolor{shadecolor}{RGB}{248,248,248}
\newenvironment{Shaded}{\begin{snugshade}}{\end{snugshade}}
\newcommand{\AlertTok}[1]{\textcolor[rgb]{0.94,0.16,0.16}{#1}}
\newcommand{\AnnotationTok}[1]{\textcolor[rgb]{0.56,0.35,0.01}{\textbf{\textit{#1}}}}
\newcommand{\AttributeTok}[1]{\textcolor[rgb]{0.77,0.63,0.00}{#1}}
\newcommand{\BaseNTok}[1]{\textcolor[rgb]{0.00,0.00,0.81}{#1}}
\newcommand{\BuiltInTok}[1]{#1}
\newcommand{\CharTok}[1]{\textcolor[rgb]{0.31,0.60,0.02}{#1}}
\newcommand{\CommentTok}[1]{\textcolor[rgb]{0.56,0.35,0.01}{\textit{#1}}}
\newcommand{\CommentVarTok}[1]{\textcolor[rgb]{0.56,0.35,0.01}{\textbf{\textit{#1}}}}
\newcommand{\ConstantTok}[1]{\textcolor[rgb]{0.00,0.00,0.00}{#1}}
\newcommand{\ControlFlowTok}[1]{\textcolor[rgb]{0.13,0.29,0.53}{\textbf{#1}}}
\newcommand{\DataTypeTok}[1]{\textcolor[rgb]{0.13,0.29,0.53}{#1}}
\newcommand{\DecValTok}[1]{\textcolor[rgb]{0.00,0.00,0.81}{#1}}
\newcommand{\DocumentationTok}[1]{\textcolor[rgb]{0.56,0.35,0.01}{\textbf{\textit{#1}}}}
\newcommand{\ErrorTok}[1]{\textcolor[rgb]{0.64,0.00,0.00}{\textbf{#1}}}
\newcommand{\ExtensionTok}[1]{#1}
\newcommand{\FloatTok}[1]{\textcolor[rgb]{0.00,0.00,0.81}{#1}}
\newcommand{\FunctionTok}[1]{\textcolor[rgb]{0.00,0.00,0.00}{#1}}
\newcommand{\ImportTok}[1]{#1}
\newcommand{\InformationTok}[1]{\textcolor[rgb]{0.56,0.35,0.01}{\textbf{\textit{#1}}}}
\newcommand{\KeywordTok}[1]{\textcolor[rgb]{0.13,0.29,0.53}{\textbf{#1}}}
\newcommand{\NormalTok}[1]{#1}
\newcommand{\OperatorTok}[1]{\textcolor[rgb]{0.81,0.36,0.00}{\textbf{#1}}}
\newcommand{\OtherTok}[1]{\textcolor[rgb]{0.56,0.35,0.01}{#1}}
\newcommand{\PreprocessorTok}[1]{\textcolor[rgb]{0.56,0.35,0.01}{\textit{#1}}}
\newcommand{\RegionMarkerTok}[1]{#1}
\newcommand{\SpecialCharTok}[1]{\textcolor[rgb]{0.00,0.00,0.00}{#1}}
\newcommand{\SpecialStringTok}[1]{\textcolor[rgb]{0.31,0.60,0.02}{#1}}
\newcommand{\StringTok}[1]{\textcolor[rgb]{0.31,0.60,0.02}{#1}}
\newcommand{\VariableTok}[1]{\textcolor[rgb]{0.00,0.00,0.00}{#1}}
\newcommand{\VerbatimStringTok}[1]{\textcolor[rgb]{0.31,0.60,0.02}{#1}}
\newcommand{\WarningTok}[1]{\textcolor[rgb]{0.56,0.35,0.01}{\textbf{\textit{#1}}}}
\usepackage{graphicx}
\makeatletter
\def\maxwidth{\ifdim\Gin@nat@width>\linewidth\linewidth\else\Gin@nat@width\fi}
\def\maxheight{\ifdim\Gin@nat@height>\textheight\textheight\else\Gin@nat@height\fi}
\makeatother
% Scale images if necessary, so that they will not overflow the page
% margins by default, and it is still possible to overwrite the defaults
% using explicit options in \includegraphics[width, height, ...]{}
\setkeys{Gin}{width=\maxwidth,height=\maxheight,keepaspectratio}
% Set default figure placement to htbp
\makeatletter
\def\fps@figure{htbp}
\makeatother
\setlength{\emergencystretch}{3em} % prevent overfull lines
\providecommand{\tightlist}{%
  \setlength{\itemsep}{0pt}\setlength{\parskip}{0pt}}
\setcounter{secnumdepth}{-\maxdimen} % remove section numbering
\ifLuaTeX
  \usepackage{selnolig}  % disable illegal ligatures
\fi

\begin{document}
\maketitle

{
\setcounter{tocdepth}{2}
\tableofcontents
}
\hypertarget{overview-preparation}{%
\section{Overview \& Preparation}\label{overview-preparation}}

The purpose of this part is to visualize all the data and assess them to
help me prepare for model fitting, which is also responsible for the 1st
part of final project requirements.

\hypertarget{package-import}{%
\subsection{Package import}\label{package-import}}

\begin{Shaded}
\begin{Highlighting}[]
\FunctionTok{library}\NormalTok{(tidyverse)}
\FunctionTok{library}\NormalTok{(visdat)}
\end{Highlighting}
\end{Shaded}

\hypertarget{data-import}{%
\subsection{Data Import}\label{data-import}}

\begin{Shaded}
\begin{Highlighting}[]
\NormalTok{df\_all }\OtherTok{\textless{}{-}}\NormalTok{ readr}\SpecialCharTok{::}\FunctionTok{read\_csv}\NormalTok{(}\StringTok{"final\_project\_train.csv"}\NormalTok{, }\AttributeTok{col\_names =} \ConstantTok{TRUE}\NormalTok{)}

\NormalTok{df\_all }\SpecialCharTok{\%\textgreater{}\%} \FunctionTok{glimpse}\NormalTok{()}
\end{Highlighting}
\end{Shaded}

\begin{verbatim}
## Rows: 677
## Columns: 38
## $ rowid    <dbl> 1, 3, 4, 5, 8, 9, 11, 14, 15, 16, 17, 18, 19, 22, 24, 25, 27,~
## $ region   <chr> "XX", "XX", "XX", "XX", "XX", "XX", "XX", "XX", "XX", "XX", "~
## $ customer <chr> "B", "B", "B", "B", "B", "B", "B", "B", "B", "B", "B", "B", "~
## $ xb_01    <dbl> 4.000000, 1.000000, 2.000000, 2.520000, 2.548387, 3.071429, 3~
## $ xb_02    <dbl> 4, 1, 2, 11, 6, 6, 10, 12, 9, 10, 8, 10, 10, 8, 6, 10, 13, 10~
## $ xb_03    <dbl> 4, 1, 2, -6, -1, 1, -4, -4, -2, -4, -2, -2, -2, -4, 1, -4, -3~
## $ xn_01    <dbl> 3.0000000, 2.0000000, 2.0000000, 1.5333333, 0.8387097, 1.8571~
## $ xn_02    <dbl> 3, 2, 4, 9, 3, 8, 6, 10, 10, 4, 6, 8, 9, 5, 7, 12, 12, 6, 6, ~
## $ xn_03    <dbl> 3, 2, 0, -3, -4, -2, -5, -6, -3, -5, -3, -6, -4, -3, 0, -5, -~
## $ xa_01    <dbl> 12.000000, 3.000000, 9.000000, 7.080000, 6.451613, 6.857143, ~
## $ xa_02    <dbl> 12, 3, 9, 29, 17, 18, 24, 27, 20, 19, 15, 24, 24, 15, 14, 26,~
## $ xa_03    <dbl> 12, 3, 9, -7, -2, 2, -9, -5, -3, -3, -1, 1, -2, -3, 3, -4, -5~
## $ xb_04    <dbl> 1.3333333, 1.0000000, 1.0000000, 0.8950476, 1.2247312, 1.1857~
## $ xb_05    <dbl> 1.3333333, 1.0000000, 1.0000000, -2.0000000, -0.5000000, 0.00~
## $ xb_06    <dbl> 1.333333, 1.000000, 1.000000, 4.000000, 4.000000, 3.000000, 6~
## $ xb_07    <dbl> 4.000000, 1.000000, 2.000000, 1.933333, 1.967742, 1.714286, 1~
## $ xb_08    <dbl> -1.00000000, 1.00000000, 0.00000000, -0.08000000, 0.35483871,~
## $ xn_04    <dbl> 1.0000000, 2.0000000, 1.0000000, 0.5268889, 0.4688172, 0.5607~
## $ xn_05    <dbl> 1.0000000, 2.0000000, 0.0000000, -1.0000000, -1.3333333, -1.0~
## $ xn_06    <dbl> 1.0, 2.0, 2.0, 2.5, 3.0, 2.0, 4.0, 4.0, 3.0, 2.0, 2.0, 2.5, 2~
## $ xn_07    <dbl> 3.000000, 2.000000, 2.500000, 1.493333, 1.225806, 1.642857, 1~
## $ xn_08    <dbl> -1.0000000, 2.0000000, -1.0000000, -0.4400000, -0.4516129, -0~
## $ xa_04    <dbl> 6.000000, 3.000000, 6.750000, 2.425333, 3.023656, 2.685714, 2~
## $ xa_05    <dbl> 6.0000000, 3.0000000, 4.5000000, -3.5000000, -0.6666667, 0.40~
## $ xa_06    <dbl> 6.000000, 3.000000, 9.000000, 9.000000, 13.000000, 6.000000, ~
## $ xa_07    <dbl> 9.000000, 3.000000, 7.500000, 4.466667, 4.612903, 4.071429, 4~
## $ xa_08    <dbl> 3.0000000, 3.0000000, 6.0000000, 0.7066667, 1.3225806, 1.3571~
## $ xw_01    <dbl> 23.00000, 17.00000, 52.50000, 64.52564, 54.75758, 58.33333, 6~
## $ xw_02    <dbl> 23, 17, 48, 0, 12, 15, 0, 0, 0, 7, 14, 0, 0, 0, 8, 8, 0, 4, 2~
## $ xw_03    <dbl> 23, 17, 57, 106, 105, 101, 107, 109, 109, 104, 109, 99, 103, ~
## $ xs_01    <dbl> 0.262073307, 0.330804757, 0.239795763, 0.142106837, 0.2442957~
## $ xs_02    <dbl> 0.26207331, 0.33080476, 0.19049123, -0.73321509, -0.12204299,~
## $ xs_03    <dbl> 0.2620733, 0.3308048, 0.2891003, 0.5500723, 1.3134719, 0.6540~
## $ xs_04    <dbl> 0.5375576, 0.4286607, 0.3676937, 0.2865445, 0.2375470, 0.2594~
## $ xs_05    <dbl> 0.5375575604, 0.4286607050, 0.2485001680, 0.0000000000, 0.043~
## $ xs_06    <dbl> 0.5375576, 0.4286607, 0.4868872, 0.6357541, 0.4327004, 0.8672~
## $ response <dbl> 2.617991, 1.184632, 2.216626, 2.726715, 1.483323, 2.039279, 1~
## $ outcome  <chr> "non_event", "non_event", "event", "non_event", "non_event", ~
\end{verbatim}

\hypertarget{exploratory-data-analysis-eda}{%
\section{Exploratory Data Analysis
(EDA)}\label{exploratory-data-analysis-eda}}

\hypertarget{general-inspection}{%
\subsection{General Inspection}\label{general-inspection}}

No missing data

\begin{Shaded}
\begin{Highlighting}[]
\NormalTok{visdat}\SpecialCharTok{::}\FunctionTok{vis\_miss}\NormalTok{(df\_all)}
\end{Highlighting}
\end{Shaded}

\includegraphics{Drake_Zhou_Part_01_files/figure-latex/unnamed-chunk-2-1.pdf}

\begin{Shaded}
\begin{Highlighting}[]
\NormalTok{df\_all }\SpecialCharTok{\%\textgreater{}\%}\NormalTok{ purrr}\SpecialCharTok{::}\FunctionTok{map\_dbl}\NormalTok{(n\_distinct)}
\end{Highlighting}
\end{Shaded}

\begin{verbatim}
##    rowid   region customer    xb_01    xb_02    xb_03    xn_01    xn_02 
##      677        3        9      229       19       21      225       18 
##    xn_03    xa_01    xa_02    xa_03    xb_04    xb_05    xb_06    xb_07 
##       18      257       38       35      364       59       51      181 
##    xb_08    xn_04    xn_05    xn_06    xn_07    xn_08    xa_04    xa_05 
##      187      360       51       47      174      174      411       87 
##    xa_06    xa_07    xa_08    xw_01    xw_02    xw_03    xs_01    xs_02 
##       87      213      212      396      102      103      676      644 
##    xs_03    xs_04    xs_05    xs_06 response  outcome 
##      672      676      663      676      677        2
\end{verbatim}

\hypertarget{distributions-of-variables.}{%
\subsection{Distributions of
variables.}\label{distributions-of-variables.}}

\hypertarget{categorical-variables}{%
\subsubsection{Categorical variables}\label{categorical-variables}}

Q: Counts for categorical variables.

A: From the figure, it looks like very imbalanced.

\begin{Shaded}
\begin{Highlighting}[]
\NormalTok{df\_all }\SpecialCharTok{\%\textgreater{}\%}
  \FunctionTok{count}\NormalTok{(outcome) }\SpecialCharTok{\%\textgreater{}\%}
  \FunctionTok{ggplot}\NormalTok{(}\AttributeTok{mapping =} \FunctionTok{aes}\NormalTok{(}\AttributeTok{x =}\NormalTok{ outcome, }\AttributeTok{y =}\NormalTok{ n))}\SpecialCharTok{+}
  \FunctionTok{geom\_col}\NormalTok{()}
\end{Highlighting}
\end{Shaded}

\includegraphics{Drake_Zhou_Part_01_files/figure-latex/solution_01_01-1.pdf}

\hypertarget{continuous-variables}{%
\subsubsection{Continuous variables}\label{continuous-variables}}

Q: Distributions for continuous variables. Are the distributions
Gaussian like?

A: Most of them looks Gaussian like

\begin{Shaded}
\begin{Highlighting}[]
\NormalTok{df\_con\_all }\OtherTok{\textless{}{-}} \FunctionTok{select}\NormalTok{(df\_all, }\FunctionTok{starts\_with}\NormalTok{(}\StringTok{"x"}\NormalTok{))}

\NormalTok{df\_con\_all }\SpecialCharTok{\%\textgreater{}\%} 
  \FunctionTok{select}\NormalTok{(}\FunctionTok{all\_of}\NormalTok{(}\FunctionTok{colnames}\NormalTok{(df\_con\_all))) }\SpecialCharTok{\%\textgreater{}\%} 
\NormalTok{  tibble}\SpecialCharTok{::}\FunctionTok{rowid\_to\_column}\NormalTok{() }\SpecialCharTok{\%\textgreater{}\%} 
  \FunctionTok{pivot\_longer}\NormalTok{(}\SpecialCharTok{!}\FunctionTok{c}\NormalTok{(}\StringTok{"rowid"}\NormalTok{)) }\SpecialCharTok{\%\textgreater{}\%}
  \FunctionTok{ggplot}\NormalTok{()}\SpecialCharTok{+}
  \FunctionTok{geom\_density}\NormalTok{(}\AttributeTok{mapping =} \FunctionTok{aes}\NormalTok{(}\AttributeTok{x =}\NormalTok{ value), }\AttributeTok{adjust =} \FloatTok{1.35}\NormalTok{, }\AttributeTok{size =} \FloatTok{0.5}\NormalTok{)}\SpecialCharTok{+}
  \FunctionTok{facet\_wrap}\NormalTok{(}\SpecialCharTok{\textasciitilde{}}\NormalTok{name, }\AttributeTok{scales =} \StringTok{"free"}\NormalTok{)}\SpecialCharTok{+}
  \FunctionTok{theme\_bw}\NormalTok{()}\SpecialCharTok{+}
  \FunctionTok{theme}\NormalTok{(}\AttributeTok{axis.text.y =} \FunctionTok{element\_blank}\NormalTok{(), }\AttributeTok{axis.text.x.bottom =} \FunctionTok{element\_blank}\NormalTok{())}
\end{Highlighting}
\end{Shaded}

\includegraphics{Drake_Zhou_Part_01_files/figure-latex/solution_01_02-1.pdf}

\hypertarget{log-response}{%
\subsubsection{Log-response}\label{log-response}}

\begin{Shaded}
\begin{Highlighting}[]
\NormalTok{df\_all }\SpecialCharTok{\%\textgreater{}\%} 
  \FunctionTok{ggplot}\NormalTok{(}\AttributeTok{mapping =} \FunctionTok{aes}\NormalTok{(}\AttributeTok{x =} \FunctionTok{log}\NormalTok{(response)))}\SpecialCharTok{+}
  \FunctionTok{geom\_histogram}\NormalTok{(}\AttributeTok{bins =} \DecValTok{35}\NormalTok{)}\SpecialCharTok{+}
  \FunctionTok{geom\_rug}\NormalTok{(}\AttributeTok{alpha =} \FloatTok{0.2}\NormalTok{) }\SpecialCharTok{+}
  \FunctionTok{theme\_bw}\NormalTok{()}
\end{Highlighting}
\end{Shaded}

\includegraphics{Drake_Zhou_Part_01_files/figure-latex/unnamed-chunk-4-1.pdf}

\hypertarget{affection-of-variables}{%
\subsection{Affection of variables}\label{affection-of-variables}}

Q: Are there differences in continuous variable distributions and
continuous variable summary statistics based on region or customer?

A: Yes, if we zoom in, we do observe some differences, but not very
significant.

\begin{Shaded}
\begin{Highlighting}[]
\NormalTok{df\_trans\_02 }\OtherTok{\textless{}{-}}\NormalTok{ df\_all }\SpecialCharTok{\%\textgreater{}\%}
  \FunctionTok{subset}\NormalTok{(}\AttributeTok{select =} \SpecialCharTok{{-}}\FunctionTok{c}\NormalTok{(rowid, outcome, response)) }\SpecialCharTok{\%\textgreater{}\%} 
  \FunctionTok{pivot\_longer}\NormalTok{(}\SpecialCharTok{!}\FunctionTok{c}\NormalTok{(}\StringTok{"region"}\NormalTok{, }\StringTok{"customer"}\NormalTok{))}
  
\NormalTok{df\_trans\_02 }\SpecialCharTok{\%\textgreater{}\%}
  \FunctionTok{ggplot}\NormalTok{(}\AttributeTok{mapping =} \FunctionTok{aes}\NormalTok{(}\AttributeTok{x =}\NormalTok{ name, }\AttributeTok{color =} \FunctionTok{as.factor}\NormalTok{(region)))}\SpecialCharTok{+}
  \FunctionTok{geom\_density}\NormalTok{()}\SpecialCharTok{+}
  \FunctionTok{facet\_wrap}\NormalTok{( }\SpecialCharTok{\textasciitilde{}}\NormalTok{ name, }\AttributeTok{scales =} \StringTok{"free\_y"}\NormalTok{)}\SpecialCharTok{+}
  \FunctionTok{theme\_bw}\NormalTok{()}\SpecialCharTok{+}
  \FunctionTok{theme}\NormalTok{(}\AttributeTok{axis.text.y =} \FunctionTok{element\_blank}\NormalTok{(), }\AttributeTok{axis.text.x.bottom =} \FunctionTok{element\_blank}\NormalTok{())}
\end{Highlighting}
\end{Shaded}

\includegraphics{Drake_Zhou_Part_01_files/figure-latex/solution_02_01-1.pdf}

Again, through this figure, we can see the different median of
continuous variables between different region.

\begin{Shaded}
\begin{Highlighting}[]
\NormalTok{df\_trans\_02 }\SpecialCharTok{\%\textgreater{}\%}
  \FunctionTok{ggplot}\NormalTok{(}\AttributeTok{mapping =} \FunctionTok{aes}\NormalTok{(}\AttributeTok{y =} \FunctionTok{as.factor}\NormalTok{(name), }\AttributeTok{x =}\NormalTok{ value))}\SpecialCharTok{+}
  \FunctionTok{geom\_boxplot}\NormalTok{(}\AttributeTok{mapping =} \FunctionTok{aes}\NormalTok{(}\AttributeTok{fill =} \FunctionTok{as.factor}\NormalTok{(region), }\AttributeTok{color =} \FunctionTok{as.factor}\NormalTok{(region)),}
               \AttributeTok{alpha =} \FloatTok{0.35}\NormalTok{, }\AttributeTok{size =} \FloatTok{0.1}\NormalTok{)}\SpecialCharTok{+}
  \FunctionTok{facet\_wrap}\NormalTok{(}\SpecialCharTok{\textasciitilde{}}\NormalTok{ name, }\AttributeTok{scales =} \StringTok{"free"}\NormalTok{)}\SpecialCharTok{+}
  \FunctionTok{scale\_fill\_viridis\_d}\NormalTok{(}\StringTok{"Region"}\NormalTok{) }\SpecialCharTok{+}
  \FunctionTok{scale\_color\_viridis\_d}\NormalTok{(}\StringTok{"Region"}\NormalTok{) }\SpecialCharTok{+}
  \FunctionTok{theme\_bw}\NormalTok{()}\SpecialCharTok{+}
  \FunctionTok{theme}\NormalTok{(}\AttributeTok{axis.text.y =} \FunctionTok{element\_blank}\NormalTok{(), }\AttributeTok{axis.text.x.bottom =} \FunctionTok{element\_blank}\NormalTok{())}
\end{Highlighting}
\end{Shaded}

\includegraphics{Drake_Zhou_Part_01_files/figure-latex/unnamed-chunk-5-1.pdf}

Q: Are there differences in continuous variable distributions and
continuous variable summary statistics based on the binary outcome?

A: Yes, if we zoom in, the differences are relative obviously

\begin{Shaded}
\begin{Highlighting}[]
\NormalTok{df\_trans\_02 }\SpecialCharTok{\%\textgreater{}\%} 
  \FunctionTok{ggplot}\NormalTok{(}\AttributeTok{mapping =} \FunctionTok{aes}\NormalTok{(}\AttributeTok{x =}\NormalTok{ name, }\AttributeTok{color =}\NormalTok{ customer))}\SpecialCharTok{+}
  \FunctionTok{geom\_density}\NormalTok{()}\SpecialCharTok{+}
  \FunctionTok{facet\_wrap}\NormalTok{( }\SpecialCharTok{\textasciitilde{}}\NormalTok{ name, }\AttributeTok{scales =} \StringTok{"free\_y"}\NormalTok{)}\SpecialCharTok{+}
  \FunctionTok{theme\_bw}\NormalTok{()}\SpecialCharTok{+}
  \FunctionTok{theme}\NormalTok{(}\AttributeTok{axis.text.y =} \FunctionTok{element\_blank}\NormalTok{(), }\AttributeTok{axis.text.x.bottom =} \FunctionTok{element\_blank}\NormalTok{())}
\end{Highlighting}
\end{Shaded}

\includegraphics{Drake_Zhou_Part_01_files/figure-latex/solution_02_02-1.pdf}

\begin{Shaded}
\begin{Highlighting}[]
\NormalTok{df\_trans\_02 }\SpecialCharTok{\%\textgreater{}\%}
  \FunctionTok{ggplot}\NormalTok{(}\AttributeTok{mapping =} \FunctionTok{aes}\NormalTok{(}\AttributeTok{x =} \FunctionTok{as.factor}\NormalTok{(name), }\AttributeTok{y =}\NormalTok{ value))}\SpecialCharTok{+}
  \FunctionTok{geom\_boxplot}\NormalTok{(}\AttributeTok{mapping =} \FunctionTok{aes}\NormalTok{(}\AttributeTok{fill =} \FunctionTok{as.factor}\NormalTok{(customer), }\AttributeTok{color =} \FunctionTok{as.factor}\NormalTok{(customer)),}
               \AttributeTok{alpha =} \FloatTok{0.35}\NormalTok{, }\AttributeTok{outlier.size =} \FloatTok{0.1}\NormalTok{)}\SpecialCharTok{+}
  \FunctionTok{facet\_wrap}\NormalTok{(}\SpecialCharTok{\textasciitilde{}}\NormalTok{ name, }\AttributeTok{scales =} \StringTok{"free"}\NormalTok{)}\SpecialCharTok{+}
  \FunctionTok{theme\_bw}\NormalTok{()}\SpecialCharTok{+}
  \FunctionTok{theme}\NormalTok{(}\AttributeTok{axis.text.y =} \FunctionTok{element\_blank}\NormalTok{(), }\AttributeTok{axis.text.x.bottom =} \FunctionTok{element\_blank}\NormalTok{())}
\end{Highlighting}
\end{Shaded}

\includegraphics{Drake_Zhou_Part_01_files/figure-latex/unnamed-chunk-6-1.pdf}

\hypertarget{correlation-and-relationshio-diagnosis}{%
\section{Correlation and Relationshio
diagnosis}\label{correlation-and-relationshio-diagnosis}}

\hypertarget{continuous-inputs-inputs}{%
\subsection{Continuous Inputs \&
Inputs}\label{continuous-inputs-inputs}}

Q: Visualize the relationships between the continuous inputs, are they
correlated?

A: Some of inputs are highly correlated to each other

\begin{Shaded}
\begin{Highlighting}[]
\NormalTok{df\_con\_all }\SpecialCharTok{\%\textgreater{}\%}
  \FunctionTok{cor}\NormalTok{() }\SpecialCharTok{\%\textgreater{}\%}
\NormalTok{  corrplot}\SpecialCharTok{::}\FunctionTok{corrplot}\NormalTok{(}\AttributeTok{type =} \StringTok{"upper"}\NormalTok{)}
\end{Highlighting}
\end{Shaded}

\includegraphics{Drake_Zhou_Part_01_files/figure-latex/solution_03-1.pdf}

\hypertarget{continuous-inputs-log-response}{%
\subsection{Continuous Inputs \&
Log-response}\label{continuous-inputs-log-response}}

Q: Visualize the relationships between the continuous outputs (response
and the log-transformed response) with respect to the continuous inputs.
Can you identify any clear trends? Do the trends depend on the
categorical inputs?

A: The input increase as the some of output parameters increase, like
xa\_01, xa\_02 and xa\_03. But some of them are not, and as we can see,
the categorical input does have impact on the output prediction.

\begin{Shaded}
\begin{Highlighting}[]
\NormalTok{df\_trans\_04 }\OtherTok{\textless{}{-}}\NormalTok{ df\_all }\SpecialCharTok{\%\textgreater{}\%}
  \FunctionTok{mutate}\NormalTok{(}\AttributeTok{log\_response =} \FunctionTok{log}\NormalTok{(response)) }\SpecialCharTok{\%\textgreater{}\%}
  \FunctionTok{select}\NormalTok{(}\FunctionTok{starts\_with}\NormalTok{(}\StringTok{\textquotesingle{}x\textquotesingle{}}\NormalTok{), log\_response, response, customer, region) }\SpecialCharTok{\%\textgreater{}\%}
  \FunctionTok{pivot\_longer}\NormalTok{(}\SpecialCharTok{!}\FunctionTok{c}\NormalTok{(log\_response, response, customer, region))}

\NormalTok{df\_trans\_04 }\SpecialCharTok{\%\textgreater{}\%} \FunctionTok{count}\NormalTok{()}
\end{Highlighting}
\end{Shaded}

\begin{verbatim}
## # A tibble: 1 x 1
##       n
##   <int>
## 1 22341
\end{verbatim}

\begin{Shaded}
\begin{Highlighting}[]
\NormalTok{df\_trans\_04 }\SpecialCharTok{\%\textgreater{}\%} 
  \FunctionTok{ggplot}\NormalTok{(}\AttributeTok{mapping =} \FunctionTok{aes}\NormalTok{(}\AttributeTok{x =}\NormalTok{ value, }\AttributeTok{y =}\NormalTok{ log\_response, }\AttributeTok{color =}\NormalTok{ region))}\SpecialCharTok{+}
  \FunctionTok{geom\_smooth}\NormalTok{(}\AttributeTok{method =} \StringTok{\textquotesingle{}loess\textquotesingle{}}\NormalTok{, }\AttributeTok{formula =}\NormalTok{ y }\SpecialCharTok{\textasciitilde{}}\NormalTok{ x, }\AttributeTok{size =} \FloatTok{0.4}\NormalTok{)}\SpecialCharTok{+}
  \FunctionTok{facet\_wrap}\NormalTok{(}\SpecialCharTok{\textasciitilde{}}\NormalTok{name, }\AttributeTok{scales =} \StringTok{"free"}\NormalTok{)}\SpecialCharTok{+}
  \FunctionTok{theme\_bw}\NormalTok{()}\SpecialCharTok{+}
  \FunctionTok{scale\_color\_viridis\_d}\NormalTok{(}\AttributeTok{option =} \StringTok{\textquotesingle{}plasma\textquotesingle{}}\NormalTok{) }\SpecialCharTok{+}
  \FunctionTok{theme}\NormalTok{(}\AttributeTok{axis.text.y =} \FunctionTok{element\_blank}\NormalTok{(), }\AttributeTok{axis.text.x.bottom =} \FunctionTok{element\_blank}\NormalTok{())}
\end{Highlighting}
\end{Shaded}

\includegraphics{Drake_Zhou_Part_01_files/figure-latex/solution_04_01-1.pdf}

\begin{Shaded}
\begin{Highlighting}[]
\NormalTok{df\_trans\_04 }\SpecialCharTok{\%\textgreater{}\%} 
  \FunctionTok{ggplot}\NormalTok{(}\AttributeTok{mapping =} \FunctionTok{aes}\NormalTok{(}\AttributeTok{x =}\NormalTok{ value, }\AttributeTok{y =}\NormalTok{ log\_response, }\AttributeTok{color =}\NormalTok{ customer))}\SpecialCharTok{+}
  \FunctionTok{geom\_smooth}\NormalTok{(}\AttributeTok{method =} \StringTok{\textquotesingle{}loess\textquotesingle{}}\NormalTok{, }\AttributeTok{formula =}\NormalTok{ y }\SpecialCharTok{\textasciitilde{}}\NormalTok{ x, }\AttributeTok{size =} \FloatTok{0.4}\NormalTok{)}\SpecialCharTok{+}
  \FunctionTok{facet\_wrap}\NormalTok{(}\SpecialCharTok{\textasciitilde{}}\NormalTok{name, }\AttributeTok{scales =} \StringTok{"free"}\NormalTok{)}\SpecialCharTok{+}
  \FunctionTok{theme\_bw}\NormalTok{()}\SpecialCharTok{+}
  \FunctionTok{theme}\NormalTok{(}\AttributeTok{axis.text.y =} \FunctionTok{element\_blank}\NormalTok{(), }\AttributeTok{axis.text.x.bottom =} \FunctionTok{element\_blank}\NormalTok{())}
\end{Highlighting}
\end{Shaded}

\includegraphics{Drake_Zhou_Part_01_files/figure-latex/solution_04_02-1.pdf}

\hypertarget{continuous-inputs-binary-outcome}{%
\subsection{Continuous Inputs \& Binary
outcome}\label{continuous-inputs-binary-outcome}}

Q:How can you visualize the behavior of the binary outcome with respect
to the continuous inputs?

A: As shown below, we can't just draw a vertical line to divide them.

\begin{Shaded}
\begin{Highlighting}[]
\NormalTok{df\_all }\SpecialCharTok{\%\textgreater{}\%}
  \FunctionTok{mutate}\NormalTok{(}\AttributeTok{log\_response =} \FunctionTok{log}\NormalTok{(response)) }\SpecialCharTok{\%\textgreater{}\%}
  \FunctionTok{select}\NormalTok{(}\FunctionTok{starts\_with}\NormalTok{(}\StringTok{\textquotesingle{}x\textquotesingle{}}\NormalTok{), outcome) }\SpecialCharTok{\%\textgreater{}\%}
  \FunctionTok{pivot\_longer}\NormalTok{(}\SpecialCharTok{!}\FunctionTok{c}\NormalTok{(outcome)) }\SpecialCharTok{\%\textgreater{}\%}
  \FunctionTok{ggplot}\NormalTok{(}\AttributeTok{mapping =} \FunctionTok{aes}\NormalTok{(}\AttributeTok{x =}\NormalTok{ value, }\AttributeTok{y =}\NormalTok{ outcome))}\SpecialCharTok{+}
  \FunctionTok{geom\_point}\NormalTok{(}\AttributeTok{mapping =} \FunctionTok{aes}\NormalTok{(}\AttributeTok{color =}\NormalTok{ outcome),}\AttributeTok{size =} \FloatTok{0.1}\NormalTok{)}\SpecialCharTok{+}
  \FunctionTok{facet\_wrap}\NormalTok{(}\SpecialCharTok{\textasciitilde{}}\NormalTok{name, }\AttributeTok{scales =} \StringTok{"free"}\NormalTok{)}\SpecialCharTok{+}
  \FunctionTok{theme\_bw}\NormalTok{()}\SpecialCharTok{+}
  \FunctionTok{theme}\NormalTok{(}\AttributeTok{axis.text.y =} \FunctionTok{element\_blank}\NormalTok{(), }\AttributeTok{axis.text.x.bottom =} \FunctionTok{element\_blank}\NormalTok{())}
\end{Highlighting}
\end{Shaded}

\includegraphics{Drake_Zhou_Part_01_files/figure-latex/solution_05-1.pdf}

\end{document}
