% Options for packages loaded elsewhere
\PassOptionsToPackage{unicode}{hyperref}
\PassOptionsToPackage{hyphens}{url}
%
\documentclass[
]{article}
\title{INFSCI 2595 Spring 2022: Homework 11}
\usepackage{etoolbox}
\makeatletter
\providecommand{\subtitle}[1]{% add subtitle to \maketitle
  \apptocmd{\@title}{\par {\large #1 \par}}{}{}
}
\makeatother
\subtitle{Assigned April 14, 2022; Due: April 21, 2022}
\author{Drake Zhou}
\date{Submission time: April 21, 2022 at 11:00PM EST}

\usepackage{amsmath,amssymb}
\usepackage{lmodern}
\usepackage{iftex}
\ifPDFTeX
  \usepackage[T1]{fontenc}
  \usepackage[utf8]{inputenc}
  \usepackage{textcomp} % provide euro and other symbols
\else % if luatex or xetex
  \usepackage{unicode-math}
  \defaultfontfeatures{Scale=MatchLowercase}
  \defaultfontfeatures[\rmfamily]{Ligatures=TeX,Scale=1}
\fi
% Use upquote if available, for straight quotes in verbatim environments
\IfFileExists{upquote.sty}{\usepackage{upquote}}{}
\IfFileExists{microtype.sty}{% use microtype if available
  \usepackage[]{microtype}
  \UseMicrotypeSet[protrusion]{basicmath} % disable protrusion for tt fonts
}{}
\makeatletter
\@ifundefined{KOMAClassName}{% if non-KOMA class
  \IfFileExists{parskip.sty}{%
    \usepackage{parskip}
  }{% else
    \setlength{\parindent}{0pt}
    \setlength{\parskip}{6pt plus 2pt minus 1pt}}
}{% if KOMA class
  \KOMAoptions{parskip=half}}
\makeatother
\usepackage{xcolor}
\IfFileExists{xurl.sty}{\usepackage{xurl}}{} % add URL line breaks if available
\IfFileExists{bookmark.sty}{\usepackage{bookmark}}{\usepackage{hyperref}}
\hypersetup{
  pdftitle={INFSCI 2595 Spring 2022: Homework 11},
  pdfauthor={Drake Zhou},
  hidelinks,
  pdfcreator={LaTeX via pandoc}}
\urlstyle{same} % disable monospaced font for URLs
\usepackage[margin=1in]{geometry}
\usepackage{color}
\usepackage{fancyvrb}
\newcommand{\VerbBar}{|}
\newcommand{\VERB}{\Verb[commandchars=\\\{\}]}
\DefineVerbatimEnvironment{Highlighting}{Verbatim}{commandchars=\\\{\}}
% Add ',fontsize=\small' for more characters per line
\usepackage{framed}
\definecolor{shadecolor}{RGB}{248,248,248}
\newenvironment{Shaded}{\begin{snugshade}}{\end{snugshade}}
\newcommand{\AlertTok}[1]{\textcolor[rgb]{0.94,0.16,0.16}{#1}}
\newcommand{\AnnotationTok}[1]{\textcolor[rgb]{0.56,0.35,0.01}{\textbf{\textit{#1}}}}
\newcommand{\AttributeTok}[1]{\textcolor[rgb]{0.77,0.63,0.00}{#1}}
\newcommand{\BaseNTok}[1]{\textcolor[rgb]{0.00,0.00,0.81}{#1}}
\newcommand{\BuiltInTok}[1]{#1}
\newcommand{\CharTok}[1]{\textcolor[rgb]{0.31,0.60,0.02}{#1}}
\newcommand{\CommentTok}[1]{\textcolor[rgb]{0.56,0.35,0.01}{\textit{#1}}}
\newcommand{\CommentVarTok}[1]{\textcolor[rgb]{0.56,0.35,0.01}{\textbf{\textit{#1}}}}
\newcommand{\ConstantTok}[1]{\textcolor[rgb]{0.00,0.00,0.00}{#1}}
\newcommand{\ControlFlowTok}[1]{\textcolor[rgb]{0.13,0.29,0.53}{\textbf{#1}}}
\newcommand{\DataTypeTok}[1]{\textcolor[rgb]{0.13,0.29,0.53}{#1}}
\newcommand{\DecValTok}[1]{\textcolor[rgb]{0.00,0.00,0.81}{#1}}
\newcommand{\DocumentationTok}[1]{\textcolor[rgb]{0.56,0.35,0.01}{\textbf{\textit{#1}}}}
\newcommand{\ErrorTok}[1]{\textcolor[rgb]{0.64,0.00,0.00}{\textbf{#1}}}
\newcommand{\ExtensionTok}[1]{#1}
\newcommand{\FloatTok}[1]{\textcolor[rgb]{0.00,0.00,0.81}{#1}}
\newcommand{\FunctionTok}[1]{\textcolor[rgb]{0.00,0.00,0.00}{#1}}
\newcommand{\ImportTok}[1]{#1}
\newcommand{\InformationTok}[1]{\textcolor[rgb]{0.56,0.35,0.01}{\textbf{\textit{#1}}}}
\newcommand{\KeywordTok}[1]{\textcolor[rgb]{0.13,0.29,0.53}{\textbf{#1}}}
\newcommand{\NormalTok}[1]{#1}
\newcommand{\OperatorTok}[1]{\textcolor[rgb]{0.81,0.36,0.00}{\textbf{#1}}}
\newcommand{\OtherTok}[1]{\textcolor[rgb]{0.56,0.35,0.01}{#1}}
\newcommand{\PreprocessorTok}[1]{\textcolor[rgb]{0.56,0.35,0.01}{\textit{#1}}}
\newcommand{\RegionMarkerTok}[1]{#1}
\newcommand{\SpecialCharTok}[1]{\textcolor[rgb]{0.00,0.00,0.00}{#1}}
\newcommand{\SpecialStringTok}[1]{\textcolor[rgb]{0.31,0.60,0.02}{#1}}
\newcommand{\StringTok}[1]{\textcolor[rgb]{0.31,0.60,0.02}{#1}}
\newcommand{\VariableTok}[1]{\textcolor[rgb]{0.00,0.00,0.00}{#1}}
\newcommand{\VerbatimStringTok}[1]{\textcolor[rgb]{0.31,0.60,0.02}{#1}}
\newcommand{\WarningTok}[1]{\textcolor[rgb]{0.56,0.35,0.01}{\textbf{\textit{#1}}}}
\usepackage{graphicx}
\makeatletter
\def\maxwidth{\ifdim\Gin@nat@width>\linewidth\linewidth\else\Gin@nat@width\fi}
\def\maxheight{\ifdim\Gin@nat@height>\textheight\textheight\else\Gin@nat@height\fi}
\makeatother
% Scale images if necessary, so that they will not overflow the page
% margins by default, and it is still possible to overwrite the defaults
% using explicit options in \includegraphics[width, height, ...]{}
\setkeys{Gin}{width=\maxwidth,height=\maxheight,keepaspectratio}
% Set default figure placement to htbp
\makeatletter
\def\fps@figure{htbp}
\makeatother
\setlength{\emergencystretch}{3em} % prevent overfull lines
\providecommand{\tightlist}{%
  \setlength{\itemsep}{0pt}\setlength{\parskip}{0pt}}
\setcounter{secnumdepth}{-\maxdimen} % remove section numbering
\ifLuaTeX
  \usepackage{selnolig}  % disable illegal ligatures
\fi

\begin{document}
\maketitle

\hypertarget{collaborators}{%
\paragraph{Collaborators}\label{collaborators}}

Include the names of your collaborators here.

\hypertarget{overview}{%
\subsection{Overview}\label{overview}}

This is an applied assignment focused on training binary classifiers.
You will try out several models discussed in lecture this semester
including: logistic regression, elastic net, neural network, random
forest, and gradient boosted trees. You will tune the models using
\texttt{caret} and then make predictions to study the trends.

\textbf{IMPORTANT}: code chunks are created for you. Each code chunk has
\texttt{eval=FALSE} set in the chunk options. You \textbf{MUST} change
it to be \texttt{eval=TRUE} in order for the code chunks to be evaluated
when rendering the document.

You are allowed to add as many code chunks as you see fit to answer the
questions.

\hypertarget{load-packages}{%
\subsection{Load packages}\label{load-packages}}

This assignment will use packages from the \texttt{tidyverse} suite and
\texttt{caret}. The models that you will train require their own set of
packages. The \texttt{caret::train()} function will prompt you to
download the necessary packages. Thus, when you run those code portions
please look at the R Console for the prompt that you must respond to.
The assignment also requires the \texttt{mlbench}, \texttt{corrplot},
and \texttt{yardstick} packages.

The \texttt{tidyverse} and \texttt{caret} packages are loaded for you
below. If you do not have \texttt{caret} please download and install
now.

\begin{Shaded}
\begin{Highlighting}[]
\FunctionTok{library}\NormalTok{(tidyverse)}
\end{Highlighting}
\end{Shaded}

\begin{verbatim}
## -- Attaching packages --------------------------------------- tidyverse 1.3.1 --
\end{verbatim}

\begin{verbatim}
## v ggplot2 3.3.5     v purrr   0.3.4
## v tibble  3.1.6     v dplyr   1.0.7
## v tidyr   1.1.4     v stringr 1.4.0
## v readr   2.1.1     v forcats 0.5.1
\end{verbatim}

\begin{verbatim}
## -- Conflicts ------------------------------------------ tidyverse_conflicts() --
## x dplyr::filter() masks stats::filter()
## x dplyr::lag()    masks stats::lag()
\end{verbatim}

\begin{Shaded}
\begin{Highlighting}[]
\FunctionTok{library}\NormalTok{(caret)}
\end{Highlighting}
\end{Shaded}

\begin{verbatim}
## Loading required package: lattice
\end{verbatim}

\begin{verbatim}
## 
## Attaching package: 'caret'
\end{verbatim}

\begin{verbatim}
## The following object is masked from 'package:purrr':
## 
##     lift
\end{verbatim}

\hypertarget{read-in-data}{%
\subsection{Read in data}\label{read-in-data}}

You will work with the \texttt{Sonar} data set in this assignment. The
Sonar data set has become a classic data set for training binary
classifiers. The data are available in the \texttt{mlbench} package. If
you do not have \texttt{mlbench} please download and install it before
proceeding.

The code chunk below reads in the \texttt{Sonar} data for you and
displays the number of rows and columns to screen.

\begin{Shaded}
\begin{Highlighting}[]
\FunctionTok{data}\NormalTok{(}\StringTok{"Sonar"}\NormalTok{, }\AttributeTok{package =} \StringTok{\textquotesingle{}mlbench\textquotesingle{}}\NormalTok{)}

\NormalTok{Sonar }\SpecialCharTok{\%\textgreater{}\%} \FunctionTok{dim}\NormalTok{()}
\end{Highlighting}
\end{Shaded}

\begin{verbatim}
## [1] 208  61
\end{verbatim}

As you can see above this a fairly high dimensional problem, especially
relative to the number of observations. The input features of the Sonar
data set correspond to features extracted from sonar signals (hence the
data name). Your task is to classify the binary outcome, \texttt{Class},
as either \texttt{"M"} for metal or \texttt{"R"} for rock.

\hypertarget{problem-01}{%
\subsection{Problem 01}\label{problem-01}}

We will perform a short exploration of the features in this data set.
First, let's look at summary statistics for each of the input variables.
The code chunk below creates a long-format dataset for you by gathering
the input features into a single column via the \texttt{pivot\_longer()}
function. The input variable ID is extracted for you and assigned to the
\texttt{input\_id} column.

\begin{Shaded}
\begin{Highlighting}[]
\NormalTok{lf\_sonar }\OtherTok{\textless{}{-}}\NormalTok{ Sonar }\SpecialCharTok{\%\textgreater{}\%} 
\NormalTok{  tibble}\SpecialCharTok{::}\FunctionTok{rowid\_to\_column}\NormalTok{(}\StringTok{"obs\_id"}\NormalTok{) }\SpecialCharTok{\%\textgreater{}\%} 
  \FunctionTok{pivot\_longer}\NormalTok{(}\SpecialCharTok{!}\FunctionTok{c}\NormalTok{(}\StringTok{"obs\_id"}\NormalTok{, }\StringTok{"Class"}\NormalTok{)) }\SpecialCharTok{\%\textgreater{}\%} 
  \FunctionTok{mutate}\NormalTok{(}\AttributeTok{input\_id =} \FunctionTok{as.numeric}\NormalTok{(stringr}\SpecialCharTok{::}\FunctionTok{str\_extract}\NormalTok{(name, }\StringTok{"}\SpecialCharTok{\textbackslash{}\textbackslash{}}\StringTok{d+"}\NormalTok{)))}

\NormalTok{lf\_sonar }\SpecialCharTok{\%\textgreater{}\%} \FunctionTok{glimpse}\NormalTok{()}
\end{Highlighting}
\end{Shaded}

\begin{verbatim}
## Rows: 12,480
## Columns: 5
## $ obs_id   <int> 1, 1, 1, 1, 1, 1, 1, 1, 1, 1, 1, 1, 1, 1, 1, 1, 1, 1, 1, 1, 1~
## $ Class    <fct> R, R, R, R, R, R, R, R, R, R, R, R, R, R, R, R, R, R, R, R, R~
## $ name     <chr> "V1", "V2", "V3", "V4", "V5", "V6", "V7", "V8", "V9", "V10", ~
## $ value    <dbl> 0.0200, 0.0371, 0.0428, 0.0207, 0.0954, 0.0986, 0.1539, 0.160~
## $ input_id <dbl> 1, 2, 3, 4, 5, 6, 7, 8, 9, 10, 11, 12, 13, 14, 15, 16, 17, 18~
\end{verbatim}

The above long-format data set will help with a few of the
visualizations we will use to explore the data.

\hypertarget{a}{%
\subsubsection{1a)}\label{a}}

You will start out by visualizing the summary statistics for each input
feature via boxplots.

\textbf{Pipe the \texttt{lf\_sonar} object into \texttt{ggplot()}. Map
the \texttt{x} aesthetic to \texttt{input\_id} and map the \texttt{y}
aesthetic to \texttt{value}. Include a \texttt{geom\_boxplot()} geom
where the \texttt{group} aesthetic is mapped to \texttt{input\_id}.}

\textbf{Describe the bounds on the input features. Do the inputs have
very different scales?}

\hypertarget{solution}{%
\paragraph{SOLUTION}\label{solution}}

\begin{Shaded}
\begin{Highlighting}[]
\DocumentationTok{\#\#\#}
\NormalTok{lf\_sonar }\SpecialCharTok{\%\textgreater{}\%}
  \FunctionTok{ggplot}\NormalTok{(}\AttributeTok{mapping =} \FunctionTok{aes}\NormalTok{(}\AttributeTok{x =}\NormalTok{ input\_id, }\AttributeTok{y =}\NormalTok{ value, }\AttributeTok{group =}\NormalTok{ input\_id))}\SpecialCharTok{+}
  \FunctionTok{geom\_boxplot}\NormalTok{()}
\end{Highlighting}
\end{Shaded}

\includegraphics{Drake_Zhou_HW_11_files/figure-latex/solution_01a-1.pdf}

What do you think?

It looks like inputs have huge variance, but the value of figure is
between 0 and 1, which is relative small. Except from those outliers,
they looks like similar scales

\hypertarget{b}{%
\subsubsection{1b)}\label{b}}

Boxplots are useful for visualizing summary statistics. However, we are
unable to visualize the shape of the distributions with boxplots. In
this problem, you will use frequency polygons (a type of histogram) to
visualize the distributional shapes for each input.

\textbf{Pipe the \texttt{lf\_sonar} object into \texttt{ggplot()} and
map the \texttt{x} aesthetic to \texttt{value}. Include a
\texttt{geom\_freqpoly()} geom with the \texttt{group} aesthetic mapped
to \texttt{input\_id}. Set the number bins equal to 21 by setting
\texttt{bins=21} in the \texttt{geom\_freqpoly()} call. Break up your
visualizations by groups of 5 inputs by including \texttt{facet\_wrap()}
where the facet variable is:}

\texttt{cut(input\_id,\ breaks=seq(1,\ 60,\ length.out\ =\ 12),\ include.lowest\ =\ TRUE)}

\textbf{Set the \texttt{scales} argument in \texttt{facet\_wrap()} to be
\texttt{scales=\textquotesingle{}free\_y\textquotesingle{}} so it's
easier to see the shapes for each group of inputs.}

\textbf{NOTE}: If your facet variable is created correctly, the facet
strips will show you the input IDs associated with each facet. For
example, the first facet will display \texttt{{[}1,6.36{]}} and the last
facet will be display \texttt{(54.6,60{]}}. Although decimal values are
shown, \texttt{input\_id} contains whole number integers. The decimals
are fine for now. It's just a simple way to break up the variables.

\hypertarget{solution-1}{%
\paragraph{SOLUTION}\label{solution-1}}

\begin{Shaded}
\begin{Highlighting}[]
\NormalTok{lf\_sonar }\SpecialCharTok{\%\textgreater{}\%}
  \FunctionTok{ggplot}\NormalTok{(}\AttributeTok{mapping =} \FunctionTok{aes}\NormalTok{(}\AttributeTok{x =}\NormalTok{ value))}\SpecialCharTok{+}
  \FunctionTok{geom\_freqpoly}\NormalTok{(}\AttributeTok{mapping =} \FunctionTok{aes}\NormalTok{(}\AttributeTok{group =}\NormalTok{ input\_id), }\AttributeTok{bins =} \DecValTok{21}\NormalTok{)}\SpecialCharTok{+}
  \FunctionTok{facet\_wrap}\NormalTok{(}\FunctionTok{vars}\NormalTok{(}\FunctionTok{cut}\NormalTok{(input\_id, }\AttributeTok{breaks=}\FunctionTok{seq}\NormalTok{(}\DecValTok{1}\NormalTok{, }\DecValTok{60}\NormalTok{, }\AttributeTok{length.out =} \DecValTok{12}\NormalTok{), }\AttributeTok{include.lowest =} \ConstantTok{TRUE}\NormalTok{)), }
             \AttributeTok{scales =} \StringTok{\textquotesingle{}free\_y\textquotesingle{}}\NormalTok{)}
\end{Highlighting}
\end{Shaded}

\includegraphics{Drake_Zhou_HW_11_files/figure-latex/solution_01b-1.pdf}

\hypertarget{c}{%
\subsubsection{1c)}\label{c}}

\textbf{Based on your visualizations, do you think the input features
should be pre-processed? If so, what processing technique should be
considered?}

\hypertarget{solution-2}{%
\paragraph{SOLUTION}\label{solution-2}}

What do you think?

Yes, according to the figure, some of values are extreme. We should
apply some pre-processing process. For example, z-score normalization.

\hypertarget{d}{%
\subsubsection{1d)}\label{d}}

Let's now consider the correlation structure of the inputs. Use the
\texttt{corrplot::corrplot()} function to create the correlation plot
matrix associated with all 60 input features. However, rather than using
the default ordering, instruct \texttt{corrplot()} to reorder inputs
such that all highly correlated inputs are grouped together.
\texttt{corrplot()} will use a hierarchical clustering method and group
all inputs for you.

\textbf{Pipe the original wide-format \texttt{Sonar} data into the
\texttt{dplyr::select()} function and select all except the
\texttt{Class} variable. Pipe the result into \texttt{cor()} and pipe
the result into \texttt{corrplot::corrplot()}. Set the \texttt{method}
argument equal to \texttt{"square"}, the \texttt{order} argument equal
to \texttt{"hclust"}, and the
`hclust.method\texttt{equal\ to}'ward.D2'`.}

\textbf{Is there a correlation structure between the inputs?}

\hypertarget{solution-3}{%
\paragraph{SOLUTION}\label{solution-3}}

Is there a correlation structure between the inputs?

Yes, there are a lot of correlated features

\begin{Shaded}
\begin{Highlighting}[]
\NormalTok{Sonar }\SpecialCharTok{\%\textgreater{}\%}
\NormalTok{  dplyr}\SpecialCharTok{::}\FunctionTok{select}\NormalTok{(}\SpecialCharTok{{-}}\NormalTok{Class) }\SpecialCharTok{\%\textgreater{}\%}
  \FunctionTok{cor}\NormalTok{() }\SpecialCharTok{\%\textgreater{}\%}
\NormalTok{  corrplot}\SpecialCharTok{::}\FunctionTok{corrplot}\NormalTok{(}\AttributeTok{method =} \StringTok{"square"}\NormalTok{, }\AttributeTok{order =} \StringTok{"hclust"}\NormalTok{, }\AttributeTok{hclust.method =} \StringTok{\textquotesingle{}ward.D2\textquotesingle{}}\NormalTok{)}
\end{Highlighting}
\end{Shaded}

\includegraphics{Drake_Zhou_HW_11_files/figure-latex/solution_01d-1.pdf}

\hypertarget{e}{%
\subsubsection{1e)}\label{e}}

The last visualization you will make as part of a quick exploratry data
analysis (EDA) is to count the number of observations per level of the
binary outcome \texttt{Class}. This is important to do before training
any binary classification model, in order to check if there is a severe
imbalance between either of the two classes.

\textbf{Pipe the \texttt{Sonar} data set into \texttt{ggplot()} and set
the \texttt{x} aesthetic equal to \texttt{Class}. Use the
\texttt{goem\_bar()} function to show a bar chart giving the number of
observations per level of \texttt{Class}.}

\textbf{Do you think we should be concerned about an imbalance between
the \texttt{"M"} and \texttt{"R"} levels?}

\hypertarget{solution-4}{%
\paragraph{SOLUTION}\label{solution-4}}

\begin{Shaded}
\begin{Highlighting}[]
\NormalTok{Sonar }\SpecialCharTok{\%\textgreater{}\%}
  \FunctionTok{ggplot}\NormalTok{(}\AttributeTok{mapping =} \FunctionTok{aes}\NormalTok{(}\AttributeTok{x =}\NormalTok{ Class))}\SpecialCharTok{+}
  \FunctionTok{geom\_bar}\NormalTok{()}
\end{Highlighting}
\end{Shaded}

\includegraphics{Drake_Zhou_HW_11_files/figure-latex/solution_01e-1.pdf}

What do you think?

No we shouldn't concerned about that, they are not exactly 50 to 50 but
still balanced.

\hypertarget{problem-02}{%
\subsection{Problem 02}\label{problem-02}}

You will train several models to predict the binary outcome
\texttt{Class}. You will try logistic regression, logistic regression
with the elastic net penalty, and more complex non-linear methods. As
stated at the beginning of the assignment, you will use \texttt{caret}
for the training, tuning, and model selection process.

\hypertarget{a-1}{%
\subsubsection{2a)}\label{a-1}}

You must start, by specifying the resampling scheme and the primary
performance metric. You will use 5-fold cross-validation, and you will
compare models by maximizing the Area Under the ROC Curve (ROC AUC). You
must specify the metric to be \texttt{"ROC"} in order to tell
\texttt{caret} to maximize the AUC (the naming convention is a little
confusing). The code chunk below is started for you, and provides some
arguments to the \texttt{trainControl()} function which are required in
order to use \texttt{"ROC"} as the primary performance metric.

\textbf{Complete the code chunk below. Finish the
\texttt{trainControl()} call such that you will use 5-fold
cross-validation. Assign \texttt{"ROC"} to the \texttt{metric\_sonar}
variable.}

\hypertarget{solution-5}{%
\paragraph{SOLUTION}\label{solution-5}}

\begin{Shaded}
\begin{Highlighting}[]
\NormalTok{ctrl\_k05\_roc }\OtherTok{\textless{}{-}} \FunctionTok{trainControl}\NormalTok{(}\AttributeTok{method =} \StringTok{\textquotesingle{}cv\textquotesingle{}}\NormalTok{,}
                             \AttributeTok{number =} \DecValTok{5}\NormalTok{,}
                             \AttributeTok{summaryFunction =}\NormalTok{ twoClassSummary,}
                             \AttributeTok{classProbs =} \ConstantTok{TRUE}\NormalTok{,}
                             \AttributeTok{savePredictions =} \ConstantTok{TRUE}\NormalTok{)}

\NormalTok{metric\_sonar }\OtherTok{\textless{}{-}} \StringTok{"ROC"}
\end{Highlighting}
\end{Shaded}

\hypertarget{b-1}{%
\subsubsection{2b)}\label{b-1}}

You will start with a logistic regression model using linear additive
terms for all input features. Remember that the short cut operator
\texttt{.} denotes using ``everything in the data set''. So you do not
have to type out all 60 input variable names in the formula interface.
The formula requires the name of the response variable, and so remember
that the outcome is the \texttt{Class} variable, not \texttt{y} as in
other assignments and lecture examples.

You will use the \texttt{train()} function to train the logistic
regression model. You must specify the formula interface, and specify
the \texttt{data} to be \texttt{Sonar}. To use the base \texttt{R}
logistic regression method in \texttt{glm()}, you must set the
\texttt{method} argument equal to \texttt{"glm"}. You must also specify
the \texttt{metric} argument to be the \texttt{metric\_sonar} variable
you assigned Problem 2a) in order for \texttt{caret} to identify the
best model by maximizing the area under the ROC curve. You must also
specify the \texttt{trControl} argument to be the
\texttt{ctrl\_k05\_roc}.

You must set the \texttt{preProcess} argument based on your answer to
Problem 1c). If you do not feel you need to pre-process the inputs then
you do not need to include the \texttt{preProcess} argument. If you feel
you should, then you should set the \texttt{preProcess} argument to your
desired pre-processing operation.

\textbf{Specify the arguments to the \texttt{train()} function in order
to train a logistic regression model for the \texttt{Sonar} data set
with 5-fold cross-validation and calculate the area under the ROC
curve.}

\textbf{The area under the ROC curve is referred to by \texttt{caret} as
\texttt{"ROC"}. The cross-validation averaged performance merics are
printed to the screen for you. What is the area under the ROC curve for
your logistic regression model?}

\emph{HINT}: You can ignore warnings displayed during the training of
the logistic regression model.

\hypertarget{solution-6}{%
\paragraph{SOLUTION}\label{solution-6}}

\begin{Shaded}
\begin{Highlighting}[]
\FunctionTok{set.seed}\NormalTok{(}\DecValTok{4321}\NormalTok{)}
\NormalTok{fit\_glm\_sonar }\OtherTok{\textless{}{-}} \FunctionTok{train}\NormalTok{(Class }\SpecialCharTok{\textasciitilde{}}\NormalTok{ (.),}
                       \AttributeTok{data =}\NormalTok{ Sonar,}
                       \AttributeTok{method =} \StringTok{"glm"}\NormalTok{,}
                       \AttributeTok{metric =}\NormalTok{ metric\_sonar,}
                       \AttributeTok{preProcess =} \FunctionTok{c}\NormalTok{(}\StringTok{"center"}\NormalTok{, }\StringTok{"scale"}\NormalTok{),}
                       \AttributeTok{trControl =}\NormalTok{ ctrl\_k05\_roc)}

\NormalTok{fit\_glm\_sonar}
\end{Highlighting}
\end{Shaded}

\begin{verbatim}
## Generalized Linear Model 
## 
## 208 samples
##  60 predictor
##   2 classes: 'M', 'R' 
## 
## Pre-processing: centered (60), scaled (60) 
## Resampling: Cross-Validated (5 fold) 
## Summary of sample sizes: 166, 167, 165, 167, 167 
## Resampling results:
## 
##   ROC        Sens       Spec     
##   0.7690207  0.7747036  0.7036842
\end{verbatim}

What is the area under the ROC curve for your logistic regression model?

As we can see through the data, the area under ROC curve is 0.7690207

\hypertarget{c-1}{%
\subsubsection{2c)}\label{c-1}}

Since there are relatively few observations based on the number of
inputs in \texttt{Sonar}, let's apply regularization to the linear
additive features. Because you are using linear additive terms you may
use the shortcut operator \texttt{.} in the formula. The \texttt{glmnet}
package will fit a logistic regression model with the elastic net
penalty term. You need to set \texttt{method} to \texttt{"glmnet"}. The
remaining arguments should be consistent to your arguments used in
Problem 2b).

\textbf{Train an elastic net model and tune the hyperparameters with
5-fold cross-validation to maximize the area under the ROC curve. You
can use the default tuning grid from \texttt{caret} and so you do NOT
need to set the \texttt{tuneGrid} argument.}

\textbf{Should you consider applying preprocessing with the regularized
model?}

\textbf{Based on the training results, does the elastic net model favor
LASSO or RIDGE more?}

\hypertarget{solution-7}{%
\paragraph{SOLUTION}\label{solution-7}}

\begin{Shaded}
\begin{Highlighting}[]
\FunctionTok{set.seed}\NormalTok{(}\DecValTok{4321}\NormalTok{)}
\NormalTok{fit\_glmnet\_sonar }\OtherTok{\textless{}{-}} \FunctionTok{train}\NormalTok{(Class }\SpecialCharTok{\textasciitilde{}}\NormalTok{ .,}
                       \AttributeTok{data =}\NormalTok{ Sonar,}
                       \AttributeTok{method =} \StringTok{"glmnet"}\NormalTok{,}
                       \AttributeTok{metric =}\NormalTok{ metric\_sonar,}
                       \AttributeTok{preProcess =} \FunctionTok{c}\NormalTok{(}\StringTok{"center"}\NormalTok{, }\StringTok{"scale"}\NormalTok{),}
                       \AttributeTok{trControl =}\NormalTok{ ctrl\_k05\_roc)}
\NormalTok{fit\_glmnet\_sonar}
\end{Highlighting}
\end{Shaded}

\begin{verbatim}
## glmnet 
## 
## 208 samples
##  60 predictor
##   2 classes: 'M', 'R' 
## 
## Pre-processing: centered (60), scaled (60) 
## Resampling: Cross-Validated (5 fold) 
## Summary of sample sizes: 166, 167, 165, 167, 167 
## Resampling results across tuning parameters:
## 
##   alpha  lambda        ROC        Sens       Spec     
##   0.10   0.0004318733  0.8381527  0.8019763  0.7221053
##   0.10   0.0043187332  0.8567839  0.8118577  0.7526316
##   0.10   0.0431873324  0.8528271  0.7762846  0.7531579
##   0.55   0.0004318733  0.8345278  0.7928854  0.7026316
##   0.55   0.0043187332  0.8543437  0.8118577  0.7426316
##   0.55   0.0431873324  0.8367568  0.8027668  0.7226316
##   1.00   0.0004318733  0.8250947  0.7837945  0.6931579
##   1.00   0.0043187332  0.8511889  0.7940711  0.7321053
##   1.00   0.0431873324  0.8319097  0.8027668  0.7226316
## 
## ROC was used to select the optimal model using the largest value.
## The final values used for the model were alpha = 0.1 and lambda = 0.004318733.
\end{verbatim}

Should you consider applying preprocessing with the regularized model?
Based on the training results, does the elastic net model favor LASSO or
RIDGE more?

Yes, we should. If alpha = 1, the elastic net will equvalent to LASSO,
since the final value of alpha for the model is 0.1, we can we the
elastic net is more favor of RIDGE.

\hypertarget{d-1}{%
\subsubsection{2d)}\label{d-1}}

Let's now try a neural network model with the \texttt{nnet} package. You
must specify the \texttt{method} argument equal to \texttt{"nnet"}.
Non-linear models will attempt to find non-linear relationships between
the inputs and the response even if the simple formula
\texttt{Class\ \textasciitilde{}\ .} is provided. Thus, you do not need
to type in all input names to the formula object. The \texttt{.}
shortcut of ``everything else'' is instructing the formula to use
``everything else'' as an input. You can use the default tuning grid,
and so you do not need to specify the \texttt{tuneGrid} argument to
\texttt{train()}.

\textbf{Train a neural network binary classifier using the
\texttt{"nnet"} package with \texttt{caret}. Does the neural network
model achieve a higher area under the ROC curve compared to the elastic
net model?}

\textbf{Should you preprocess the inputs before training the neural
network?}

\textbf{NOTE}: The \texttt{trace} argument is set to \texttt{FALSE} for
you already. Otherwise the iteration results are printed. We are focused
on the model performance results in this assignment, and so we do not
need the iteration results printed to the screen.

\hypertarget{solution-8}{%
\paragraph{SOLUTION}\label{solution-8}}

\begin{Shaded}
\begin{Highlighting}[]
\FunctionTok{set.seed}\NormalTok{(}\DecValTok{4321}\NormalTok{)}
\NormalTok{fit\_nnet\_sonar }\OtherTok{\textless{}{-}} \FunctionTok{train}\NormalTok{(Class }\SpecialCharTok{\textasciitilde{}}\NormalTok{ .,}
                        \AttributeTok{data =}\NormalTok{ Sonar,}
                        \AttributeTok{method =} \StringTok{\textquotesingle{}nnet\textquotesingle{}}\NormalTok{,}
                        \AttributeTok{metric =}\NormalTok{ metric\_sonar,}
                        \AttributeTok{preProcess =} \FunctionTok{c}\NormalTok{(}\StringTok{"center"}\NormalTok{, }\StringTok{"scale"}\NormalTok{),}
                        \AttributeTok{trControl =}\NormalTok{ ctrl\_k05\_roc,}
                        \AttributeTok{trace =} \ConstantTok{FALSE}\NormalTok{)}

\NormalTok{fit\_nnet\_sonar}
\end{Highlighting}
\end{Shaded}

\begin{verbatim}
## Neural Network 
## 
## 208 samples
##  60 predictor
##   2 classes: 'M', 'R' 
## 
## Pre-processing: centered (60), scaled (60) 
## Resampling: Cross-Validated (5 fold) 
## Summary of sample sizes: 166, 167, 165, 167, 167 
## Resampling results across tuning parameters:
## 
##   size  decay  ROC        Sens       Spec     
##   1     0e+00  0.7778292  0.7316206  0.7547368
##   1     1e-04  0.8122769  0.7403162  0.7547368
##   1     1e-01  0.8423237  0.7675889  0.7226316
##   3     0e+00  0.7711421  0.7584980  0.7226316
##   3     1e-04  0.8440862  0.8035573  0.7521053
##   3     1e-01  0.8659455  0.7948617  0.7115789
##   5     0e+00  0.8004421  0.7494071  0.7631579
##   5     1e-04  0.8361234  0.7675889  0.7310526
##   5     1e-01  0.8859871  0.8122530  0.7736842
## 
## ROC was used to select the optimal model using the largest value.
## The final values used for the model were size = 5 and decay = 0.1.
\end{verbatim}

Does the neural network model achieve a higher area under the ROC curve
compared to the elastic net model?

Yes, according to the data, the value of ROC column is bigger than
before, which represent the area under ROC curve is bigger.

Should you preprocess the inputs before training the neural network?

Yes, in order to standardize the inputs to avoid some problem like
exploding gradients.

\hypertarget{e-1}{%
\subsubsection{2e)}\label{e-1}}

Neural networks are challenging to interpret. Several methods have been
devised to try and provide interpretibility by providing ways to rank
the input variable importance using the weights associated with all
hidden units and the output layer.\\
\textbf{You may use the default \texttt{caret} provided method for
ranking variable importance for the neural network. Plot the variable
importances using the default plot method. Then plot the variable
importances a second time but set the \texttt{top} argument equal to 20
in the \texttt{plot()} method call. This will limit to the plot to just
the top 20 ranked inputs.}

\textbf{What are the top 4 ranked input features as viewed by the neural
network?}

\hypertarget{solution-9}{%
\paragraph{SOLUTION}\label{solution-9}}

Plot all variable importances.

\begin{Shaded}
\begin{Highlighting}[]
\DocumentationTok{\#\#\#}
\FunctionTok{plot}\NormalTok{(}\FunctionTok{varImp}\NormalTok{(fit\_nnet\_sonar))}
\end{Highlighting}
\end{Shaded}

\includegraphics{Drake_Zhou_HW_11_files/figure-latex/solution_02e-1.pdf}

Plot just the top 20 variable importances.

\begin{Shaded}
\begin{Highlighting}[]
\DocumentationTok{\#\#\#}
\FunctionTok{plot}\NormalTok{(}\FunctionTok{varImp}\NormalTok{(fit\_nnet\_sonar), }\AttributeTok{top =} \DecValTok{20}\NormalTok{)}
\end{Highlighting}
\end{Shaded}

\includegraphics{Drake_Zhou_HW_11_files/figure-latex/solution_02e_b-1.pdf}

What do you think?

V50, V57, V31 and V22

\hypertarget{f}{%
\subsubsection{2f)}\label{f}}

You will use predictions to help gain further understanding of the
influence of the top ranked input features. Two functions are provided
for you below which you will use to construct an input grid to make
predictions. The first function, \texttt{make\_test\_input\_list()} is
defined in the code chunk below. The first argument to
\texttt{make\_test\_input\_list()} is a variable name. The second input
argument, \texttt{top\_4\_inputs}, is a character vector. The second
input argument will hold the names associated with the top 4 ranked
input features. The last input argument, \texttt{all\_data}, is a tibble
or data.frame. The last input argument is intended to hold the training
data set.

The \texttt{make\_test\_input\_list()} function therefore creates a grid
of 25 evenly spaced points between the training set min and max bounds
for the top 2 ranked inputs. The third and fourth ranked inputs are
given 5 unique values, based on specific quantiles of the training set.
All other inputs are set to constant values equal to their training set
medians.

\begin{Shaded}
\begin{Highlighting}[]
\NormalTok{make\_test\_input\_list }\OtherTok{\textless{}{-}} \ControlFlowTok{function}\NormalTok{(var\_name, top\_4\_inputs, all\_data)}
\NormalTok{\{}
\NormalTok{  xvar }\OtherTok{\textless{}{-}}\NormalTok{ all\_data }\SpecialCharTok{\%\textgreater{}\%} \FunctionTok{select}\NormalTok{(var\_name) }\SpecialCharTok{\%\textgreater{}\%} \FunctionTok{pull}\NormalTok{()}
  
  \ControlFlowTok{if}\NormalTok{ (var\_name }\SpecialCharTok{\%in\%}\NormalTok{ top\_4\_inputs[}\DecValTok{1}\SpecialCharTok{:}\DecValTok{2}\NormalTok{])\{}
    \CommentTok{\# use 25 unique values between the min/max values}
\NormalTok{    xgrid }\OtherTok{\textless{}{-}} \FunctionTok{seq}\NormalTok{(}\FunctionTok{min}\NormalTok{(xvar), }\FunctionTok{max}\NormalTok{(xvar), }\AttributeTok{length.out =} \DecValTok{25}\NormalTok{)}
\NormalTok{  \} }\ControlFlowTok{else} \ControlFlowTok{if}\NormalTok{ (var\_name }\SpecialCharTok{\%in\%}\NormalTok{ top\_4\_inputs[}\DecValTok{3}\SpecialCharTok{:}\DecValTok{4}\NormalTok{])\{}
    \CommentTok{\# specify quantiles to use}
\NormalTok{    xgrid }\OtherTok{\textless{}{-}} \FunctionTok{quantile}\NormalTok{(xvar, }\AttributeTok{probs =} \FunctionTok{c}\NormalTok{(}\FloatTok{0.05}\NormalTok{, }\FloatTok{0.25}\NormalTok{, }\FloatTok{0.5}\NormalTok{, }\FloatTok{0.75}\NormalTok{, }\FloatTok{0.95}\NormalTok{), }\AttributeTok{na.rm =} \ConstantTok{TRUE}\NormalTok{)}
\NormalTok{    xgrid }\OtherTok{\textless{}{-}} \FunctionTok{as.vector}\NormalTok{(xgrid)}
\NormalTok{  \} }\ControlFlowTok{else}\NormalTok{ \{}
    \CommentTok{\# set to their median values}
\NormalTok{    xgrid }\OtherTok{\textless{}{-}} \FunctionTok{median}\NormalTok{(xvar, }\AttributeTok{na.rm =} \ConstantTok{TRUE}\NormalTok{)}
\NormalTok{  \}}
  
  \FunctionTok{return}\NormalTok{(xgrid)}
\NormalTok{\}}
\end{Highlighting}
\end{Shaded}

The second function, \texttt{make\_test\_input\_grid()}, is a wrapper
function. It iteratively applies the \texttt{make\_test\_input\_list()}
function to all inputs and then uses the \texttt{expand.grid()} function
to create full factorial grid based on the list of unique input values.

\begin{Shaded}
\begin{Highlighting}[]
\NormalTok{make\_test\_input\_grid }\OtherTok{\textless{}{-}} \ControlFlowTok{function}\NormalTok{(all\_input\_names, top\_4\_inputs, all\_data)}
\NormalTok{\{}
\NormalTok{  test\_list }\OtherTok{\textless{}{-}}\NormalTok{ purrr}\SpecialCharTok{::}\FunctionTok{map}\NormalTok{(all\_input\_names, }
\NormalTok{                          make\_test\_input\_list,}
                          \AttributeTok{top\_4\_inputs =}\NormalTok{ top\_4\_inputs,}
                          \AttributeTok{all\_data =}\NormalTok{ all\_data)}
  
  \FunctionTok{expand.grid}\NormalTok{(test\_list, }
              \AttributeTok{KEEP.OUT.ATTRS =} \ConstantTok{FALSE}\NormalTok{,}
              \AttributeTok{stringsAsFactors =} \ConstantTok{FALSE}\NormalTok{) }\SpecialCharTok{\%\textgreater{}\%} 
\NormalTok{    purrr}\SpecialCharTok{::}\FunctionTok{set\_names}\NormalTok{(all\_input\_names)}
\NormalTok{\}}
\end{Highlighting}
\end{Shaded}

\textbf{You will use the \texttt{make\_test\_input\_grid()} function to
create a grid of input values based on the top 4 ranked inputs, as
viewed by your tuned neural network model.}

\textbf{Complete the code chunk below by creating two vectors. The
first, \texttt{my\_input\_names}, is a regular vector for the names of
the input variables in the \texttt{Sonar} data set. The second,
\texttt{my\_top\_ranked\_inputs}, is a regular vector for the names of
the top 4 ranked inputs based on your neural network model.}

\textbf{Call the \texttt{make\_test\_input\_grid()} function and assign
the result to \texttt{viz\_input\_grid}. Display the number of rows and
columns in \texttt{viz\_input\_grid} to the screen.}

\hypertarget{solution-10}{%
\paragraph{SOLUTION}\label{solution-10}}

Add as many code chunk as you feel necessary to complete this problem.

\begin{Shaded}
\begin{Highlighting}[]
\DocumentationTok{\#\#\# Initialize parameters}
\NormalTok{my\_input\_name }\OtherTok{\textless{}{-}} \FunctionTok{colnames}\NormalTok{(Sonar)[}\SpecialCharTok{{-}}\DecValTok{61}\NormalTok{]}
\NormalTok{my\_top\_ranked\_inputs }\OtherTok{\textless{}{-}} \FunctionTok{c}\NormalTok{(}\StringTok{"V50"}\NormalTok{,}\StringTok{"V57"}\NormalTok{,}\StringTok{"V31"}\NormalTok{,}\StringTok{"V22"}\NormalTok{)}

\DocumentationTok{\#\#\# Excecute function}
\NormalTok{viz\_input\_grid }\OtherTok{\textless{}{-}} \FunctionTok{make\_test\_input\_grid}\NormalTok{(my\_input\_name, my\_top\_ranked\_inputs, Sonar)}
\end{Highlighting}
\end{Shaded}

\begin{verbatim}
## Note: Using an external vector in selections is ambiguous.
## i Use `all_of(var_name)` instead of `var_name` to silence this message.
## i See <https://tidyselect.r-lib.org/reference/faq-external-vector.html>.
## This message is displayed once per session.
\end{verbatim}

\begin{Shaded}
\begin{Highlighting}[]
\DocumentationTok{\#\#\# Display result}
\FunctionTok{dim}\NormalTok{(viz\_input\_grid)}
\end{Highlighting}
\end{Shaded}

\begin{verbatim}
## [1] 15625    60
\end{verbatim}

\hypertarget{problem-03}{%
\subsection{Problem 03}\label{problem-03}}

We now have everything necessary to make predictions and study behavior.
We have a test or visualization grid and we have several models that we
can use to consider the influence between the inputs and the response.
Since we are working with a binary classification problem, we can view
the response two ways. First, we can consider the classification.
Second, we can visualize the predicted event probability.

\hypertarget{a-2}{%
\subsubsection{3a)}\label{a-2}}

We will first classify using the neural network model, and then
visualize those classifications as a surface plot. Predictions are made
with the \texttt{predict()} function. The first argument is the model
object and the second argument is the data set we wish to make
predictions with.

When we are working with binary classification models, the default
result of the \texttt{predict()} function is the class assuming a 50\%
threshold value.

\textbf{Make classifications on the test visualization grid using the
neural network model and assign the result to the
\texttt{pred\_class\_nnet} object. Check the data type of
\texttt{pred\_class\_nnet} and look at the first few classifications
with the \texttt{head()} function.}

\hypertarget{solution-11}{%
\paragraph{SOLUTION}\label{solution-11}}

\begin{Shaded}
\begin{Highlighting}[]
\NormalTok{pred\_class\_nnet }\OtherTok{\textless{}{-}} \FunctionTok{predict}\NormalTok{(fit\_nnet\_sonar, viz\_input\_grid)}
\FunctionTok{typeof}\NormalTok{(pred\_class\_nnet)}
\end{Highlighting}
\end{Shaded}

\begin{verbatim}
## [1] "integer"
\end{verbatim}

\begin{Shaded}
\begin{Highlighting}[]
\FunctionTok{head}\NormalTok{(pred\_class\_nnet)}
\end{Highlighting}
\end{Shaded}

\begin{verbatim}
## [1] M M M M M M
## Levels: M R
\end{verbatim}

\hypertarget{b-2}{%
\subsubsection{3b)}\label{b-2}}

You will now visualize the classification surface based on your neural
network model, over your defined test grid.

\textbf{Pipe \texttt{viz\_input\_grid} into a \texttt{mutate()} call and
assign the \texttt{pred\_class} variable equal to
\texttt{pred\_class\_nnet}. Pipe the result into \texttt{ggplot()} where
the \texttt{x} aesthetic is your top ranked input and the \texttt{y}
aesthetic is your second ranked input. Use a \texttt{geom\_raster()}
geom to visualize the surface by mapping the \texttt{fill} aesthetic to
\texttt{pred\_class}. Include \texttt{facet\_grid()} to break up the
prediction surfaces based on the third and fourth ranked inputs by
assigning the vertical facets to the 4th ranked input and the horizontal
facets to the 3rd ranked input. Specify the fill by including the
\texttt{scale\_fill\_brewer()} with the \texttt{palette} argument equal
to \texttt{\textquotesingle{}Set1\textquotesingle{}}.}

\hypertarget{solution-12}{%
\paragraph{SOLUTION}\label{solution-12}}

\begin{Shaded}
\begin{Highlighting}[]
\DocumentationTok{\#\#\#}
\NormalTok{viz\_input\_grid }\SpecialCharTok{\%\textgreater{}\%}
  \FunctionTok{mutate}\NormalTok{(}\AttributeTok{pred\_class =}\NormalTok{ pred\_class\_nnet) }\SpecialCharTok{\%\textgreater{}\%}
  \FunctionTok{ggplot}\NormalTok{(}\AttributeTok{mapping =} \FunctionTok{aes}\NormalTok{(}\AttributeTok{x =}\NormalTok{ V50, }\AttributeTok{y =}\NormalTok{ V57))}\SpecialCharTok{+}
  \FunctionTok{geom\_raster}\NormalTok{(}\AttributeTok{mapping =} \FunctionTok{aes}\NormalTok{(}\AttributeTok{fill =}\NormalTok{ pred\_class))}\SpecialCharTok{+}
  \FunctionTok{facet\_grid}\NormalTok{(}\AttributeTok{cols =} \FunctionTok{vars}\NormalTok{(V22), }\AttributeTok{rows =} \FunctionTok{vars}\NormalTok{(V31))}\SpecialCharTok{+}
  \FunctionTok{scale\_fill\_brewer}\NormalTok{(}\AttributeTok{palette =} \StringTok{\textquotesingle{}Set1\textquotesingle{}}\NormalTok{)}
\end{Highlighting}
\end{Shaded}

\includegraphics{Drake_Zhou_HW_11_files/figure-latex/solution_03b-1.pdf}

\hypertarget{c-2}{%
\subsubsection{3c)}\label{c-2}}

You will now make classifications with the elastic net model on the test
input grid.

\textbf{Predict the classifications with the elastic net model and
assign the result to the \texttt{pred\_class\_enet} object.}

\hypertarget{solution-13}{%
\paragraph{SOLUTION}\label{solution-13}}

\begin{Shaded}
\begin{Highlighting}[]
\DocumentationTok{\#\#\#}
\NormalTok{pred\_class\_enet }\OtherTok{\textless{}{-}} \FunctionTok{predict}\NormalTok{(fit\_glmnet\_sonar, viz\_input\_grid)}
\end{Highlighting}
\end{Shaded}

\hypertarget{d-2}{%
\subsubsection{3d)}\label{d-2}}

Visualize the classification surface based on the elastic net model.
Follow the same steps you used to make the surface with the neural
network classifications.

\textbf{How do the visualizations of the classification surface compare
with the neural network?}

\hypertarget{solution-14}{%
\paragraph{SOLUTION}\label{solution-14}}

Classification of elastic net model gives more M result compare to
neural network, which means elastic net model is more sensitive, at
least for these two parameters.

\begin{Shaded}
\begin{Highlighting}[]
\NormalTok{viz\_input\_grid }\SpecialCharTok{\%\textgreater{}\%}
  \FunctionTok{mutate}\NormalTok{(}\AttributeTok{pred\_class =}\NormalTok{ pred\_class\_enet) }\SpecialCharTok{\%\textgreater{}\%}
  \FunctionTok{ggplot}\NormalTok{(}\AttributeTok{mapping =} \FunctionTok{aes}\NormalTok{(}\AttributeTok{x =}\NormalTok{ V50, }\AttributeTok{y =}\NormalTok{ V57))}\SpecialCharTok{+}
  \FunctionTok{geom\_raster}\NormalTok{(}\AttributeTok{mapping =} \FunctionTok{aes}\NormalTok{(}\AttributeTok{fill =}\NormalTok{ pred\_class))}\SpecialCharTok{+}
  \FunctionTok{facet\_grid}\NormalTok{(}\AttributeTok{cols =} \FunctionTok{vars}\NormalTok{(V22), }\AttributeTok{rows =} \FunctionTok{vars}\NormalTok{(V31))}\SpecialCharTok{+}
  \FunctionTok{scale\_fill\_brewer}\NormalTok{(}\AttributeTok{palette =} \StringTok{\textquotesingle{}Set1\textquotesingle{}}\NormalTok{)}
\end{Highlighting}
\end{Shaded}

\includegraphics{Drake_Zhou_HW_11_files/figure-latex/solution_03d-1.pdf}

\hypertarget{e-2}{%
\subsubsection{3e)}\label{e-2}}

In addition to visualizing the classifications, we can also predict the
class probabilities. This allows us to test out our classification
threshold, if we would like. It also helps us understand if the
predictions are near the threshold boundary.

Class probability predictions are also made with the \texttt{predict()}
function. The first argument is still the model object, and the second
argument is still the data set we wish to predict. We must include a
third argument, \texttt{type}, which instructs the model the ``type'' of
prediction to make. Setting the \texttt{type} argument equal to
\texttt{\textquotesingle{}prob\textquotesingle{}} will return the class
probabilities, instead of the classification label.

\textbf{Predict the class probabilities with the neural network model.
Assign the result to the \texttt{pred\_prob\_nnet} object. Display the
data type of \texttt{pred\_prob\_nnet} to the screen, as well as the
result of the \texttt{head()} function applied to
\texttt{pred\_prob\_nnet}. How do you know how to access the probability
of the \texttt{\textquotesingle{}M\textquotesingle{}} class?}

\hypertarget{solution-15}{%
\paragraph{SOLUTION}\label{solution-15}}

How do you know how to access the probability of the
\texttt{\textquotesingle{}M\textquotesingle{}} class?

By pred\_prob\_nnet\$M

\begin{Shaded}
\begin{Highlighting}[]
\NormalTok{pred\_prob\_nnet }\OtherTok{\textless{}{-}} \FunctionTok{predict}\NormalTok{(fit\_nnet\_sonar, viz\_input\_grid, }\AttributeTok{type =} \StringTok{\textquotesingle{}prob\textquotesingle{}}\NormalTok{)}
\FunctionTok{typeof}\NormalTok{(pred\_prob\_nnet)}
\end{Highlighting}
\end{Shaded}

\begin{verbatim}
## [1] "list"
\end{verbatim}

\begin{Shaded}
\begin{Highlighting}[]
\FunctionTok{head}\NormalTok{(pred\_prob\_nnet)}
\end{Highlighting}
\end{Shaded}

\begin{verbatim}
##           M           R
## 1 0.9863185 0.013681453
## 2 0.9952580 0.004742048
## 3 0.9981842 0.001815811
## 4 0.9988205 0.001179527
## 5 0.9991026 0.000897417
## 6 0.9748333 0.025166724
\end{verbatim}

\hypertarget{f-1}{%
\subsubsection{3f)}\label{f-1}}

You will now visualize the predicted probability surface for the
\texttt{M} class.

\textbf{Bind the columns of \texttt{viz\_input\_grid} to the columns in
\texttt{pred\_prob\_nnet} using \texttt{bind\_cols()}. Pipe the result
into \texttt{ggplot()}. Map the \texttt{x} aesthetic to the top ranked
input and the \texttt{y} aesthetic to the second ranked input. Use a
\texttt{geom\_raster()} geom to visualize the predicted \texttt{M} class
surface by mapping the \texttt{fill} aesthetic to \texttt{M}. Include
\texttt{facet\_grid()} to break up the prediction surfaces based on the
third and fourth ranked inputs by assigning the vertical facets to the
4th ranked input and the horizontal facets to the 3rd ranked input.}

\textbf{Specify the fill to be a diverging fill scale by including
\texttt{scale\_fill\_gradient2()}. Set the \texttt{limits} argument to
\texttt{c(0,1)}. Set the \texttt{low} argument to
\texttt{\textquotesingle{}blue\textquotesingle{}}, the \texttt{high}
argument to \texttt{\textquotesingle{}red\textquotesingle{}}, and the
\texttt{mid} argument to
\texttt{\textquotesingle{}white\textquotesingle{}}. Set the
\texttt{midpoint} equal to \texttt{0.5}. These choices will give you
high predicted probabilities that are bright red, with low predicted
probabilities as bright blue and 50\% probabilities as white. You will
therefore be able to clearly see the 50\% threshold decision boundary!}

\hypertarget{solution-16}{%
\paragraph{SOLUTION}\label{solution-16}}

\begin{Shaded}
\begin{Highlighting}[]
\NormalTok{viz\_input\_grid }\SpecialCharTok{\%\textgreater{}\%}
  \FunctionTok{bind\_cols}\NormalTok{(pred\_prob\_nnet) }\SpecialCharTok{\%\textgreater{}\%}
  \FunctionTok{ggplot}\NormalTok{(}\AttributeTok{mapping =} \FunctionTok{aes}\NormalTok{(}\AttributeTok{x =}\NormalTok{ V50, }\AttributeTok{y =}\NormalTok{ V57))}\SpecialCharTok{+}
  \FunctionTok{geom\_raster}\NormalTok{(}\AttributeTok{mapping =} \FunctionTok{aes}\NormalTok{(}\AttributeTok{fill =}\NormalTok{ M))}\SpecialCharTok{+}
  \FunctionTok{facet\_grid}\NormalTok{(}\AttributeTok{rows =} \FunctionTok{vars}\NormalTok{(V31), }\AttributeTok{cols =} \FunctionTok{vars}\NormalTok{(V22))}\SpecialCharTok{+}
  \FunctionTok{scale\_fill\_gradient2}\NormalTok{(}\AttributeTok{limits =} \FunctionTok{c}\NormalTok{(}\DecValTok{0}\NormalTok{,}\DecValTok{1}\NormalTok{), }
                       \AttributeTok{low =} \StringTok{\textquotesingle{}blue\textquotesingle{}}\NormalTok{, }
                       \AttributeTok{high =} \StringTok{\textquotesingle{}red\textquotesingle{}}\NormalTok{, }
                       \AttributeTok{mid =} \StringTok{\textquotesingle{}white\textquotesingle{}}\NormalTok{, }
                       \AttributeTok{midpoint =} \FloatTok{0.5}\NormalTok{)}
\end{Highlighting}
\end{Shaded}

\includegraphics{Drake_Zhou_HW_11_files/figure-latex/solution_03f-1.pdf}

\hypertarget{problem-04}{%
\subsection{Problem 04}\label{problem-04}}

Let's now try out tree-based methods and compare their performance with
the elastic net and neural network models.

\hypertarget{a-3}{%
\subsubsection{4a)}\label{a-3}}

We could start with a standard decision tree with CART, but instead we
will jump straight to the random forest. To train a random forest model,
you must specify \texttt{method} equal to \texttt{"rf"}.\\
You can use the default tuning grid to the \texttt{mtry} tuning
parameter.

\textbf{Do you need to consider preprocessing the inputs for a random
forest?}

\textbf{Train a random forest binary classifier by setting the
\texttt{method} argument equal to \texttt{"rf"}. The code chunk below
includes the \texttt{importance=TRUE} argument for you.}

\textbf{What value of \texttt{mtry} was selected as the best, based on
the cross-validation results? Why do you think that value was selected?}

\hypertarget{solution-17}{%
\paragraph{SOLUTION}\label{solution-17}}

Do you need to consider preprocessing the inputs for a random forest?

No.~Because tree based model has nothing to do with preprocessing.

What value of \texttt{mtry} was selected as the best, based on the
cross-validation results? Why do you think that value was selected?

2 which lead to the highest score of ROC, Sens and Spec.

\begin{Shaded}
\begin{Highlighting}[]
\FunctionTok{set.seed}\NormalTok{(}\DecValTok{4321}\NormalTok{)}
\NormalTok{fit\_rf\_sonar }\OtherTok{\textless{}{-}} \FunctionTok{train}\NormalTok{(Class }\SpecialCharTok{\textasciitilde{}}\NormalTok{ ., }
                      \AttributeTok{data =}\NormalTok{ Sonar,}
                      \AttributeTok{method =} \StringTok{"rf"}\NormalTok{,}
                      \AttributeTok{metric =}\NormalTok{ metric\_sonar,}
                      \AttributeTok{trControl =}\NormalTok{ ctrl\_k05\_roc,}
                      \AttributeTok{importance =} \ConstantTok{TRUE}\NormalTok{)}

\NormalTok{fit\_rf\_sonar}
\end{Highlighting}
\end{Shaded}

\begin{verbatim}
## Random Forest 
## 
## 208 samples
##  60 predictor
##   2 classes: 'M', 'R' 
## 
## No pre-processing
## Resampling: Cross-Validated (5 fold) 
## Summary of sample sizes: 166, 167, 165, 167, 167 
## Resampling results across tuning parameters:
## 
##   mtry  ROC        Sens       Spec     
##    2    0.9374724  0.9193676  0.7636842
##   31    0.9045808  0.8557312  0.7326316
##   60    0.8909954  0.8557312  0.6705263
## 
## ROC was used to select the optimal model using the largest value.
## The final value used for the model was mtry = 2.
\end{verbatim}

\hypertarget{b-3}{%
\subsubsection{4b)}\label{b-3}}

Now try training a boosted tree model with \texttt{xgboost}. You must
set the \texttt{method} argument to \texttt{xgbTree} in order to tell
\texttt{caret} to use the XGBoost implementation of the boosted tree
algorithm. By default, many different tuning parameters are considered,
so instead of printing out the results, the best tuning parameters are
printed out for you. Then, the performance results are plotted for you.

\textbf{Train the XGBoost model with 5-fold cross-validation to maximize
the area under the ROC curve.}

\emph{NOTE}: The following code may take a few minutes to complete.

\hypertarget{solution-18}{%
\paragraph{SOLUTION}\label{solution-18}}

Train the model.

\begin{Shaded}
\begin{Highlighting}[]
\FunctionTok{set.seed}\NormalTok{(}\DecValTok{4321}\NormalTok{)}
\NormalTok{fit\_xgb\_sonar }\OtherTok{\textless{}{-}} \FunctionTok{train}\NormalTok{(Class }\SpecialCharTok{\textasciitilde{}}\NormalTok{.,}
                       \AttributeTok{data =}\NormalTok{ Sonar,}
                       \AttributeTok{method =} \StringTok{"xgbTree"}\NormalTok{,}
                       \AttributeTok{metric =}\NormalTok{ metric\_sonar,}
                       \AttributeTok{trControl =}\NormalTok{ ctrl\_k05\_roc)}
\end{Highlighting}
\end{Shaded}

\begin{verbatim}
## [23:12:23] WARNING: amalgamation/../src/c_api/c_api.cc:785: `ntree_limit` is deprecated, use `iteration_range` instead.
## [23:12:23] WARNING: amalgamation/../src/c_api/c_api.cc:785: `ntree_limit` is deprecated, use `iteration_range` instead.
## [23:12:23] WARNING: amalgamation/../src/c_api/c_api.cc:785: `ntree_limit` is deprecated, use `iteration_range` instead.
## [23:12:23] WARNING: amalgamation/../src/c_api/c_api.cc:785: `ntree_limit` is deprecated, use `iteration_range` instead.
## [23:12:23] WARNING: amalgamation/../src/c_api/c_api.cc:785: `ntree_limit` is deprecated, use `iteration_range` instead.
## [23:12:23] WARNING: amalgamation/../src/c_api/c_api.cc:785: `ntree_limit` is deprecated, use `iteration_range` instead.
## [23:12:23] WARNING: amalgamation/../src/c_api/c_api.cc:785: `ntree_limit` is deprecated, use `iteration_range` instead.
## [23:12:23] WARNING: amalgamation/../src/c_api/c_api.cc:785: `ntree_limit` is deprecated, use `iteration_range` instead.
## [23:12:23] WARNING: amalgamation/../src/c_api/c_api.cc:785: `ntree_limit` is deprecated, use `iteration_range` instead.
## [23:12:23] WARNING: amalgamation/../src/c_api/c_api.cc:785: `ntree_limit` is deprecated, use `iteration_range` instead.
## [23:12:23] WARNING: amalgamation/../src/c_api/c_api.cc:785: `ntree_limit` is deprecated, use `iteration_range` instead.
## [23:12:23] WARNING: amalgamation/../src/c_api/c_api.cc:785: `ntree_limit` is deprecated, use `iteration_range` instead.
## [23:12:23] WARNING: amalgamation/../src/c_api/c_api.cc:785: `ntree_limit` is deprecated, use `iteration_range` instead.
## [23:12:23] WARNING: amalgamation/../src/c_api/c_api.cc:785: `ntree_limit` is deprecated, use `iteration_range` instead.
## [23:12:23] WARNING: amalgamation/../src/c_api/c_api.cc:785: `ntree_limit` is deprecated, use `iteration_range` instead.
## [23:12:23] WARNING: amalgamation/../src/c_api/c_api.cc:785: `ntree_limit` is deprecated, use `iteration_range` instead.
## [23:12:23] WARNING: amalgamation/../src/c_api/c_api.cc:785: `ntree_limit` is deprecated, use `iteration_range` instead.
## [23:12:23] WARNING: amalgamation/../src/c_api/c_api.cc:785: `ntree_limit` is deprecated, use `iteration_range` instead.
## [23:12:23] WARNING: amalgamation/../src/c_api/c_api.cc:785: `ntree_limit` is deprecated, use `iteration_range` instead.
## [23:12:23] WARNING: amalgamation/../src/c_api/c_api.cc:785: `ntree_limit` is deprecated, use `iteration_range` instead.
## [23:12:23] WARNING: amalgamation/../src/c_api/c_api.cc:785: `ntree_limit` is deprecated, use `iteration_range` instead.
## [23:12:23] WARNING: amalgamation/../src/c_api/c_api.cc:785: `ntree_limit` is deprecated, use `iteration_range` instead.
## [23:12:23] WARNING: amalgamation/../src/c_api/c_api.cc:785: `ntree_limit` is deprecated, use `iteration_range` instead.
## [23:12:23] WARNING: amalgamation/../src/c_api/c_api.cc:785: `ntree_limit` is deprecated, use `iteration_range` instead.
## [23:12:23] WARNING: amalgamation/../src/c_api/c_api.cc:785: `ntree_limit` is deprecated, use `iteration_range` instead.
## [23:12:23] WARNING: amalgamation/../src/c_api/c_api.cc:785: `ntree_limit` is deprecated, use `iteration_range` instead.
## [23:12:23] WARNING: amalgamation/../src/c_api/c_api.cc:785: `ntree_limit` is deprecated, use `iteration_range` instead.
## [23:12:23] WARNING: amalgamation/../src/c_api/c_api.cc:785: `ntree_limit` is deprecated, use `iteration_range` instead.
## [23:12:23] WARNING: amalgamation/../src/c_api/c_api.cc:785: `ntree_limit` is deprecated, use `iteration_range` instead.
## [23:12:23] WARNING: amalgamation/../src/c_api/c_api.cc:785: `ntree_limit` is deprecated, use `iteration_range` instead.
## [23:12:23] WARNING: amalgamation/../src/c_api/c_api.cc:785: `ntree_limit` is deprecated, use `iteration_range` instead.
## [23:12:23] WARNING: amalgamation/../src/c_api/c_api.cc:785: `ntree_limit` is deprecated, use `iteration_range` instead.
## [23:12:23] WARNING: amalgamation/../src/c_api/c_api.cc:785: `ntree_limit` is deprecated, use `iteration_range` instead.
## [23:12:23] WARNING: amalgamation/../src/c_api/c_api.cc:785: `ntree_limit` is deprecated, use `iteration_range` instead.
## [23:12:23] WARNING: amalgamation/../src/c_api/c_api.cc:785: `ntree_limit` is deprecated, use `iteration_range` instead.
## [23:12:23] WARNING: amalgamation/../src/c_api/c_api.cc:785: `ntree_limit` is deprecated, use `iteration_range` instead.
## [23:12:24] WARNING: amalgamation/../src/c_api/c_api.cc:785: `ntree_limit` is deprecated, use `iteration_range` instead.
## [23:12:24] WARNING: amalgamation/../src/c_api/c_api.cc:785: `ntree_limit` is deprecated, use `iteration_range` instead.
## [23:12:24] WARNING: amalgamation/../src/c_api/c_api.cc:785: `ntree_limit` is deprecated, use `iteration_range` instead.
## [23:12:24] WARNING: amalgamation/../src/c_api/c_api.cc:785: `ntree_limit` is deprecated, use `iteration_range` instead.
## [23:12:24] WARNING: amalgamation/../src/c_api/c_api.cc:785: `ntree_limit` is deprecated, use `iteration_range` instead.
## [23:12:24] WARNING: amalgamation/../src/c_api/c_api.cc:785: `ntree_limit` is deprecated, use `iteration_range` instead.
## [23:12:24] WARNING: amalgamation/../src/c_api/c_api.cc:785: `ntree_limit` is deprecated, use `iteration_range` instead.
## [23:12:24] WARNING: amalgamation/../src/c_api/c_api.cc:785: `ntree_limit` is deprecated, use `iteration_range` instead.
## [23:12:24] WARNING: amalgamation/../src/c_api/c_api.cc:785: `ntree_limit` is deprecated, use `iteration_range` instead.
## [23:12:24] WARNING: amalgamation/../src/c_api/c_api.cc:785: `ntree_limit` is deprecated, use `iteration_range` instead.
## [23:12:24] WARNING: amalgamation/../src/c_api/c_api.cc:785: `ntree_limit` is deprecated, use `iteration_range` instead.
## [23:12:24] WARNING: amalgamation/../src/c_api/c_api.cc:785: `ntree_limit` is deprecated, use `iteration_range` instead.
## [23:12:24] WARNING: amalgamation/../src/c_api/c_api.cc:785: `ntree_limit` is deprecated, use `iteration_range` instead.
## [23:12:24] WARNING: amalgamation/../src/c_api/c_api.cc:785: `ntree_limit` is deprecated, use `iteration_range` instead.
## [23:12:24] WARNING: amalgamation/../src/c_api/c_api.cc:785: `ntree_limit` is deprecated, use `iteration_range` instead.
## [23:12:24] WARNING: amalgamation/../src/c_api/c_api.cc:785: `ntree_limit` is deprecated, use `iteration_range` instead.
## [23:12:24] WARNING: amalgamation/../src/c_api/c_api.cc:785: `ntree_limit` is deprecated, use `iteration_range` instead.
## [23:12:24] WARNING: amalgamation/../src/c_api/c_api.cc:785: `ntree_limit` is deprecated, use `iteration_range` instead.
## [23:12:24] WARNING: amalgamation/../src/c_api/c_api.cc:785: `ntree_limit` is deprecated, use `iteration_range` instead.
## [23:12:24] WARNING: amalgamation/../src/c_api/c_api.cc:785: `ntree_limit` is deprecated, use `iteration_range` instead.
## [23:12:24] WARNING: amalgamation/../src/c_api/c_api.cc:785: `ntree_limit` is deprecated, use `iteration_range` instead.
## [23:12:24] WARNING: amalgamation/../src/c_api/c_api.cc:785: `ntree_limit` is deprecated, use `iteration_range` instead.
## [23:12:24] WARNING: amalgamation/../src/c_api/c_api.cc:785: `ntree_limit` is deprecated, use `iteration_range` instead.
## [23:12:24] WARNING: amalgamation/../src/c_api/c_api.cc:785: `ntree_limit` is deprecated, use `iteration_range` instead.
## [23:12:24] WARNING: amalgamation/../src/c_api/c_api.cc:785: `ntree_limit` is deprecated, use `iteration_range` instead.
## [23:12:24] WARNING: amalgamation/../src/c_api/c_api.cc:785: `ntree_limit` is deprecated, use `iteration_range` instead.
## [23:12:24] WARNING: amalgamation/../src/c_api/c_api.cc:785: `ntree_limit` is deprecated, use `iteration_range` instead.
## [23:12:24] WARNING: amalgamation/../src/c_api/c_api.cc:785: `ntree_limit` is deprecated, use `iteration_range` instead.
## [23:12:24] WARNING: amalgamation/../src/c_api/c_api.cc:785: `ntree_limit` is deprecated, use `iteration_range` instead.
## [23:12:24] WARNING: amalgamation/../src/c_api/c_api.cc:785: `ntree_limit` is deprecated, use `iteration_range` instead.
## [23:12:24] WARNING: amalgamation/../src/c_api/c_api.cc:785: `ntree_limit` is deprecated, use `iteration_range` instead.
## [23:12:24] WARNING: amalgamation/../src/c_api/c_api.cc:785: `ntree_limit` is deprecated, use `iteration_range` instead.
## [23:12:24] WARNING: amalgamation/../src/c_api/c_api.cc:785: `ntree_limit` is deprecated, use `iteration_range` instead.
## [23:12:24] WARNING: amalgamation/../src/c_api/c_api.cc:785: `ntree_limit` is deprecated, use `iteration_range` instead.
## [23:12:24] WARNING: amalgamation/../src/c_api/c_api.cc:785: `ntree_limit` is deprecated, use `iteration_range` instead.
## [23:12:24] WARNING: amalgamation/../src/c_api/c_api.cc:785: `ntree_limit` is deprecated, use `iteration_range` instead.
## [23:12:24] WARNING: amalgamation/../src/c_api/c_api.cc:785: `ntree_limit` is deprecated, use `iteration_range` instead.
## [23:12:24] WARNING: amalgamation/../src/c_api/c_api.cc:785: `ntree_limit` is deprecated, use `iteration_range` instead.
## [23:12:24] WARNING: amalgamation/../src/c_api/c_api.cc:785: `ntree_limit` is deprecated, use `iteration_range` instead.
## [23:12:24] WARNING: amalgamation/../src/c_api/c_api.cc:785: `ntree_limit` is deprecated, use `iteration_range` instead.
## [23:12:24] WARNING: amalgamation/../src/c_api/c_api.cc:785: `ntree_limit` is deprecated, use `iteration_range` instead.
## [23:12:24] WARNING: amalgamation/../src/c_api/c_api.cc:785: `ntree_limit` is deprecated, use `iteration_range` instead.
## [23:12:24] WARNING: amalgamation/../src/c_api/c_api.cc:785: `ntree_limit` is deprecated, use `iteration_range` instead.
## [23:12:24] WARNING: amalgamation/../src/c_api/c_api.cc:785: `ntree_limit` is deprecated, use `iteration_range` instead.
## [23:12:24] WARNING: amalgamation/../src/c_api/c_api.cc:785: `ntree_limit` is deprecated, use `iteration_range` instead.
## [23:12:24] WARNING: amalgamation/../src/c_api/c_api.cc:785: `ntree_limit` is deprecated, use `iteration_range` instead.
## [23:12:24] WARNING: amalgamation/../src/c_api/c_api.cc:785: `ntree_limit` is deprecated, use `iteration_range` instead.
## [23:12:24] WARNING: amalgamation/../src/c_api/c_api.cc:785: `ntree_limit` is deprecated, use `iteration_range` instead.
## [23:12:24] WARNING: amalgamation/../src/c_api/c_api.cc:785: `ntree_limit` is deprecated, use `iteration_range` instead.
## [23:12:24] WARNING: amalgamation/../src/c_api/c_api.cc:785: `ntree_limit` is deprecated, use `iteration_range` instead.
## [23:12:24] WARNING: amalgamation/../src/c_api/c_api.cc:785: `ntree_limit` is deprecated, use `iteration_range` instead.
## [23:12:24] WARNING: amalgamation/../src/c_api/c_api.cc:785: `ntree_limit` is deprecated, use `iteration_range` instead.
## [23:12:24] WARNING: amalgamation/../src/c_api/c_api.cc:785: `ntree_limit` is deprecated, use `iteration_range` instead.
## [23:12:24] WARNING: amalgamation/../src/c_api/c_api.cc:785: `ntree_limit` is deprecated, use `iteration_range` instead.
## [23:12:24] WARNING: amalgamation/../src/c_api/c_api.cc:785: `ntree_limit` is deprecated, use `iteration_range` instead.
## [23:12:24] WARNING: amalgamation/../src/c_api/c_api.cc:785: `ntree_limit` is deprecated, use `iteration_range` instead.
## [23:12:24] WARNING: amalgamation/../src/c_api/c_api.cc:785: `ntree_limit` is deprecated, use `iteration_range` instead.
## [23:12:24] WARNING: amalgamation/../src/c_api/c_api.cc:785: `ntree_limit` is deprecated, use `iteration_range` instead.
## [23:12:24] WARNING: amalgamation/../src/c_api/c_api.cc:785: `ntree_limit` is deprecated, use `iteration_range` instead.
## [23:12:24] WARNING: amalgamation/../src/c_api/c_api.cc:785: `ntree_limit` is deprecated, use `iteration_range` instead.
## [23:12:24] WARNING: amalgamation/../src/c_api/c_api.cc:785: `ntree_limit` is deprecated, use `iteration_range` instead.
## [23:12:24] WARNING: amalgamation/../src/c_api/c_api.cc:785: `ntree_limit` is deprecated, use `iteration_range` instead.
## [23:12:24] WARNING: amalgamation/../src/c_api/c_api.cc:785: `ntree_limit` is deprecated, use `iteration_range` instead.
## [23:12:24] WARNING: amalgamation/../src/c_api/c_api.cc:785: `ntree_limit` is deprecated, use `iteration_range` instead.
## [23:12:24] WARNING: amalgamation/../src/c_api/c_api.cc:785: `ntree_limit` is deprecated, use `iteration_range` instead.
## [23:12:24] WARNING: amalgamation/../src/c_api/c_api.cc:785: `ntree_limit` is deprecated, use `iteration_range` instead.
## [23:12:24] WARNING: amalgamation/../src/c_api/c_api.cc:785: `ntree_limit` is deprecated, use `iteration_range` instead.
## [23:12:24] WARNING: amalgamation/../src/c_api/c_api.cc:785: `ntree_limit` is deprecated, use `iteration_range` instead.
## [23:12:24] WARNING: amalgamation/../src/c_api/c_api.cc:785: `ntree_limit` is deprecated, use `iteration_range` instead.
## [23:12:24] WARNING: amalgamation/../src/c_api/c_api.cc:785: `ntree_limit` is deprecated, use `iteration_range` instead.
## [23:12:24] WARNING: amalgamation/../src/c_api/c_api.cc:785: `ntree_limit` is deprecated, use `iteration_range` instead.
## [23:12:24] WARNING: amalgamation/../src/c_api/c_api.cc:785: `ntree_limit` is deprecated, use `iteration_range` instead.
## [23:12:24] WARNING: amalgamation/../src/c_api/c_api.cc:785: `ntree_limit` is deprecated, use `iteration_range` instead.
## [23:12:24] WARNING: amalgamation/../src/c_api/c_api.cc:785: `ntree_limit` is deprecated, use `iteration_range` instead.
## [23:12:24] WARNING: amalgamation/../src/c_api/c_api.cc:785: `ntree_limit` is deprecated, use `iteration_range` instead.
## [23:12:24] WARNING: amalgamation/../src/c_api/c_api.cc:785: `ntree_limit` is deprecated, use `iteration_range` instead.
## [23:12:24] WARNING: amalgamation/../src/c_api/c_api.cc:785: `ntree_limit` is deprecated, use `iteration_range` instead.
## [23:12:24] WARNING: amalgamation/../src/c_api/c_api.cc:785: `ntree_limit` is deprecated, use `iteration_range` instead.
## [23:12:24] WARNING: amalgamation/../src/c_api/c_api.cc:785: `ntree_limit` is deprecated, use `iteration_range` instead.
## [23:12:24] WARNING: amalgamation/../src/c_api/c_api.cc:785: `ntree_limit` is deprecated, use `iteration_range` instead.
## [23:12:24] WARNING: amalgamation/../src/c_api/c_api.cc:785: `ntree_limit` is deprecated, use `iteration_range` instead.
## [23:12:24] WARNING: amalgamation/../src/c_api/c_api.cc:785: `ntree_limit` is deprecated, use `iteration_range` instead.
## [23:12:24] WARNING: amalgamation/../src/c_api/c_api.cc:785: `ntree_limit` is deprecated, use `iteration_range` instead.
## [23:12:24] WARNING: amalgamation/../src/c_api/c_api.cc:785: `ntree_limit` is deprecated, use `iteration_range` instead.
## [23:12:24] WARNING: amalgamation/../src/c_api/c_api.cc:785: `ntree_limit` is deprecated, use `iteration_range` instead.
## [23:12:24] WARNING: amalgamation/../src/c_api/c_api.cc:785: `ntree_limit` is deprecated, use `iteration_range` instead.
## [23:12:24] WARNING: amalgamation/../src/c_api/c_api.cc:785: `ntree_limit` is deprecated, use `iteration_range` instead.
## [23:12:24] WARNING: amalgamation/../src/c_api/c_api.cc:785: `ntree_limit` is deprecated, use `iteration_range` instead.
## [23:12:24] WARNING: amalgamation/../src/c_api/c_api.cc:785: `ntree_limit` is deprecated, use `iteration_range` instead.
## [23:12:24] WARNING: amalgamation/../src/c_api/c_api.cc:785: `ntree_limit` is deprecated, use `iteration_range` instead.
## [23:12:24] WARNING: amalgamation/../src/c_api/c_api.cc:785: `ntree_limit` is deprecated, use `iteration_range` instead.
## [23:12:24] WARNING: amalgamation/../src/c_api/c_api.cc:785: `ntree_limit` is deprecated, use `iteration_range` instead.
## [23:12:24] WARNING: amalgamation/../src/c_api/c_api.cc:785: `ntree_limit` is deprecated, use `iteration_range` instead.
## [23:12:24] WARNING: amalgamation/../src/c_api/c_api.cc:785: `ntree_limit` is deprecated, use `iteration_range` instead.
## [23:12:24] WARNING: amalgamation/../src/c_api/c_api.cc:785: `ntree_limit` is deprecated, use `iteration_range` instead.
## [23:12:24] WARNING: amalgamation/../src/c_api/c_api.cc:785: `ntree_limit` is deprecated, use `iteration_range` instead.
## [23:12:24] WARNING: amalgamation/../src/c_api/c_api.cc:785: `ntree_limit` is deprecated, use `iteration_range` instead.
## [23:12:24] WARNING: amalgamation/../src/c_api/c_api.cc:785: `ntree_limit` is deprecated, use `iteration_range` instead.
## [23:12:24] WARNING: amalgamation/../src/c_api/c_api.cc:785: `ntree_limit` is deprecated, use `iteration_range` instead.
## [23:12:24] WARNING: amalgamation/../src/c_api/c_api.cc:785: `ntree_limit` is deprecated, use `iteration_range` instead.
## [23:12:24] WARNING: amalgamation/../src/c_api/c_api.cc:785: `ntree_limit` is deprecated, use `iteration_range` instead.
## [23:12:24] WARNING: amalgamation/../src/c_api/c_api.cc:785: `ntree_limit` is deprecated, use `iteration_range` instead.
## [23:12:24] WARNING: amalgamation/../src/c_api/c_api.cc:785: `ntree_limit` is deprecated, use `iteration_range` instead.
## [23:12:24] WARNING: amalgamation/../src/c_api/c_api.cc:785: `ntree_limit` is deprecated, use `iteration_range` instead.
## [23:12:24] WARNING: amalgamation/../src/c_api/c_api.cc:785: `ntree_limit` is deprecated, use `iteration_range` instead.
## [23:12:24] WARNING: amalgamation/../src/c_api/c_api.cc:785: `ntree_limit` is deprecated, use `iteration_range` instead.
## [23:12:24] WARNING: amalgamation/../src/c_api/c_api.cc:785: `ntree_limit` is deprecated, use `iteration_range` instead.
## [23:12:24] WARNING: amalgamation/../src/c_api/c_api.cc:785: `ntree_limit` is deprecated, use `iteration_range` instead.
## [23:12:24] WARNING: amalgamation/../src/c_api/c_api.cc:785: `ntree_limit` is deprecated, use `iteration_range` instead.
## [23:12:24] WARNING: amalgamation/../src/c_api/c_api.cc:785: `ntree_limit` is deprecated, use `iteration_range` instead.
## [23:12:24] WARNING: amalgamation/../src/c_api/c_api.cc:785: `ntree_limit` is deprecated, use `iteration_range` instead.
## [23:12:24] WARNING: amalgamation/../src/c_api/c_api.cc:785: `ntree_limit` is deprecated, use `iteration_range` instead.
## [23:12:24] WARNING: amalgamation/../src/c_api/c_api.cc:785: `ntree_limit` is deprecated, use `iteration_range` instead.
## [23:12:24] WARNING: amalgamation/../src/c_api/c_api.cc:785: `ntree_limit` is deprecated, use `iteration_range` instead.
## [23:12:24] WARNING: amalgamation/../src/c_api/c_api.cc:785: `ntree_limit` is deprecated, use `iteration_range` instead.
## [23:12:24] WARNING: amalgamation/../src/c_api/c_api.cc:785: `ntree_limit` is deprecated, use `iteration_range` instead.
## [23:12:24] WARNING: amalgamation/../src/c_api/c_api.cc:785: `ntree_limit` is deprecated, use `iteration_range` instead.
## [23:12:24] WARNING: amalgamation/../src/c_api/c_api.cc:785: `ntree_limit` is deprecated, use `iteration_range` instead.
## [23:12:24] WARNING: amalgamation/../src/c_api/c_api.cc:785: `ntree_limit` is deprecated, use `iteration_range` instead.
## [23:12:24] WARNING: amalgamation/../src/c_api/c_api.cc:785: `ntree_limit` is deprecated, use `iteration_range` instead.
## [23:12:24] WARNING: amalgamation/../src/c_api/c_api.cc:785: `ntree_limit` is deprecated, use `iteration_range` instead.
## [23:12:24] WARNING: amalgamation/../src/c_api/c_api.cc:785: `ntree_limit` is deprecated, use `iteration_range` instead.
## [23:12:24] WARNING: amalgamation/../src/c_api/c_api.cc:785: `ntree_limit` is deprecated, use `iteration_range` instead.
## [23:12:24] WARNING: amalgamation/../src/c_api/c_api.cc:785: `ntree_limit` is deprecated, use `iteration_range` instead.
## [23:12:24] WARNING: amalgamation/../src/c_api/c_api.cc:785: `ntree_limit` is deprecated, use `iteration_range` instead.
## [23:12:24] WARNING: amalgamation/../src/c_api/c_api.cc:785: `ntree_limit` is deprecated, use `iteration_range` instead.
## [23:12:24] WARNING: amalgamation/../src/c_api/c_api.cc:785: `ntree_limit` is deprecated, use `iteration_range` instead.
## [23:12:24] WARNING: amalgamation/../src/c_api/c_api.cc:785: `ntree_limit` is deprecated, use `iteration_range` instead.
## [23:12:24] WARNING: amalgamation/../src/c_api/c_api.cc:785: `ntree_limit` is deprecated, use `iteration_range` instead.
## [23:12:24] WARNING: amalgamation/../src/c_api/c_api.cc:785: `ntree_limit` is deprecated, use `iteration_range` instead.
## [23:12:24] WARNING: amalgamation/../src/c_api/c_api.cc:785: `ntree_limit` is deprecated, use `iteration_range` instead.
## [23:12:24] WARNING: amalgamation/../src/c_api/c_api.cc:785: `ntree_limit` is deprecated, use `iteration_range` instead.
## [23:12:25] WARNING: amalgamation/../src/c_api/c_api.cc:785: `ntree_limit` is deprecated, use `iteration_range` instead.
## [23:12:25] WARNING: amalgamation/../src/c_api/c_api.cc:785: `ntree_limit` is deprecated, use `iteration_range` instead.
## [23:12:25] WARNING: amalgamation/../src/c_api/c_api.cc:785: `ntree_limit` is deprecated, use `iteration_range` instead.
## [23:12:25] WARNING: amalgamation/../src/c_api/c_api.cc:785: `ntree_limit` is deprecated, use `iteration_range` instead.
## [23:12:25] WARNING: amalgamation/../src/c_api/c_api.cc:785: `ntree_limit` is deprecated, use `iteration_range` instead.
## [23:12:25] WARNING: amalgamation/../src/c_api/c_api.cc:785: `ntree_limit` is deprecated, use `iteration_range` instead.
## [23:12:25] WARNING: amalgamation/../src/c_api/c_api.cc:785: `ntree_limit` is deprecated, use `iteration_range` instead.
## [23:12:25] WARNING: amalgamation/../src/c_api/c_api.cc:785: `ntree_limit` is deprecated, use `iteration_range` instead.
## [23:12:25] WARNING: amalgamation/../src/c_api/c_api.cc:785: `ntree_limit` is deprecated, use `iteration_range` instead.
## [23:12:25] WARNING: amalgamation/../src/c_api/c_api.cc:785: `ntree_limit` is deprecated, use `iteration_range` instead.
## [23:12:25] WARNING: amalgamation/../src/c_api/c_api.cc:785: `ntree_limit` is deprecated, use `iteration_range` instead.
## [23:12:25] WARNING: amalgamation/../src/c_api/c_api.cc:785: `ntree_limit` is deprecated, use `iteration_range` instead.
## [23:12:25] WARNING: amalgamation/../src/c_api/c_api.cc:785: `ntree_limit` is deprecated, use `iteration_range` instead.
## [23:12:25] WARNING: amalgamation/../src/c_api/c_api.cc:785: `ntree_limit` is deprecated, use `iteration_range` instead.
## [23:12:25] WARNING: amalgamation/../src/c_api/c_api.cc:785: `ntree_limit` is deprecated, use `iteration_range` instead.
## [23:12:25] WARNING: amalgamation/../src/c_api/c_api.cc:785: `ntree_limit` is deprecated, use `iteration_range` instead.
## [23:12:25] WARNING: amalgamation/../src/c_api/c_api.cc:785: `ntree_limit` is deprecated, use `iteration_range` instead.
## [23:12:25] WARNING: amalgamation/../src/c_api/c_api.cc:785: `ntree_limit` is deprecated, use `iteration_range` instead.
## [23:12:25] WARNING: amalgamation/../src/c_api/c_api.cc:785: `ntree_limit` is deprecated, use `iteration_range` instead.
## [23:12:25] WARNING: amalgamation/../src/c_api/c_api.cc:785: `ntree_limit` is deprecated, use `iteration_range` instead.
## [23:12:25] WARNING: amalgamation/../src/c_api/c_api.cc:785: `ntree_limit` is deprecated, use `iteration_range` instead.
## [23:12:25] WARNING: amalgamation/../src/c_api/c_api.cc:785: `ntree_limit` is deprecated, use `iteration_range` instead.
## [23:12:25] WARNING: amalgamation/../src/c_api/c_api.cc:785: `ntree_limit` is deprecated, use `iteration_range` instead.
## [23:12:25] WARNING: amalgamation/../src/c_api/c_api.cc:785: `ntree_limit` is deprecated, use `iteration_range` instead.
## [23:12:25] WARNING: amalgamation/../src/c_api/c_api.cc:785: `ntree_limit` is deprecated, use `iteration_range` instead.
## [23:12:25] WARNING: amalgamation/../src/c_api/c_api.cc:785: `ntree_limit` is deprecated, use `iteration_range` instead.
## [23:12:25] WARNING: amalgamation/../src/c_api/c_api.cc:785: `ntree_limit` is deprecated, use `iteration_range` instead.
## [23:12:25] WARNING: amalgamation/../src/c_api/c_api.cc:785: `ntree_limit` is deprecated, use `iteration_range` instead.
## [23:12:25] WARNING: amalgamation/../src/c_api/c_api.cc:785: `ntree_limit` is deprecated, use `iteration_range` instead.
## [23:12:25] WARNING: amalgamation/../src/c_api/c_api.cc:785: `ntree_limit` is deprecated, use `iteration_range` instead.
## [23:12:25] WARNING: amalgamation/../src/c_api/c_api.cc:785: `ntree_limit` is deprecated, use `iteration_range` instead.
## [23:12:25] WARNING: amalgamation/../src/c_api/c_api.cc:785: `ntree_limit` is deprecated, use `iteration_range` instead.
## [23:12:25] WARNING: amalgamation/../src/c_api/c_api.cc:785: `ntree_limit` is deprecated, use `iteration_range` instead.
## [23:12:25] WARNING: amalgamation/../src/c_api/c_api.cc:785: `ntree_limit` is deprecated, use `iteration_range` instead.
## [23:12:25] WARNING: amalgamation/../src/c_api/c_api.cc:785: `ntree_limit` is deprecated, use `iteration_range` instead.
## [23:12:25] WARNING: amalgamation/../src/c_api/c_api.cc:785: `ntree_limit` is deprecated, use `iteration_range` instead.
## [23:12:25] WARNING: amalgamation/../src/c_api/c_api.cc:785: `ntree_limit` is deprecated, use `iteration_range` instead.
## [23:12:25] WARNING: amalgamation/../src/c_api/c_api.cc:785: `ntree_limit` is deprecated, use `iteration_range` instead.
## [23:12:25] WARNING: amalgamation/../src/c_api/c_api.cc:785: `ntree_limit` is deprecated, use `iteration_range` instead.
## [23:12:25] WARNING: amalgamation/../src/c_api/c_api.cc:785: `ntree_limit` is deprecated, use `iteration_range` instead.
## [23:12:25] WARNING: amalgamation/../src/c_api/c_api.cc:785: `ntree_limit` is deprecated, use `iteration_range` instead.
## [23:12:25] WARNING: amalgamation/../src/c_api/c_api.cc:785: `ntree_limit` is deprecated, use `iteration_range` instead.
## [23:12:25] WARNING: amalgamation/../src/c_api/c_api.cc:785: `ntree_limit` is deprecated, use `iteration_range` instead.
## [23:12:25] WARNING: amalgamation/../src/c_api/c_api.cc:785: `ntree_limit` is deprecated, use `iteration_range` instead.
## [23:12:25] WARNING: amalgamation/../src/c_api/c_api.cc:785: `ntree_limit` is deprecated, use `iteration_range` instead.
## [23:12:25] WARNING: amalgamation/../src/c_api/c_api.cc:785: `ntree_limit` is deprecated, use `iteration_range` instead.
## [23:12:25] WARNING: amalgamation/../src/c_api/c_api.cc:785: `ntree_limit` is deprecated, use `iteration_range` instead.
## [23:12:25] WARNING: amalgamation/../src/c_api/c_api.cc:785: `ntree_limit` is deprecated, use `iteration_range` instead.
## [23:12:25] WARNING: amalgamation/../src/c_api/c_api.cc:785: `ntree_limit` is deprecated, use `iteration_range` instead.
## [23:12:25] WARNING: amalgamation/../src/c_api/c_api.cc:785: `ntree_limit` is deprecated, use `iteration_range` instead.
## [23:12:25] WARNING: amalgamation/../src/c_api/c_api.cc:785: `ntree_limit` is deprecated, use `iteration_range` instead.
## [23:12:25] WARNING: amalgamation/../src/c_api/c_api.cc:785: `ntree_limit` is deprecated, use `iteration_range` instead.
## [23:12:25] WARNING: amalgamation/../src/c_api/c_api.cc:785: `ntree_limit` is deprecated, use `iteration_range` instead.
## [23:12:25] WARNING: amalgamation/../src/c_api/c_api.cc:785: `ntree_limit` is deprecated, use `iteration_range` instead.
## [23:12:25] WARNING: amalgamation/../src/c_api/c_api.cc:785: `ntree_limit` is deprecated, use `iteration_range` instead.
## [23:12:25] WARNING: amalgamation/../src/c_api/c_api.cc:785: `ntree_limit` is deprecated, use `iteration_range` instead.
## [23:12:25] WARNING: amalgamation/../src/c_api/c_api.cc:785: `ntree_limit` is deprecated, use `iteration_range` instead.
## [23:12:25] WARNING: amalgamation/../src/c_api/c_api.cc:785: `ntree_limit` is deprecated, use `iteration_range` instead.
## [23:12:25] WARNING: amalgamation/../src/c_api/c_api.cc:785: `ntree_limit` is deprecated, use `iteration_range` instead.
## [23:12:25] WARNING: amalgamation/../src/c_api/c_api.cc:785: `ntree_limit` is deprecated, use `iteration_range` instead.
## [23:12:25] WARNING: amalgamation/../src/c_api/c_api.cc:785: `ntree_limit` is deprecated, use `iteration_range` instead.
## [23:12:25] WARNING: amalgamation/../src/c_api/c_api.cc:785: `ntree_limit` is deprecated, use `iteration_range` instead.
## [23:12:25] WARNING: amalgamation/../src/c_api/c_api.cc:785: `ntree_limit` is deprecated, use `iteration_range` instead.
## [23:12:25] WARNING: amalgamation/../src/c_api/c_api.cc:785: `ntree_limit` is deprecated, use `iteration_range` instead.
## [23:12:25] WARNING: amalgamation/../src/c_api/c_api.cc:785: `ntree_limit` is deprecated, use `iteration_range` instead.
## [23:12:25] WARNING: amalgamation/../src/c_api/c_api.cc:785: `ntree_limit` is deprecated, use `iteration_range` instead.
## [23:12:25] WARNING: amalgamation/../src/c_api/c_api.cc:785: `ntree_limit` is deprecated, use `iteration_range` instead.
## [23:12:25] WARNING: amalgamation/../src/c_api/c_api.cc:785: `ntree_limit` is deprecated, use `iteration_range` instead.
## [23:12:25] WARNING: amalgamation/../src/c_api/c_api.cc:785: `ntree_limit` is deprecated, use `iteration_range` instead.
## [23:12:25] WARNING: amalgamation/../src/c_api/c_api.cc:785: `ntree_limit` is deprecated, use `iteration_range` instead.
## [23:12:25] WARNING: amalgamation/../src/c_api/c_api.cc:785: `ntree_limit` is deprecated, use `iteration_range` instead.
## [23:12:25] WARNING: amalgamation/../src/c_api/c_api.cc:785: `ntree_limit` is deprecated, use `iteration_range` instead.
## [23:12:25] WARNING: amalgamation/../src/c_api/c_api.cc:785: `ntree_limit` is deprecated, use `iteration_range` instead.
## [23:12:25] WARNING: amalgamation/../src/c_api/c_api.cc:785: `ntree_limit` is deprecated, use `iteration_range` instead.
## [23:12:25] WARNING: amalgamation/../src/c_api/c_api.cc:785: `ntree_limit` is deprecated, use `iteration_range` instead.
## [23:12:25] WARNING: amalgamation/../src/c_api/c_api.cc:785: `ntree_limit` is deprecated, use `iteration_range` instead.
## [23:12:25] WARNING: amalgamation/../src/c_api/c_api.cc:785: `ntree_limit` is deprecated, use `iteration_range` instead.
## [23:12:25] WARNING: amalgamation/../src/c_api/c_api.cc:785: `ntree_limit` is deprecated, use `iteration_range` instead.
## [23:12:25] WARNING: amalgamation/../src/c_api/c_api.cc:785: `ntree_limit` is deprecated, use `iteration_range` instead.
## [23:12:25] WARNING: amalgamation/../src/c_api/c_api.cc:785: `ntree_limit` is deprecated, use `iteration_range` instead.
## [23:12:25] WARNING: amalgamation/../src/c_api/c_api.cc:785: `ntree_limit` is deprecated, use `iteration_range` instead.
## [23:12:25] WARNING: amalgamation/../src/c_api/c_api.cc:785: `ntree_limit` is deprecated, use `iteration_range` instead.
## [23:12:25] WARNING: amalgamation/../src/c_api/c_api.cc:785: `ntree_limit` is deprecated, use `iteration_range` instead.
## [23:12:25] WARNING: amalgamation/../src/c_api/c_api.cc:785: `ntree_limit` is deprecated, use `iteration_range` instead.
## [23:12:25] WARNING: amalgamation/../src/c_api/c_api.cc:785: `ntree_limit` is deprecated, use `iteration_range` instead.
## [23:12:25] WARNING: amalgamation/../src/c_api/c_api.cc:785: `ntree_limit` is deprecated, use `iteration_range` instead.
## [23:12:25] WARNING: amalgamation/../src/c_api/c_api.cc:785: `ntree_limit` is deprecated, use `iteration_range` instead.
## [23:12:25] WARNING: amalgamation/../src/c_api/c_api.cc:785: `ntree_limit` is deprecated, use `iteration_range` instead.
## [23:12:25] WARNING: amalgamation/../src/c_api/c_api.cc:785: `ntree_limit` is deprecated, use `iteration_range` instead.
## [23:12:25] WARNING: amalgamation/../src/c_api/c_api.cc:785: `ntree_limit` is deprecated, use `iteration_range` instead.
## [23:12:25] WARNING: amalgamation/../src/c_api/c_api.cc:785: `ntree_limit` is deprecated, use `iteration_range` instead.
## [23:12:25] WARNING: amalgamation/../src/c_api/c_api.cc:785: `ntree_limit` is deprecated, use `iteration_range` instead.
## [23:12:25] WARNING: amalgamation/../src/c_api/c_api.cc:785: `ntree_limit` is deprecated, use `iteration_range` instead.
## [23:12:25] WARNING: amalgamation/../src/c_api/c_api.cc:785: `ntree_limit` is deprecated, use `iteration_range` instead.
## [23:12:25] WARNING: amalgamation/../src/c_api/c_api.cc:785: `ntree_limit` is deprecated, use `iteration_range` instead.
## [23:12:25] WARNING: amalgamation/../src/c_api/c_api.cc:785: `ntree_limit` is deprecated, use `iteration_range` instead.
## [23:12:25] WARNING: amalgamation/../src/c_api/c_api.cc:785: `ntree_limit` is deprecated, use `iteration_range` instead.
## [23:12:25] WARNING: amalgamation/../src/c_api/c_api.cc:785: `ntree_limit` is deprecated, use `iteration_range` instead.
## [23:12:25] WARNING: amalgamation/../src/c_api/c_api.cc:785: `ntree_limit` is deprecated, use `iteration_range` instead.
## [23:12:25] WARNING: amalgamation/../src/c_api/c_api.cc:785: `ntree_limit` is deprecated, use `iteration_range` instead.
## [23:12:25] WARNING: amalgamation/../src/c_api/c_api.cc:785: `ntree_limit` is deprecated, use `iteration_range` instead.
## [23:12:25] WARNING: amalgamation/../src/c_api/c_api.cc:785: `ntree_limit` is deprecated, use `iteration_range` instead.
## [23:12:25] WARNING: amalgamation/../src/c_api/c_api.cc:785: `ntree_limit` is deprecated, use `iteration_range` instead.
## [23:12:25] WARNING: amalgamation/../src/c_api/c_api.cc:785: `ntree_limit` is deprecated, use `iteration_range` instead.
## [23:12:25] WARNING: amalgamation/../src/c_api/c_api.cc:785: `ntree_limit` is deprecated, use `iteration_range` instead.
## [23:12:25] WARNING: amalgamation/../src/c_api/c_api.cc:785: `ntree_limit` is deprecated, use `iteration_range` instead.
## [23:12:25] WARNING: amalgamation/../src/c_api/c_api.cc:785: `ntree_limit` is deprecated, use `iteration_range` instead.
## [23:12:25] WARNING: amalgamation/../src/c_api/c_api.cc:785: `ntree_limit` is deprecated, use `iteration_range` instead.
## [23:12:25] WARNING: amalgamation/../src/c_api/c_api.cc:785: `ntree_limit` is deprecated, use `iteration_range` instead.
## [23:12:25] WARNING: amalgamation/../src/c_api/c_api.cc:785: `ntree_limit` is deprecated, use `iteration_range` instead.
## [23:12:25] WARNING: amalgamation/../src/c_api/c_api.cc:785: `ntree_limit` is deprecated, use `iteration_range` instead.
## [23:12:25] WARNING: amalgamation/../src/c_api/c_api.cc:785: `ntree_limit` is deprecated, use `iteration_range` instead.
## [23:12:25] WARNING: amalgamation/../src/c_api/c_api.cc:785: `ntree_limit` is deprecated, use `iteration_range` instead.
## [23:12:25] WARNING: amalgamation/../src/c_api/c_api.cc:785: `ntree_limit` is deprecated, use `iteration_range` instead.
## [23:12:25] WARNING: amalgamation/../src/c_api/c_api.cc:785: `ntree_limit` is deprecated, use `iteration_range` instead.
## [23:12:25] WARNING: amalgamation/../src/c_api/c_api.cc:785: `ntree_limit` is deprecated, use `iteration_range` instead.
## [23:12:25] WARNING: amalgamation/../src/c_api/c_api.cc:785: `ntree_limit` is deprecated, use `iteration_range` instead.
## [23:12:25] WARNING: amalgamation/../src/c_api/c_api.cc:785: `ntree_limit` is deprecated, use `iteration_range` instead.
## [23:12:25] WARNING: amalgamation/../src/c_api/c_api.cc:785: `ntree_limit` is deprecated, use `iteration_range` instead.
## [23:12:25] WARNING: amalgamation/../src/c_api/c_api.cc:785: `ntree_limit` is deprecated, use `iteration_range` instead.
## [23:12:25] WARNING: amalgamation/../src/c_api/c_api.cc:785: `ntree_limit` is deprecated, use `iteration_range` instead.
## [23:12:25] WARNING: amalgamation/../src/c_api/c_api.cc:785: `ntree_limit` is deprecated, use `iteration_range` instead.
## [23:12:25] WARNING: amalgamation/../src/c_api/c_api.cc:785: `ntree_limit` is deprecated, use `iteration_range` instead.
## [23:12:25] WARNING: amalgamation/../src/c_api/c_api.cc:785: `ntree_limit` is deprecated, use `iteration_range` instead.
## [23:12:25] WARNING: amalgamation/../src/c_api/c_api.cc:785: `ntree_limit` is deprecated, use `iteration_range` instead.
## [23:12:25] WARNING: amalgamation/../src/c_api/c_api.cc:785: `ntree_limit` is deprecated, use `iteration_range` instead.
## [23:12:25] WARNING: amalgamation/../src/c_api/c_api.cc:785: `ntree_limit` is deprecated, use `iteration_range` instead.
## [23:12:25] WARNING: amalgamation/../src/c_api/c_api.cc:785: `ntree_limit` is deprecated, use `iteration_range` instead.
## [23:12:25] WARNING: amalgamation/../src/c_api/c_api.cc:785: `ntree_limit` is deprecated, use `iteration_range` instead.
## [23:12:25] WARNING: amalgamation/../src/c_api/c_api.cc:785: `ntree_limit` is deprecated, use `iteration_range` instead.
## [23:12:25] WARNING: amalgamation/../src/c_api/c_api.cc:785: `ntree_limit` is deprecated, use `iteration_range` instead.
## [23:12:25] WARNING: amalgamation/../src/c_api/c_api.cc:785: `ntree_limit` is deprecated, use `iteration_range` instead.
## [23:12:26] WARNING: amalgamation/../src/c_api/c_api.cc:785: `ntree_limit` is deprecated, use `iteration_range` instead.
## [23:12:26] WARNING: amalgamation/../src/c_api/c_api.cc:785: `ntree_limit` is deprecated, use `iteration_range` instead.
## [23:12:26] WARNING: amalgamation/../src/c_api/c_api.cc:785: `ntree_limit` is deprecated, use `iteration_range` instead.
## [23:12:26] WARNING: amalgamation/../src/c_api/c_api.cc:785: `ntree_limit` is deprecated, use `iteration_range` instead.
## [23:12:26] WARNING: amalgamation/../src/c_api/c_api.cc:785: `ntree_limit` is deprecated, use `iteration_range` instead.
## [23:12:26] WARNING: amalgamation/../src/c_api/c_api.cc:785: `ntree_limit` is deprecated, use `iteration_range` instead.
## [23:12:26] WARNING: amalgamation/../src/c_api/c_api.cc:785: `ntree_limit` is deprecated, use `iteration_range` instead.
## [23:12:26] WARNING: amalgamation/../src/c_api/c_api.cc:785: `ntree_limit` is deprecated, use `iteration_range` instead.
## [23:12:26] WARNING: amalgamation/../src/c_api/c_api.cc:785: `ntree_limit` is deprecated, use `iteration_range` instead.
## [23:12:26] WARNING: amalgamation/../src/c_api/c_api.cc:785: `ntree_limit` is deprecated, use `iteration_range` instead.
## [23:12:26] WARNING: amalgamation/../src/c_api/c_api.cc:785: `ntree_limit` is deprecated, use `iteration_range` instead.
## [23:12:26] WARNING: amalgamation/../src/c_api/c_api.cc:785: `ntree_limit` is deprecated, use `iteration_range` instead.
## [23:12:26] WARNING: amalgamation/../src/c_api/c_api.cc:785: `ntree_limit` is deprecated, use `iteration_range` instead.
## [23:12:26] WARNING: amalgamation/../src/c_api/c_api.cc:785: `ntree_limit` is deprecated, use `iteration_range` instead.
## [23:12:26] WARNING: amalgamation/../src/c_api/c_api.cc:785: `ntree_limit` is deprecated, use `iteration_range` instead.
## [23:12:26] WARNING: amalgamation/../src/c_api/c_api.cc:785: `ntree_limit` is deprecated, use `iteration_range` instead.
## [23:12:26] WARNING: amalgamation/../src/c_api/c_api.cc:785: `ntree_limit` is deprecated, use `iteration_range` instead.
## [23:12:26] WARNING: amalgamation/../src/c_api/c_api.cc:785: `ntree_limit` is deprecated, use `iteration_range` instead.
## [23:12:26] WARNING: amalgamation/../src/c_api/c_api.cc:785: `ntree_limit` is deprecated, use `iteration_range` instead.
## [23:12:26] WARNING: amalgamation/../src/c_api/c_api.cc:785: `ntree_limit` is deprecated, use `iteration_range` instead.
## [23:12:26] WARNING: amalgamation/../src/c_api/c_api.cc:785: `ntree_limit` is deprecated, use `iteration_range` instead.
## [23:12:26] WARNING: amalgamation/../src/c_api/c_api.cc:785: `ntree_limit` is deprecated, use `iteration_range` instead.
## [23:12:26] WARNING: amalgamation/../src/c_api/c_api.cc:785: `ntree_limit` is deprecated, use `iteration_range` instead.
## [23:12:26] WARNING: amalgamation/../src/c_api/c_api.cc:785: `ntree_limit` is deprecated, use `iteration_range` instead.
## [23:12:26] WARNING: amalgamation/../src/c_api/c_api.cc:785: `ntree_limit` is deprecated, use `iteration_range` instead.
## [23:12:26] WARNING: amalgamation/../src/c_api/c_api.cc:785: `ntree_limit` is deprecated, use `iteration_range` instead.
## [23:12:26] WARNING: amalgamation/../src/c_api/c_api.cc:785: `ntree_limit` is deprecated, use `iteration_range` instead.
## [23:12:26] WARNING: amalgamation/../src/c_api/c_api.cc:785: `ntree_limit` is deprecated, use `iteration_range` instead.
## [23:12:26] WARNING: amalgamation/../src/c_api/c_api.cc:785: `ntree_limit` is deprecated, use `iteration_range` instead.
## [23:12:26] WARNING: amalgamation/../src/c_api/c_api.cc:785: `ntree_limit` is deprecated, use `iteration_range` instead.
## [23:12:26] WARNING: amalgamation/../src/c_api/c_api.cc:785: `ntree_limit` is deprecated, use `iteration_range` instead.
## [23:12:26] WARNING: amalgamation/../src/c_api/c_api.cc:785: `ntree_limit` is deprecated, use `iteration_range` instead.
## [23:12:26] WARNING: amalgamation/../src/c_api/c_api.cc:785: `ntree_limit` is deprecated, use `iteration_range` instead.
## [23:12:26] WARNING: amalgamation/../src/c_api/c_api.cc:785: `ntree_limit` is deprecated, use `iteration_range` instead.
## [23:12:26] WARNING: amalgamation/../src/c_api/c_api.cc:785: `ntree_limit` is deprecated, use `iteration_range` instead.
## [23:12:26] WARNING: amalgamation/../src/c_api/c_api.cc:785: `ntree_limit` is deprecated, use `iteration_range` instead.
## [23:12:26] WARNING: amalgamation/../src/c_api/c_api.cc:785: `ntree_limit` is deprecated, use `iteration_range` instead.
## [23:12:26] WARNING: amalgamation/../src/c_api/c_api.cc:785: `ntree_limit` is deprecated, use `iteration_range` instead.
## [23:12:26] WARNING: amalgamation/../src/c_api/c_api.cc:785: `ntree_limit` is deprecated, use `iteration_range` instead.
## [23:12:26] WARNING: amalgamation/../src/c_api/c_api.cc:785: `ntree_limit` is deprecated, use `iteration_range` instead.
## [23:12:26] WARNING: amalgamation/../src/c_api/c_api.cc:785: `ntree_limit` is deprecated, use `iteration_range` instead.
## [23:12:26] WARNING: amalgamation/../src/c_api/c_api.cc:785: `ntree_limit` is deprecated, use `iteration_range` instead.
## [23:12:26] WARNING: amalgamation/../src/c_api/c_api.cc:785: `ntree_limit` is deprecated, use `iteration_range` instead.
## [23:12:26] WARNING: amalgamation/../src/c_api/c_api.cc:785: `ntree_limit` is deprecated, use `iteration_range` instead.
## [23:12:26] WARNING: amalgamation/../src/c_api/c_api.cc:785: `ntree_limit` is deprecated, use `iteration_range` instead.
## [23:12:26] WARNING: amalgamation/../src/c_api/c_api.cc:785: `ntree_limit` is deprecated, use `iteration_range` instead.
## [23:12:26] WARNING: amalgamation/../src/c_api/c_api.cc:785: `ntree_limit` is deprecated, use `iteration_range` instead.
## [23:12:26] WARNING: amalgamation/../src/c_api/c_api.cc:785: `ntree_limit` is deprecated, use `iteration_range` instead.
## [23:12:26] WARNING: amalgamation/../src/c_api/c_api.cc:785: `ntree_limit` is deprecated, use `iteration_range` instead.
## [23:12:26] WARNING: amalgamation/../src/c_api/c_api.cc:785: `ntree_limit` is deprecated, use `iteration_range` instead.
## [23:12:26] WARNING: amalgamation/../src/c_api/c_api.cc:785: `ntree_limit` is deprecated, use `iteration_range` instead.
## [23:12:26] WARNING: amalgamation/../src/c_api/c_api.cc:785: `ntree_limit` is deprecated, use `iteration_range` instead.
## [23:12:26] WARNING: amalgamation/../src/c_api/c_api.cc:785: `ntree_limit` is deprecated, use `iteration_range` instead.
## [23:12:26] WARNING: amalgamation/../src/c_api/c_api.cc:785: `ntree_limit` is deprecated, use `iteration_range` instead.
## [23:12:26] WARNING: amalgamation/../src/c_api/c_api.cc:785: `ntree_limit` is deprecated, use `iteration_range` instead.
## [23:12:26] WARNING: amalgamation/../src/c_api/c_api.cc:785: `ntree_limit` is deprecated, use `iteration_range` instead.
## [23:12:26] WARNING: amalgamation/../src/c_api/c_api.cc:785: `ntree_limit` is deprecated, use `iteration_range` instead.
## [23:12:26] WARNING: amalgamation/../src/c_api/c_api.cc:785: `ntree_limit` is deprecated, use `iteration_range` instead.
## [23:12:26] WARNING: amalgamation/../src/c_api/c_api.cc:785: `ntree_limit` is deprecated, use `iteration_range` instead.
## [23:12:26] WARNING: amalgamation/../src/c_api/c_api.cc:785: `ntree_limit` is deprecated, use `iteration_range` instead.
## [23:12:26] WARNING: amalgamation/../src/c_api/c_api.cc:785: `ntree_limit` is deprecated, use `iteration_range` instead.
## [23:12:26] WARNING: amalgamation/../src/c_api/c_api.cc:785: `ntree_limit` is deprecated, use `iteration_range` instead.
## [23:12:26] WARNING: amalgamation/../src/c_api/c_api.cc:785: `ntree_limit` is deprecated, use `iteration_range` instead.
## [23:12:26] WARNING: amalgamation/../src/c_api/c_api.cc:785: `ntree_limit` is deprecated, use `iteration_range` instead.
## [23:12:26] WARNING: amalgamation/../src/c_api/c_api.cc:785: `ntree_limit` is deprecated, use `iteration_range` instead.
## [23:12:26] WARNING: amalgamation/../src/c_api/c_api.cc:785: `ntree_limit` is deprecated, use `iteration_range` instead.
## [23:12:26] WARNING: amalgamation/../src/c_api/c_api.cc:785: `ntree_limit` is deprecated, use `iteration_range` instead.
## [23:12:26] WARNING: amalgamation/../src/c_api/c_api.cc:785: `ntree_limit` is deprecated, use `iteration_range` instead.
## [23:12:26] WARNING: amalgamation/../src/c_api/c_api.cc:785: `ntree_limit` is deprecated, use `iteration_range` instead.
## [23:12:26] WARNING: amalgamation/../src/c_api/c_api.cc:785: `ntree_limit` is deprecated, use `iteration_range` instead.
## [23:12:26] WARNING: amalgamation/../src/c_api/c_api.cc:785: `ntree_limit` is deprecated, use `iteration_range` instead.
## [23:12:26] WARNING: amalgamation/../src/c_api/c_api.cc:785: `ntree_limit` is deprecated, use `iteration_range` instead.
## [23:12:26] WARNING: amalgamation/../src/c_api/c_api.cc:785: `ntree_limit` is deprecated, use `iteration_range` instead.
## [23:12:26] WARNING: amalgamation/../src/c_api/c_api.cc:785: `ntree_limit` is deprecated, use `iteration_range` instead.
## [23:12:26] WARNING: amalgamation/../src/c_api/c_api.cc:785: `ntree_limit` is deprecated, use `iteration_range` instead.
## [23:12:26] WARNING: amalgamation/../src/c_api/c_api.cc:785: `ntree_limit` is deprecated, use `iteration_range` instead.
## [23:12:26] WARNING: amalgamation/../src/c_api/c_api.cc:785: `ntree_limit` is deprecated, use `iteration_range` instead.
## [23:12:26] WARNING: amalgamation/../src/c_api/c_api.cc:785: `ntree_limit` is deprecated, use `iteration_range` instead.
## [23:12:26] WARNING: amalgamation/../src/c_api/c_api.cc:785: `ntree_limit` is deprecated, use `iteration_range` instead.
## [23:12:26] WARNING: amalgamation/../src/c_api/c_api.cc:785: `ntree_limit` is deprecated, use `iteration_range` instead.
## [23:12:26] WARNING: amalgamation/../src/c_api/c_api.cc:785: `ntree_limit` is deprecated, use `iteration_range` instead.
## [23:12:26] WARNING: amalgamation/../src/c_api/c_api.cc:785: `ntree_limit` is deprecated, use `iteration_range` instead.
## [23:12:26] WARNING: amalgamation/../src/c_api/c_api.cc:785: `ntree_limit` is deprecated, use `iteration_range` instead.
## [23:12:26] WARNING: amalgamation/../src/c_api/c_api.cc:785: `ntree_limit` is deprecated, use `iteration_range` instead.
## [23:12:26] WARNING: amalgamation/../src/c_api/c_api.cc:785: `ntree_limit` is deprecated, use `iteration_range` instead.
## [23:12:26] WARNING: amalgamation/../src/c_api/c_api.cc:785: `ntree_limit` is deprecated, use `iteration_range` instead.
## [23:12:26] WARNING: amalgamation/../src/c_api/c_api.cc:785: `ntree_limit` is deprecated, use `iteration_range` instead.
## [23:12:26] WARNING: amalgamation/../src/c_api/c_api.cc:785: `ntree_limit` is deprecated, use `iteration_range` instead.
## [23:12:26] WARNING: amalgamation/../src/c_api/c_api.cc:785: `ntree_limit` is deprecated, use `iteration_range` instead.
## [23:12:26] WARNING: amalgamation/../src/c_api/c_api.cc:785: `ntree_limit` is deprecated, use `iteration_range` instead.
## [23:12:26] WARNING: amalgamation/../src/c_api/c_api.cc:785: `ntree_limit` is deprecated, use `iteration_range` instead.
## [23:12:26] WARNING: amalgamation/../src/c_api/c_api.cc:785: `ntree_limit` is deprecated, use `iteration_range` instead.
## [23:12:26] WARNING: amalgamation/../src/c_api/c_api.cc:785: `ntree_limit` is deprecated, use `iteration_range` instead.
## [23:12:26] WARNING: amalgamation/../src/c_api/c_api.cc:785: `ntree_limit` is deprecated, use `iteration_range` instead.
## [23:12:26] WARNING: amalgamation/../src/c_api/c_api.cc:785: `ntree_limit` is deprecated, use `iteration_range` instead.
## [23:12:26] WARNING: amalgamation/../src/c_api/c_api.cc:785: `ntree_limit` is deprecated, use `iteration_range` instead.
## [23:12:26] WARNING: amalgamation/../src/c_api/c_api.cc:785: `ntree_limit` is deprecated, use `iteration_range` instead.
## [23:12:26] WARNING: amalgamation/../src/c_api/c_api.cc:785: `ntree_limit` is deprecated, use `iteration_range` instead.
## [23:12:26] WARNING: amalgamation/../src/c_api/c_api.cc:785: `ntree_limit` is deprecated, use `iteration_range` instead.
## [23:12:26] WARNING: amalgamation/../src/c_api/c_api.cc:785: `ntree_limit` is deprecated, use `iteration_range` instead.
## [23:12:26] WARNING: amalgamation/../src/c_api/c_api.cc:785: `ntree_limit` is deprecated, use `iteration_range` instead.
## [23:12:26] WARNING: amalgamation/../src/c_api/c_api.cc:785: `ntree_limit` is deprecated, use `iteration_range` instead.
## [23:12:26] WARNING: amalgamation/../src/c_api/c_api.cc:785: `ntree_limit` is deprecated, use `iteration_range` instead.
## [23:12:26] WARNING: amalgamation/../src/c_api/c_api.cc:785: `ntree_limit` is deprecated, use `iteration_range` instead.
## [23:12:26] WARNING: amalgamation/../src/c_api/c_api.cc:785: `ntree_limit` is deprecated, use `iteration_range` instead.
## [23:12:26] WARNING: amalgamation/../src/c_api/c_api.cc:785: `ntree_limit` is deprecated, use `iteration_range` instead.
## [23:12:26] WARNING: amalgamation/../src/c_api/c_api.cc:785: `ntree_limit` is deprecated, use `iteration_range` instead.
## [23:12:26] WARNING: amalgamation/../src/c_api/c_api.cc:785: `ntree_limit` is deprecated, use `iteration_range` instead.
## [23:12:26] WARNING: amalgamation/../src/c_api/c_api.cc:785: `ntree_limit` is deprecated, use `iteration_range` instead.
## [23:12:26] WARNING: amalgamation/../src/c_api/c_api.cc:785: `ntree_limit` is deprecated, use `iteration_range` instead.
## [23:12:26] WARNING: amalgamation/../src/c_api/c_api.cc:785: `ntree_limit` is deprecated, use `iteration_range` instead.
## [23:12:26] WARNING: amalgamation/../src/c_api/c_api.cc:785: `ntree_limit` is deprecated, use `iteration_range` instead.
## [23:12:26] WARNING: amalgamation/../src/c_api/c_api.cc:785: `ntree_limit` is deprecated, use `iteration_range` instead.
## [23:12:26] WARNING: amalgamation/../src/c_api/c_api.cc:785: `ntree_limit` is deprecated, use `iteration_range` instead.
## [23:12:26] WARNING: amalgamation/../src/c_api/c_api.cc:785: `ntree_limit` is deprecated, use `iteration_range` instead.
## [23:12:26] WARNING: amalgamation/../src/c_api/c_api.cc:785: `ntree_limit` is deprecated, use `iteration_range` instead.
## [23:12:26] WARNING: amalgamation/../src/c_api/c_api.cc:785: `ntree_limit` is deprecated, use `iteration_range` instead.
## [23:12:26] WARNING: amalgamation/../src/c_api/c_api.cc:785: `ntree_limit` is deprecated, use `iteration_range` instead.
## [23:12:26] WARNING: amalgamation/../src/c_api/c_api.cc:785: `ntree_limit` is deprecated, use `iteration_range` instead.
## [23:12:26] WARNING: amalgamation/../src/c_api/c_api.cc:785: `ntree_limit` is deprecated, use `iteration_range` instead.
## [23:12:26] WARNING: amalgamation/../src/c_api/c_api.cc:785: `ntree_limit` is deprecated, use `iteration_range` instead.
## [23:12:26] WARNING: amalgamation/../src/c_api/c_api.cc:785: `ntree_limit` is deprecated, use `iteration_range` instead.
## [23:12:26] WARNING: amalgamation/../src/c_api/c_api.cc:785: `ntree_limit` is deprecated, use `iteration_range` instead.
## [23:12:26] WARNING: amalgamation/../src/c_api/c_api.cc:785: `ntree_limit` is deprecated, use `iteration_range` instead.
## [23:12:26] WARNING: amalgamation/../src/c_api/c_api.cc:785: `ntree_limit` is deprecated, use `iteration_range` instead.
## [23:12:26] WARNING: amalgamation/../src/c_api/c_api.cc:785: `ntree_limit` is deprecated, use `iteration_range` instead.
## [23:12:26] WARNING: amalgamation/../src/c_api/c_api.cc:785: `ntree_limit` is deprecated, use `iteration_range` instead.
## [23:12:26] WARNING: amalgamation/../src/c_api/c_api.cc:785: `ntree_limit` is deprecated, use `iteration_range` instead.
## [23:12:26] WARNING: amalgamation/../src/c_api/c_api.cc:785: `ntree_limit` is deprecated, use `iteration_range` instead.
## [23:12:26] WARNING: amalgamation/../src/c_api/c_api.cc:785: `ntree_limit` is deprecated, use `iteration_range` instead.
## [23:12:26] WARNING: amalgamation/../src/c_api/c_api.cc:785: `ntree_limit` is deprecated, use `iteration_range` instead.
## [23:12:26] WARNING: amalgamation/../src/c_api/c_api.cc:785: `ntree_limit` is deprecated, use `iteration_range` instead.
## [23:12:26] WARNING: amalgamation/../src/c_api/c_api.cc:785: `ntree_limit` is deprecated, use `iteration_range` instead.
## [23:12:26] WARNING: amalgamation/../src/c_api/c_api.cc:785: `ntree_limit` is deprecated, use `iteration_range` instead.
## [23:12:26] WARNING: amalgamation/../src/c_api/c_api.cc:785: `ntree_limit` is deprecated, use `iteration_range` instead.
## [23:12:26] WARNING: amalgamation/../src/c_api/c_api.cc:785: `ntree_limit` is deprecated, use `iteration_range` instead.
## [23:12:27] WARNING: amalgamation/../src/c_api/c_api.cc:785: `ntree_limit` is deprecated, use `iteration_range` instead.
## [23:12:27] WARNING: amalgamation/../src/c_api/c_api.cc:785: `ntree_limit` is deprecated, use `iteration_range` instead.
## [23:12:27] WARNING: amalgamation/../src/c_api/c_api.cc:785: `ntree_limit` is deprecated, use `iteration_range` instead.
## [23:12:27] WARNING: amalgamation/../src/c_api/c_api.cc:785: `ntree_limit` is deprecated, use `iteration_range` instead.
## [23:12:27] WARNING: amalgamation/../src/c_api/c_api.cc:785: `ntree_limit` is deprecated, use `iteration_range` instead.
## [23:12:27] WARNING: amalgamation/../src/c_api/c_api.cc:785: `ntree_limit` is deprecated, use `iteration_range` instead.
## [23:12:27] WARNING: amalgamation/../src/c_api/c_api.cc:785: `ntree_limit` is deprecated, use `iteration_range` instead.
## [23:12:27] WARNING: amalgamation/../src/c_api/c_api.cc:785: `ntree_limit` is deprecated, use `iteration_range` instead.
## [23:12:27] WARNING: amalgamation/../src/c_api/c_api.cc:785: `ntree_limit` is deprecated, use `iteration_range` instead.
## [23:12:27] WARNING: amalgamation/../src/c_api/c_api.cc:785: `ntree_limit` is deprecated, use `iteration_range` instead.
## [23:12:27] WARNING: amalgamation/../src/c_api/c_api.cc:785: `ntree_limit` is deprecated, use `iteration_range` instead.
## [23:12:27] WARNING: amalgamation/../src/c_api/c_api.cc:785: `ntree_limit` is deprecated, use `iteration_range` instead.
## [23:12:27] WARNING: amalgamation/../src/c_api/c_api.cc:785: `ntree_limit` is deprecated, use `iteration_range` instead.
## [23:12:27] WARNING: amalgamation/../src/c_api/c_api.cc:785: `ntree_limit` is deprecated, use `iteration_range` instead.
## [23:12:27] WARNING: amalgamation/../src/c_api/c_api.cc:785: `ntree_limit` is deprecated, use `iteration_range` instead.
## [23:12:27] WARNING: amalgamation/../src/c_api/c_api.cc:785: `ntree_limit` is deprecated, use `iteration_range` instead.
## [23:12:27] WARNING: amalgamation/../src/c_api/c_api.cc:785: `ntree_limit` is deprecated, use `iteration_range` instead.
## [23:12:27] WARNING: amalgamation/../src/c_api/c_api.cc:785: `ntree_limit` is deprecated, use `iteration_range` instead.
## [23:12:27] WARNING: amalgamation/../src/c_api/c_api.cc:785: `ntree_limit` is deprecated, use `iteration_range` instead.
## [23:12:27] WARNING: amalgamation/../src/c_api/c_api.cc:785: `ntree_limit` is deprecated, use `iteration_range` instead.
## [23:12:27] WARNING: amalgamation/../src/c_api/c_api.cc:785: `ntree_limit` is deprecated, use `iteration_range` instead.
## [23:12:27] WARNING: amalgamation/../src/c_api/c_api.cc:785: `ntree_limit` is deprecated, use `iteration_range` instead.
## [23:12:27] WARNING: amalgamation/../src/c_api/c_api.cc:785: `ntree_limit` is deprecated, use `iteration_range` instead.
## [23:12:27] WARNING: amalgamation/../src/c_api/c_api.cc:785: `ntree_limit` is deprecated, use `iteration_range` instead.
## [23:12:27] WARNING: amalgamation/../src/c_api/c_api.cc:785: `ntree_limit` is deprecated, use `iteration_range` instead.
## [23:12:27] WARNING: amalgamation/../src/c_api/c_api.cc:785: `ntree_limit` is deprecated, use `iteration_range` instead.
## [23:12:27] WARNING: amalgamation/../src/c_api/c_api.cc:785: `ntree_limit` is deprecated, use `iteration_range` instead.
## [23:12:27] WARNING: amalgamation/../src/c_api/c_api.cc:785: `ntree_limit` is deprecated, use `iteration_range` instead.
## [23:12:27] WARNING: amalgamation/../src/c_api/c_api.cc:785: `ntree_limit` is deprecated, use `iteration_range` instead.
## [23:12:27] WARNING: amalgamation/../src/c_api/c_api.cc:785: `ntree_limit` is deprecated, use `iteration_range` instead.
## [23:12:27] WARNING: amalgamation/../src/c_api/c_api.cc:785: `ntree_limit` is deprecated, use `iteration_range` instead.
## [23:12:27] WARNING: amalgamation/../src/c_api/c_api.cc:785: `ntree_limit` is deprecated, use `iteration_range` instead.
## [23:12:27] WARNING: amalgamation/../src/c_api/c_api.cc:785: `ntree_limit` is deprecated, use `iteration_range` instead.
## [23:12:27] WARNING: amalgamation/../src/c_api/c_api.cc:785: `ntree_limit` is deprecated, use `iteration_range` instead.
## [23:12:27] WARNING: amalgamation/../src/c_api/c_api.cc:785: `ntree_limit` is deprecated, use `iteration_range` instead.
## [23:12:27] WARNING: amalgamation/../src/c_api/c_api.cc:785: `ntree_limit` is deprecated, use `iteration_range` instead.
## [23:12:27] WARNING: amalgamation/../src/c_api/c_api.cc:785: `ntree_limit` is deprecated, use `iteration_range` instead.
## [23:12:27] WARNING: amalgamation/../src/c_api/c_api.cc:785: `ntree_limit` is deprecated, use `iteration_range` instead.
## [23:12:27] WARNING: amalgamation/../src/c_api/c_api.cc:785: `ntree_limit` is deprecated, use `iteration_range` instead.
## [23:12:27] WARNING: amalgamation/../src/c_api/c_api.cc:785: `ntree_limit` is deprecated, use `iteration_range` instead.
## [23:12:27] WARNING: amalgamation/../src/c_api/c_api.cc:785: `ntree_limit` is deprecated, use `iteration_range` instead.
## [23:12:27] WARNING: amalgamation/../src/c_api/c_api.cc:785: `ntree_limit` is deprecated, use `iteration_range` instead.
## [23:12:27] WARNING: amalgamation/../src/c_api/c_api.cc:785: `ntree_limit` is deprecated, use `iteration_range` instead.
## [23:12:27] WARNING: amalgamation/../src/c_api/c_api.cc:785: `ntree_limit` is deprecated, use `iteration_range` instead.
## [23:12:27] WARNING: amalgamation/../src/c_api/c_api.cc:785: `ntree_limit` is deprecated, use `iteration_range` instead.
## [23:12:27] WARNING: amalgamation/../src/c_api/c_api.cc:785: `ntree_limit` is deprecated, use `iteration_range` instead.
## [23:12:27] WARNING: amalgamation/../src/c_api/c_api.cc:785: `ntree_limit` is deprecated, use `iteration_range` instead.
## [23:12:27] WARNING: amalgamation/../src/c_api/c_api.cc:785: `ntree_limit` is deprecated, use `iteration_range` instead.
## [23:12:27] WARNING: amalgamation/../src/c_api/c_api.cc:785: `ntree_limit` is deprecated, use `iteration_range` instead.
## [23:12:27] WARNING: amalgamation/../src/c_api/c_api.cc:785: `ntree_limit` is deprecated, use `iteration_range` instead.
## [23:12:27] WARNING: amalgamation/../src/c_api/c_api.cc:785: `ntree_limit` is deprecated, use `iteration_range` instead.
## [23:12:27] WARNING: amalgamation/../src/c_api/c_api.cc:785: `ntree_limit` is deprecated, use `iteration_range` instead.
## [23:12:27] WARNING: amalgamation/../src/c_api/c_api.cc:785: `ntree_limit` is deprecated, use `iteration_range` instead.
## [23:12:27] WARNING: amalgamation/../src/c_api/c_api.cc:785: `ntree_limit` is deprecated, use `iteration_range` instead.
## [23:12:27] WARNING: amalgamation/../src/c_api/c_api.cc:785: `ntree_limit` is deprecated, use `iteration_range` instead.
## [23:12:27] WARNING: amalgamation/../src/c_api/c_api.cc:785: `ntree_limit` is deprecated, use `iteration_range` instead.
## [23:12:27] WARNING: amalgamation/../src/c_api/c_api.cc:785: `ntree_limit` is deprecated, use `iteration_range` instead.
## [23:12:27] WARNING: amalgamation/../src/c_api/c_api.cc:785: `ntree_limit` is deprecated, use `iteration_range` instead.
## [23:12:27] WARNING: amalgamation/../src/c_api/c_api.cc:785: `ntree_limit` is deprecated, use `iteration_range` instead.
## [23:12:27] WARNING: amalgamation/../src/c_api/c_api.cc:785: `ntree_limit` is deprecated, use `iteration_range` instead.
## [23:12:27] WARNING: amalgamation/../src/c_api/c_api.cc:785: `ntree_limit` is deprecated, use `iteration_range` instead.
## [23:12:27] WARNING: amalgamation/../src/c_api/c_api.cc:785: `ntree_limit` is deprecated, use `iteration_range` instead.
## [23:12:27] WARNING: amalgamation/../src/c_api/c_api.cc:785: `ntree_limit` is deprecated, use `iteration_range` instead.
## [23:12:27] WARNING: amalgamation/../src/c_api/c_api.cc:785: `ntree_limit` is deprecated, use `iteration_range` instead.
## [23:12:27] WARNING: amalgamation/../src/c_api/c_api.cc:785: `ntree_limit` is deprecated, use `iteration_range` instead.
## [23:12:27] WARNING: amalgamation/../src/c_api/c_api.cc:785: `ntree_limit` is deprecated, use `iteration_range` instead.
## [23:12:27] WARNING: amalgamation/../src/c_api/c_api.cc:785: `ntree_limit` is deprecated, use `iteration_range` instead.
## [23:12:27] WARNING: amalgamation/../src/c_api/c_api.cc:785: `ntree_limit` is deprecated, use `iteration_range` instead.
## [23:12:27] WARNING: amalgamation/../src/c_api/c_api.cc:785: `ntree_limit` is deprecated, use `iteration_range` instead.
## [23:12:27] WARNING: amalgamation/../src/c_api/c_api.cc:785: `ntree_limit` is deprecated, use `iteration_range` instead.
## [23:12:27] WARNING: amalgamation/../src/c_api/c_api.cc:785: `ntree_limit` is deprecated, use `iteration_range` instead.
## [23:12:27] WARNING: amalgamation/../src/c_api/c_api.cc:785: `ntree_limit` is deprecated, use `iteration_range` instead.
## [23:12:27] WARNING: amalgamation/../src/c_api/c_api.cc:785: `ntree_limit` is deprecated, use `iteration_range` instead.
## [23:12:27] WARNING: amalgamation/../src/c_api/c_api.cc:785: `ntree_limit` is deprecated, use `iteration_range` instead.
## [23:12:27] WARNING: amalgamation/../src/c_api/c_api.cc:785: `ntree_limit` is deprecated, use `iteration_range` instead.
## [23:12:27] WARNING: amalgamation/../src/c_api/c_api.cc:785: `ntree_limit` is deprecated, use `iteration_range` instead.
## [23:12:27] WARNING: amalgamation/../src/c_api/c_api.cc:785: `ntree_limit` is deprecated, use `iteration_range` instead.
## [23:12:27] WARNING: amalgamation/../src/c_api/c_api.cc:785: `ntree_limit` is deprecated, use `iteration_range` instead.
## [23:12:27] WARNING: amalgamation/../src/c_api/c_api.cc:785: `ntree_limit` is deprecated, use `iteration_range` instead.
## [23:12:27] WARNING: amalgamation/../src/c_api/c_api.cc:785: `ntree_limit` is deprecated, use `iteration_range` instead.
## [23:12:27] WARNING: amalgamation/../src/c_api/c_api.cc:785: `ntree_limit` is deprecated, use `iteration_range` instead.
## [23:12:27] WARNING: amalgamation/../src/c_api/c_api.cc:785: `ntree_limit` is deprecated, use `iteration_range` instead.
## [23:12:27] WARNING: amalgamation/../src/c_api/c_api.cc:785: `ntree_limit` is deprecated, use `iteration_range` instead.
## [23:12:27] WARNING: amalgamation/../src/c_api/c_api.cc:785: `ntree_limit` is deprecated, use `iteration_range` instead.
## [23:12:27] WARNING: amalgamation/../src/c_api/c_api.cc:785: `ntree_limit` is deprecated, use `iteration_range` instead.
## [23:12:27] WARNING: amalgamation/../src/c_api/c_api.cc:785: `ntree_limit` is deprecated, use `iteration_range` instead.
## [23:12:27] WARNING: amalgamation/../src/c_api/c_api.cc:785: `ntree_limit` is deprecated, use `iteration_range` instead.
## [23:12:27] WARNING: amalgamation/../src/c_api/c_api.cc:785: `ntree_limit` is deprecated, use `iteration_range` instead.
## [23:12:27] WARNING: amalgamation/../src/c_api/c_api.cc:785: `ntree_limit` is deprecated, use `iteration_range` instead.
## [23:12:27] WARNING: amalgamation/../src/c_api/c_api.cc:785: `ntree_limit` is deprecated, use `iteration_range` instead.
## [23:12:27] WARNING: amalgamation/../src/c_api/c_api.cc:785: `ntree_limit` is deprecated, use `iteration_range` instead.
## [23:12:27] WARNING: amalgamation/../src/c_api/c_api.cc:785: `ntree_limit` is deprecated, use `iteration_range` instead.
## [23:12:27] WARNING: amalgamation/../src/c_api/c_api.cc:785: `ntree_limit` is deprecated, use `iteration_range` instead.
## [23:12:27] WARNING: amalgamation/../src/c_api/c_api.cc:785: `ntree_limit` is deprecated, use `iteration_range` instead.
## [23:12:27] WARNING: amalgamation/../src/c_api/c_api.cc:785: `ntree_limit` is deprecated, use `iteration_range` instead.
## [23:12:27] WARNING: amalgamation/../src/c_api/c_api.cc:785: `ntree_limit` is deprecated, use `iteration_range` instead.
## [23:12:27] WARNING: amalgamation/../src/c_api/c_api.cc:785: `ntree_limit` is deprecated, use `iteration_range` instead.
## [23:12:27] WARNING: amalgamation/../src/c_api/c_api.cc:785: `ntree_limit` is deprecated, use `iteration_range` instead.
## [23:12:27] WARNING: amalgamation/../src/c_api/c_api.cc:785: `ntree_limit` is deprecated, use `iteration_range` instead.
## [23:12:27] WARNING: amalgamation/../src/c_api/c_api.cc:785: `ntree_limit` is deprecated, use `iteration_range` instead.
## [23:12:27] WARNING: amalgamation/../src/c_api/c_api.cc:785: `ntree_limit` is deprecated, use `iteration_range` instead.
## [23:12:27] WARNING: amalgamation/../src/c_api/c_api.cc:785: `ntree_limit` is deprecated, use `iteration_range` instead.
## [23:12:27] WARNING: amalgamation/../src/c_api/c_api.cc:785: `ntree_limit` is deprecated, use `iteration_range` instead.
## [23:12:27] WARNING: amalgamation/../src/c_api/c_api.cc:785: `ntree_limit` is deprecated, use `iteration_range` instead.
## [23:12:27] WARNING: amalgamation/../src/c_api/c_api.cc:785: `ntree_limit` is deprecated, use `iteration_range` instead.
## [23:12:27] WARNING: amalgamation/../src/c_api/c_api.cc:785: `ntree_limit` is deprecated, use `iteration_range` instead.
## [23:12:27] WARNING: amalgamation/../src/c_api/c_api.cc:785: `ntree_limit` is deprecated, use `iteration_range` instead.
## [23:12:27] WARNING: amalgamation/../src/c_api/c_api.cc:785: `ntree_limit` is deprecated, use `iteration_range` instead.
## [23:12:27] WARNING: amalgamation/../src/c_api/c_api.cc:785: `ntree_limit` is deprecated, use `iteration_range` instead.
## [23:12:27] WARNING: amalgamation/../src/c_api/c_api.cc:785: `ntree_limit` is deprecated, use `iteration_range` instead.
## [23:12:27] WARNING: amalgamation/../src/c_api/c_api.cc:785: `ntree_limit` is deprecated, use `iteration_range` instead.
## [23:12:27] WARNING: amalgamation/../src/c_api/c_api.cc:785: `ntree_limit` is deprecated, use `iteration_range` instead.
## [23:12:27] WARNING: amalgamation/../src/c_api/c_api.cc:785: `ntree_limit` is deprecated, use `iteration_range` instead.
## [23:12:27] WARNING: amalgamation/../src/c_api/c_api.cc:785: `ntree_limit` is deprecated, use `iteration_range` instead.
## [23:12:27] WARNING: amalgamation/../src/c_api/c_api.cc:785: `ntree_limit` is deprecated, use `iteration_range` instead.
## [23:12:27] WARNING: amalgamation/../src/c_api/c_api.cc:785: `ntree_limit` is deprecated, use `iteration_range` instead.
## [23:12:27] WARNING: amalgamation/../src/c_api/c_api.cc:785: `ntree_limit` is deprecated, use `iteration_range` instead.
## [23:12:27] WARNING: amalgamation/../src/c_api/c_api.cc:785: `ntree_limit` is deprecated, use `iteration_range` instead.
## [23:12:27] WARNING: amalgamation/../src/c_api/c_api.cc:785: `ntree_limit` is deprecated, use `iteration_range` instead.
## [23:12:27] WARNING: amalgamation/../src/c_api/c_api.cc:785: `ntree_limit` is deprecated, use `iteration_range` instead.
## [23:12:27] WARNING: amalgamation/../src/c_api/c_api.cc:785: `ntree_limit` is deprecated, use `iteration_range` instead.
## [23:12:27] WARNING: amalgamation/../src/c_api/c_api.cc:785: `ntree_limit` is deprecated, use `iteration_range` instead.
## [23:12:27] WARNING: amalgamation/../src/c_api/c_api.cc:785: `ntree_limit` is deprecated, use `iteration_range` instead.
## [23:12:27] WARNING: amalgamation/../src/c_api/c_api.cc:785: `ntree_limit` is deprecated, use `iteration_range` instead.
## [23:12:27] WARNING: amalgamation/../src/c_api/c_api.cc:785: `ntree_limit` is deprecated, use `iteration_range` instead.
## [23:12:27] WARNING: amalgamation/../src/c_api/c_api.cc:785: `ntree_limit` is deprecated, use `iteration_range` instead.
## [23:12:27] WARNING: amalgamation/../src/c_api/c_api.cc:785: `ntree_limit` is deprecated, use `iteration_range` instead.
## [23:12:27] WARNING: amalgamation/../src/c_api/c_api.cc:785: `ntree_limit` is deprecated, use `iteration_range` instead.
## [23:12:27] WARNING: amalgamation/../src/c_api/c_api.cc:785: `ntree_limit` is deprecated, use `iteration_range` instead.
## [23:12:27] WARNING: amalgamation/../src/c_api/c_api.cc:785: `ntree_limit` is deprecated, use `iteration_range` instead.
## [23:12:27] WARNING: amalgamation/../src/c_api/c_api.cc:785: `ntree_limit` is deprecated, use `iteration_range` instead.
## [23:12:27] WARNING: amalgamation/../src/c_api/c_api.cc:785: `ntree_limit` is deprecated, use `iteration_range` instead.
## [23:12:28] WARNING: amalgamation/../src/c_api/c_api.cc:785: `ntree_limit` is deprecated, use `iteration_range` instead.
## [23:12:28] WARNING: amalgamation/../src/c_api/c_api.cc:785: `ntree_limit` is deprecated, use `iteration_range` instead.
## [23:12:28] WARNING: amalgamation/../src/c_api/c_api.cc:785: `ntree_limit` is deprecated, use `iteration_range` instead.
## [23:12:28] WARNING: amalgamation/../src/c_api/c_api.cc:785: `ntree_limit` is deprecated, use `iteration_range` instead.
## [23:12:28] WARNING: amalgamation/../src/c_api/c_api.cc:785: `ntree_limit` is deprecated, use `iteration_range` instead.
## [23:12:28] WARNING: amalgamation/../src/c_api/c_api.cc:785: `ntree_limit` is deprecated, use `iteration_range` instead.
## [23:12:28] WARNING: amalgamation/../src/c_api/c_api.cc:785: `ntree_limit` is deprecated, use `iteration_range` instead.
## [23:12:28] WARNING: amalgamation/../src/c_api/c_api.cc:785: `ntree_limit` is deprecated, use `iteration_range` instead.
## [23:12:28] WARNING: amalgamation/../src/c_api/c_api.cc:785: `ntree_limit` is deprecated, use `iteration_range` instead.
## [23:12:28] WARNING: amalgamation/../src/c_api/c_api.cc:785: `ntree_limit` is deprecated, use `iteration_range` instead.
## [23:12:28] WARNING: amalgamation/../src/c_api/c_api.cc:785: `ntree_limit` is deprecated, use `iteration_range` instead.
## [23:12:28] WARNING: amalgamation/../src/c_api/c_api.cc:785: `ntree_limit` is deprecated, use `iteration_range` instead.
## [23:12:28] WARNING: amalgamation/../src/c_api/c_api.cc:785: `ntree_limit` is deprecated, use `iteration_range` instead.
## [23:12:28] WARNING: amalgamation/../src/c_api/c_api.cc:785: `ntree_limit` is deprecated, use `iteration_range` instead.
## [23:12:28] WARNING: amalgamation/../src/c_api/c_api.cc:785: `ntree_limit` is deprecated, use `iteration_range` instead.
## [23:12:28] WARNING: amalgamation/../src/c_api/c_api.cc:785: `ntree_limit` is deprecated, use `iteration_range` instead.
## [23:12:28] WARNING: amalgamation/../src/c_api/c_api.cc:785: `ntree_limit` is deprecated, use `iteration_range` instead.
## [23:12:28] WARNING: amalgamation/../src/c_api/c_api.cc:785: `ntree_limit` is deprecated, use `iteration_range` instead.
## [23:12:28] WARNING: amalgamation/../src/c_api/c_api.cc:785: `ntree_limit` is deprecated, use `iteration_range` instead.
## [23:12:28] WARNING: amalgamation/../src/c_api/c_api.cc:785: `ntree_limit` is deprecated, use `iteration_range` instead.
## [23:12:28] WARNING: amalgamation/../src/c_api/c_api.cc:785: `ntree_limit` is deprecated, use `iteration_range` instead.
## [23:12:28] WARNING: amalgamation/../src/c_api/c_api.cc:785: `ntree_limit` is deprecated, use `iteration_range` instead.
## [23:12:28] WARNING: amalgamation/../src/c_api/c_api.cc:785: `ntree_limit` is deprecated, use `iteration_range` instead.
## [23:12:28] WARNING: amalgamation/../src/c_api/c_api.cc:785: `ntree_limit` is deprecated, use `iteration_range` instead.
## [23:12:28] WARNING: amalgamation/../src/c_api/c_api.cc:785: `ntree_limit` is deprecated, use `iteration_range` instead.
## [23:12:28] WARNING: amalgamation/../src/c_api/c_api.cc:785: `ntree_limit` is deprecated, use `iteration_range` instead.
## [23:12:28] WARNING: amalgamation/../src/c_api/c_api.cc:785: `ntree_limit` is deprecated, use `iteration_range` instead.
## [23:12:28] WARNING: amalgamation/../src/c_api/c_api.cc:785: `ntree_limit` is deprecated, use `iteration_range` instead.
## [23:12:28] WARNING: amalgamation/../src/c_api/c_api.cc:785: `ntree_limit` is deprecated, use `iteration_range` instead.
## [23:12:28] WARNING: amalgamation/../src/c_api/c_api.cc:785: `ntree_limit` is deprecated, use `iteration_range` instead.
## [23:12:28] WARNING: amalgamation/../src/c_api/c_api.cc:785: `ntree_limit` is deprecated, use `iteration_range` instead.
## [23:12:28] WARNING: amalgamation/../src/c_api/c_api.cc:785: `ntree_limit` is deprecated, use `iteration_range` instead.
## [23:12:28] WARNING: amalgamation/../src/c_api/c_api.cc:785: `ntree_limit` is deprecated, use `iteration_range` instead.
## [23:12:28] WARNING: amalgamation/../src/c_api/c_api.cc:785: `ntree_limit` is deprecated, use `iteration_range` instead.
## [23:12:28] WARNING: amalgamation/../src/c_api/c_api.cc:785: `ntree_limit` is deprecated, use `iteration_range` instead.
## [23:12:28] WARNING: amalgamation/../src/c_api/c_api.cc:785: `ntree_limit` is deprecated, use `iteration_range` instead.
## [23:12:28] WARNING: amalgamation/../src/c_api/c_api.cc:785: `ntree_limit` is deprecated, use `iteration_range` instead.
## [23:12:28] WARNING: amalgamation/../src/c_api/c_api.cc:785: `ntree_limit` is deprecated, use `iteration_range` instead.
## [23:12:28] WARNING: amalgamation/../src/c_api/c_api.cc:785: `ntree_limit` is deprecated, use `iteration_range` instead.
## [23:12:28] WARNING: amalgamation/../src/c_api/c_api.cc:785: `ntree_limit` is deprecated, use `iteration_range` instead.
## [23:12:28] WARNING: amalgamation/../src/c_api/c_api.cc:785: `ntree_limit` is deprecated, use `iteration_range` instead.
## [23:12:28] WARNING: amalgamation/../src/c_api/c_api.cc:785: `ntree_limit` is deprecated, use `iteration_range` instead.
## [23:12:28] WARNING: amalgamation/../src/c_api/c_api.cc:785: `ntree_limit` is deprecated, use `iteration_range` instead.
## [23:12:28] WARNING: amalgamation/../src/c_api/c_api.cc:785: `ntree_limit` is deprecated, use `iteration_range` instead.
## [23:12:28] WARNING: amalgamation/../src/c_api/c_api.cc:785: `ntree_limit` is deprecated, use `iteration_range` instead.
## [23:12:28] WARNING: amalgamation/../src/c_api/c_api.cc:785: `ntree_limit` is deprecated, use `iteration_range` instead.
## [23:12:28] WARNING: amalgamation/../src/c_api/c_api.cc:785: `ntree_limit` is deprecated, use `iteration_range` instead.
## [23:12:28] WARNING: amalgamation/../src/c_api/c_api.cc:785: `ntree_limit` is deprecated, use `iteration_range` instead.
## [23:12:28] WARNING: amalgamation/../src/c_api/c_api.cc:785: `ntree_limit` is deprecated, use `iteration_range` instead.
## [23:12:28] WARNING: amalgamation/../src/c_api/c_api.cc:785: `ntree_limit` is deprecated, use `iteration_range` instead.
## [23:12:28] WARNING: amalgamation/../src/c_api/c_api.cc:785: `ntree_limit` is deprecated, use `iteration_range` instead.
## [23:12:28] WARNING: amalgamation/../src/c_api/c_api.cc:785: `ntree_limit` is deprecated, use `iteration_range` instead.
## [23:12:28] WARNING: amalgamation/../src/c_api/c_api.cc:785: `ntree_limit` is deprecated, use `iteration_range` instead.
## [23:12:28] WARNING: amalgamation/../src/c_api/c_api.cc:785: `ntree_limit` is deprecated, use `iteration_range` instead.
## [23:12:28] WARNING: amalgamation/../src/c_api/c_api.cc:785: `ntree_limit` is deprecated, use `iteration_range` instead.
## [23:12:28] WARNING: amalgamation/../src/c_api/c_api.cc:785: `ntree_limit` is deprecated, use `iteration_range` instead.
## [23:12:28] WARNING: amalgamation/../src/c_api/c_api.cc:785: `ntree_limit` is deprecated, use `iteration_range` instead.
## [23:12:28] WARNING: amalgamation/../src/c_api/c_api.cc:785: `ntree_limit` is deprecated, use `iteration_range` instead.
## [23:12:28] WARNING: amalgamation/../src/c_api/c_api.cc:785: `ntree_limit` is deprecated, use `iteration_range` instead.
## [23:12:28] WARNING: amalgamation/../src/c_api/c_api.cc:785: `ntree_limit` is deprecated, use `iteration_range` instead.
## [23:12:28] WARNING: amalgamation/../src/c_api/c_api.cc:785: `ntree_limit` is deprecated, use `iteration_range` instead.
## [23:12:28] WARNING: amalgamation/../src/c_api/c_api.cc:785: `ntree_limit` is deprecated, use `iteration_range` instead.
## [23:12:28] WARNING: amalgamation/../src/c_api/c_api.cc:785: `ntree_limit` is deprecated, use `iteration_range` instead.
## [23:12:28] WARNING: amalgamation/../src/c_api/c_api.cc:785: `ntree_limit` is deprecated, use `iteration_range` instead.
## [23:12:28] WARNING: amalgamation/../src/c_api/c_api.cc:785: `ntree_limit` is deprecated, use `iteration_range` instead.
## [23:12:28] WARNING: amalgamation/../src/c_api/c_api.cc:785: `ntree_limit` is deprecated, use `iteration_range` instead.
## [23:12:28] WARNING: amalgamation/../src/c_api/c_api.cc:785: `ntree_limit` is deprecated, use `iteration_range` instead.
## [23:12:28] WARNING: amalgamation/../src/c_api/c_api.cc:785: `ntree_limit` is deprecated, use `iteration_range` instead.
## [23:12:28] WARNING: amalgamation/../src/c_api/c_api.cc:785: `ntree_limit` is deprecated, use `iteration_range` instead.
## [23:12:28] WARNING: amalgamation/../src/c_api/c_api.cc:785: `ntree_limit` is deprecated, use `iteration_range` instead.
## [23:12:28] WARNING: amalgamation/../src/c_api/c_api.cc:785: `ntree_limit` is deprecated, use `iteration_range` instead.
## [23:12:28] WARNING: amalgamation/../src/c_api/c_api.cc:785: `ntree_limit` is deprecated, use `iteration_range` instead.
## [23:12:28] WARNING: amalgamation/../src/c_api/c_api.cc:785: `ntree_limit` is deprecated, use `iteration_range` instead.
## [23:12:28] WARNING: amalgamation/../src/c_api/c_api.cc:785: `ntree_limit` is deprecated, use `iteration_range` instead.
## [23:12:28] WARNING: amalgamation/../src/c_api/c_api.cc:785: `ntree_limit` is deprecated, use `iteration_range` instead.
## [23:12:28] WARNING: amalgamation/../src/c_api/c_api.cc:785: `ntree_limit` is deprecated, use `iteration_range` instead.
## [23:12:28] WARNING: amalgamation/../src/c_api/c_api.cc:785: `ntree_limit` is deprecated, use `iteration_range` instead.
## [23:12:28] WARNING: amalgamation/../src/c_api/c_api.cc:785: `ntree_limit` is deprecated, use `iteration_range` instead.
## [23:12:28] WARNING: amalgamation/../src/c_api/c_api.cc:785: `ntree_limit` is deprecated, use `iteration_range` instead.
## [23:12:28] WARNING: amalgamation/../src/c_api/c_api.cc:785: `ntree_limit` is deprecated, use `iteration_range` instead.
## [23:12:28] WARNING: amalgamation/../src/c_api/c_api.cc:785: `ntree_limit` is deprecated, use `iteration_range` instead.
## [23:12:28] WARNING: amalgamation/../src/c_api/c_api.cc:785: `ntree_limit` is deprecated, use `iteration_range` instead.
## [23:12:28] WARNING: amalgamation/../src/c_api/c_api.cc:785: `ntree_limit` is deprecated, use `iteration_range` instead.
## [23:12:28] WARNING: amalgamation/../src/c_api/c_api.cc:785: `ntree_limit` is deprecated, use `iteration_range` instead.
## [23:12:28] WARNING: amalgamation/../src/c_api/c_api.cc:785: `ntree_limit` is deprecated, use `iteration_range` instead.
## [23:12:28] WARNING: amalgamation/../src/c_api/c_api.cc:785: `ntree_limit` is deprecated, use `iteration_range` instead.
## [23:12:28] WARNING: amalgamation/../src/c_api/c_api.cc:785: `ntree_limit` is deprecated, use `iteration_range` instead.
## [23:12:28] WARNING: amalgamation/../src/c_api/c_api.cc:785: `ntree_limit` is deprecated, use `iteration_range` instead.
## [23:12:28] WARNING: amalgamation/../src/c_api/c_api.cc:785: `ntree_limit` is deprecated, use `iteration_range` instead.
## [23:12:28] WARNING: amalgamation/../src/c_api/c_api.cc:785: `ntree_limit` is deprecated, use `iteration_range` instead.
## [23:12:28] WARNING: amalgamation/../src/c_api/c_api.cc:785: `ntree_limit` is deprecated, use `iteration_range` instead.
## [23:12:28] WARNING: amalgamation/../src/c_api/c_api.cc:785: `ntree_limit` is deprecated, use `iteration_range` instead.
## [23:12:28] WARNING: amalgamation/../src/c_api/c_api.cc:785: `ntree_limit` is deprecated, use `iteration_range` instead.
## [23:12:28] WARNING: amalgamation/../src/c_api/c_api.cc:785: `ntree_limit` is deprecated, use `iteration_range` instead.
## [23:12:28] WARNING: amalgamation/../src/c_api/c_api.cc:785: `ntree_limit` is deprecated, use `iteration_range` instead.
## [23:12:28] WARNING: amalgamation/../src/c_api/c_api.cc:785: `ntree_limit` is deprecated, use `iteration_range` instead.
## [23:12:28] WARNING: amalgamation/../src/c_api/c_api.cc:785: `ntree_limit` is deprecated, use `iteration_range` instead.
## [23:12:28] WARNING: amalgamation/../src/c_api/c_api.cc:785: `ntree_limit` is deprecated, use `iteration_range` instead.
## [23:12:28] WARNING: amalgamation/../src/c_api/c_api.cc:785: `ntree_limit` is deprecated, use `iteration_range` instead.
## [23:12:28] WARNING: amalgamation/../src/c_api/c_api.cc:785: `ntree_limit` is deprecated, use `iteration_range` instead.
## [23:12:28] WARNING: amalgamation/../src/c_api/c_api.cc:785: `ntree_limit` is deprecated, use `iteration_range` instead.
## [23:12:28] WARNING: amalgamation/../src/c_api/c_api.cc:785: `ntree_limit` is deprecated, use `iteration_range` instead.
## [23:12:28] WARNING: amalgamation/../src/c_api/c_api.cc:785: `ntree_limit` is deprecated, use `iteration_range` instead.
## [23:12:28] WARNING: amalgamation/../src/c_api/c_api.cc:785: `ntree_limit` is deprecated, use `iteration_range` instead.
## [23:12:28] WARNING: amalgamation/../src/c_api/c_api.cc:785: `ntree_limit` is deprecated, use `iteration_range` instead.
## [23:12:28] WARNING: amalgamation/../src/c_api/c_api.cc:785: `ntree_limit` is deprecated, use `iteration_range` instead.
## [23:12:28] WARNING: amalgamation/../src/c_api/c_api.cc:785: `ntree_limit` is deprecated, use `iteration_range` instead.
## [23:12:28] WARNING: amalgamation/../src/c_api/c_api.cc:785: `ntree_limit` is deprecated, use `iteration_range` instead.
## [23:12:28] WARNING: amalgamation/../src/c_api/c_api.cc:785: `ntree_limit` is deprecated, use `iteration_range` instead.
## [23:12:28] WARNING: amalgamation/../src/c_api/c_api.cc:785: `ntree_limit` is deprecated, use `iteration_range` instead.
## [23:12:28] WARNING: amalgamation/../src/c_api/c_api.cc:785: `ntree_limit` is deprecated, use `iteration_range` instead.
## [23:12:28] WARNING: amalgamation/../src/c_api/c_api.cc:785: `ntree_limit` is deprecated, use `iteration_range` instead.
## [23:12:29] WARNING: amalgamation/../src/c_api/c_api.cc:785: `ntree_limit` is deprecated, use `iteration_range` instead.
## [23:12:29] WARNING: amalgamation/../src/c_api/c_api.cc:785: `ntree_limit` is deprecated, use `iteration_range` instead.
## [23:12:29] WARNING: amalgamation/../src/c_api/c_api.cc:785: `ntree_limit` is deprecated, use `iteration_range` instead.
## [23:12:29] WARNING: amalgamation/../src/c_api/c_api.cc:785: `ntree_limit` is deprecated, use `iteration_range` instead.
## [23:12:29] WARNING: amalgamation/../src/c_api/c_api.cc:785: `ntree_limit` is deprecated, use `iteration_range` instead.
## [23:12:29] WARNING: amalgamation/../src/c_api/c_api.cc:785: `ntree_limit` is deprecated, use `iteration_range` instead.
## [23:12:29] WARNING: amalgamation/../src/c_api/c_api.cc:785: `ntree_limit` is deprecated, use `iteration_range` instead.
## [23:12:29] WARNING: amalgamation/../src/c_api/c_api.cc:785: `ntree_limit` is deprecated, use `iteration_range` instead.
## [23:12:29] WARNING: amalgamation/../src/c_api/c_api.cc:785: `ntree_limit` is deprecated, use `iteration_range` instead.
## [23:12:29] WARNING: amalgamation/../src/c_api/c_api.cc:785: `ntree_limit` is deprecated, use `iteration_range` instead.
## [23:12:29] WARNING: amalgamation/../src/c_api/c_api.cc:785: `ntree_limit` is deprecated, use `iteration_range` instead.
## [23:12:29] WARNING: amalgamation/../src/c_api/c_api.cc:785: `ntree_limit` is deprecated, use `iteration_range` instead.
## [23:12:29] WARNING: amalgamation/../src/c_api/c_api.cc:785: `ntree_limit` is deprecated, use `iteration_range` instead.
## [23:12:29] WARNING: amalgamation/../src/c_api/c_api.cc:785: `ntree_limit` is deprecated, use `iteration_range` instead.
## [23:12:29] WARNING: amalgamation/../src/c_api/c_api.cc:785: `ntree_limit` is deprecated, use `iteration_range` instead.
## [23:12:29] WARNING: amalgamation/../src/c_api/c_api.cc:785: `ntree_limit` is deprecated, use `iteration_range` instead.
## [23:12:29] WARNING: amalgamation/../src/c_api/c_api.cc:785: `ntree_limit` is deprecated, use `iteration_range` instead.
## [23:12:29] WARNING: amalgamation/../src/c_api/c_api.cc:785: `ntree_limit` is deprecated, use `iteration_range` instead.
## [23:12:29] WARNING: amalgamation/../src/c_api/c_api.cc:785: `ntree_limit` is deprecated, use `iteration_range` instead.
## [23:12:29] WARNING: amalgamation/../src/c_api/c_api.cc:785: `ntree_limit` is deprecated, use `iteration_range` instead.
## [23:12:29] WARNING: amalgamation/../src/c_api/c_api.cc:785: `ntree_limit` is deprecated, use `iteration_range` instead.
## [23:12:29] WARNING: amalgamation/../src/c_api/c_api.cc:785: `ntree_limit` is deprecated, use `iteration_range` instead.
## [23:12:29] WARNING: amalgamation/../src/c_api/c_api.cc:785: `ntree_limit` is deprecated, use `iteration_range` instead.
## [23:12:29] WARNING: amalgamation/../src/c_api/c_api.cc:785: `ntree_limit` is deprecated, use `iteration_range` instead.
## [23:12:29] WARNING: amalgamation/../src/c_api/c_api.cc:785: `ntree_limit` is deprecated, use `iteration_range` instead.
## [23:12:29] WARNING: amalgamation/../src/c_api/c_api.cc:785: `ntree_limit` is deprecated, use `iteration_range` instead.
## [23:12:29] WARNING: amalgamation/../src/c_api/c_api.cc:785: `ntree_limit` is deprecated, use `iteration_range` instead.
## [23:12:29] WARNING: amalgamation/../src/c_api/c_api.cc:785: `ntree_limit` is deprecated, use `iteration_range` instead.
## [23:12:29] WARNING: amalgamation/../src/c_api/c_api.cc:785: `ntree_limit` is deprecated, use `iteration_range` instead.
## [23:12:29] WARNING: amalgamation/../src/c_api/c_api.cc:785: `ntree_limit` is deprecated, use `iteration_range` instead.
## [23:12:29] WARNING: amalgamation/../src/c_api/c_api.cc:785: `ntree_limit` is deprecated, use `iteration_range` instead.
## [23:12:29] WARNING: amalgamation/../src/c_api/c_api.cc:785: `ntree_limit` is deprecated, use `iteration_range` instead.
## [23:12:29] WARNING: amalgamation/../src/c_api/c_api.cc:785: `ntree_limit` is deprecated, use `iteration_range` instead.
## [23:12:29] WARNING: amalgamation/../src/c_api/c_api.cc:785: `ntree_limit` is deprecated, use `iteration_range` instead.
## [23:12:29] WARNING: amalgamation/../src/c_api/c_api.cc:785: `ntree_limit` is deprecated, use `iteration_range` instead.
## [23:12:29] WARNING: amalgamation/../src/c_api/c_api.cc:785: `ntree_limit` is deprecated, use `iteration_range` instead.
## [23:12:29] WARNING: amalgamation/../src/c_api/c_api.cc:785: `ntree_limit` is deprecated, use `iteration_range` instead.
## [23:12:29] WARNING: amalgamation/../src/c_api/c_api.cc:785: `ntree_limit` is deprecated, use `iteration_range` instead.
## [23:12:29] WARNING: amalgamation/../src/c_api/c_api.cc:785: `ntree_limit` is deprecated, use `iteration_range` instead.
## [23:12:29] WARNING: amalgamation/../src/c_api/c_api.cc:785: `ntree_limit` is deprecated, use `iteration_range` instead.
\end{verbatim}

Print out the best tuning parameters.

\begin{Shaded}
\begin{Highlighting}[]
\NormalTok{fit\_xgb\_sonar}\SpecialCharTok{$}\NormalTok{bestTune}
\end{Highlighting}
\end{Shaded}

\begin{verbatim}
##    nrounds max_depth eta gamma colsample_bytree min_child_weight subsample
## 29     100         2 0.3     0              0.8                1       0.5
\end{verbatim}

Plot the cross-validation results.

\begin{Shaded}
\begin{Highlighting}[]
\FunctionTok{plot}\NormalTok{(fit\_xgb\_sonar)}
\end{Highlighting}
\end{Shaded}

\includegraphics{Drake_Zhou_HW_11_files/figure-latex/solution_04f_c-1.pdf}
\includegraphics{Drake_Zhou_HW_11_files/figure-latex/solution_04f_c-2.pdf}

\hypertarget{c-3}{%
\subsubsection{4c)}\label{c-3}}

\textbf{Use the \texttt{caret} variable importance method and plot the
top 20 variable importances for the random forest model and the XGBoost
model.}

\textbf{Are the top 4 ranked inputs the same as those identified by the
neural network?}

\hypertarget{solution-19}{%
\paragraph{SOLUTION}\label{solution-19}}

Plot the top 20 ranked inputs based on the random forest.

\begin{Shaded}
\begin{Highlighting}[]
\DocumentationTok{\#\#\#}
\FunctionTok{plot}\NormalTok{(}\FunctionTok{varImp}\NormalTok{(fit\_rf\_sonar), }\AttributeTok{top =} \DecValTok{20}\NormalTok{)}
\end{Highlighting}
\end{Shaded}

\includegraphics{Drake_Zhou_HW_11_files/figure-latex/solution_04c_c-1.pdf}

Plot the top 20 ranked inputs based on XGBoost.

\begin{Shaded}
\begin{Highlighting}[]
\DocumentationTok{\#\#\#}
\FunctionTok{plot}\NormalTok{(}\FunctionTok{varImp}\NormalTok{(fit\_xgb\_sonar), }\AttributeTok{top =} \DecValTok{20}\NormalTok{)}
\end{Highlighting}
\end{Shaded}

\includegraphics{Drake_Zhou_HW_11_files/figure-latex/solution_04c_b-1.pdf}

Are the top 4 ranked inputs the same as those identified by the neural
network?

No, according to figures above, they are not the same.

\hypertarget{d-3}{%
\subsubsection{4d)}\label{d-3}}

\textbf{Create an input test grid based on the top ranked inputs from
the random forest model.}

\textbf{Assign the result to the \texttt{viz\_input\_grid\_rf} object.}

\hypertarget{solution-20}{%
\paragraph{SOLUTION}\label{solution-20}}

\begin{Shaded}
\begin{Highlighting}[]
\NormalTok{viz\_input\_grid\_rf }\OtherTok{\textless{}{-}} \FunctionTok{make\_test\_input\_grid}\NormalTok{(}\FunctionTok{colnames}\NormalTok{(Sonar)[}\SpecialCharTok{{-}}\DecValTok{61}\NormalTok{],}
                                          \FunctionTok{c}\NormalTok{(}\StringTok{"V11"}\NormalTok{,}\StringTok{"V12"}\NormalTok{,}\StringTok{"V9"}\NormalTok{,}\StringTok{"V10"}\NormalTok{),}
\NormalTok{                                          Sonar)}
\FunctionTok{dim}\NormalTok{(viz\_input\_grid\_rf)}
\end{Highlighting}
\end{Shaded}

\begin{verbatim}
## [1] 15625    60
\end{verbatim}

\hypertarget{e-3}{%
\subsubsection{4e)}\label{e-3}}

\textbf{Predict the class probabilities with the random forest model for
the original neural network based input grid and the new random forest
ranked input grid.}

\textbf{Assign the predicted probabilities based on the neural network
input grid to \texttt{pred\_prob\_rf\_on\_nnet\_grid}. Assign the
predicted probabilities based on the random forest input grid to
\texttt{pred\_prob\_rf\_on\_rf\_grid}.}

\hypertarget{solution-21}{%
\paragraph{SOLUTION}\label{solution-21}}

Random forest predictions on the neural network based input grid.

\begin{Shaded}
\begin{Highlighting}[]
\DocumentationTok{\#\#\#}
\NormalTok{pred\_prob\_rf\_on\_nnet\_grid }\OtherTok{\textless{}{-}} \FunctionTok{predict}\NormalTok{(fit\_rf\_sonar, viz\_input\_grid, }\AttributeTok{type =} \StringTok{\textquotesingle{}prob\textquotesingle{}}\NormalTok{)}
\end{Highlighting}
\end{Shaded}

Random forest predictions on the random forest based input grid.

\begin{Shaded}
\begin{Highlighting}[]
\DocumentationTok{\#\#\#}
\NormalTok{pred\_prob\_rf\_on\_rf\_grid }\OtherTok{\textless{}{-}} \FunctionTok{predict}\NormalTok{(fit\_rf\_sonar, viz\_input\_grid\_rf, }\AttributeTok{type =} \StringTok{\textquotesingle{}prob\textquotesingle{}}\NormalTok{)}
\end{Highlighting}
\end{Shaded}

\hypertarget{f-2}{%
\subsubsection{4f)}\label{f-2}}

\textbf{Visualize the predicted probability surfaces for the \texttt{M}
class using the same visualization approach used in Problem 3f).}

\hypertarget{solution-22}{%
\paragraph{SOLUTION}\label{solution-22}}

Visualize the predicted probability surface based on the neural network
grid.

\begin{Shaded}
\begin{Highlighting}[]
\DocumentationTok{\#\#\#}
\NormalTok{viz\_input\_grid }\SpecialCharTok{\%\textgreater{}\%}
  \FunctionTok{bind\_cols}\NormalTok{(pred\_prob\_rf\_on\_nnet\_grid) }\SpecialCharTok{\%\textgreater{}\%}
  \FunctionTok{ggplot}\NormalTok{(}\AttributeTok{mapping =} \FunctionTok{aes}\NormalTok{(}\AttributeTok{x =}\NormalTok{ V50, }\AttributeTok{y =}\NormalTok{ V57))}\SpecialCharTok{+}
  \FunctionTok{geom\_raster}\NormalTok{(}\AttributeTok{mapping =} \FunctionTok{aes}\NormalTok{(}\AttributeTok{fill =}\NormalTok{ M))}\SpecialCharTok{+}
  \FunctionTok{facet\_grid}\NormalTok{(}\AttributeTok{rows =} \FunctionTok{vars}\NormalTok{(V31), }\AttributeTok{cols =} \FunctionTok{vars}\NormalTok{(V22))}\SpecialCharTok{+}
  \FunctionTok{scale\_fill\_gradient2}\NormalTok{(}\AttributeTok{limits =} \FunctionTok{c}\NormalTok{(}\DecValTok{0}\NormalTok{,}\DecValTok{1}\NormalTok{), }
                       \AttributeTok{low =} \StringTok{\textquotesingle{}blue\textquotesingle{}}\NormalTok{, }
                       \AttributeTok{high =} \StringTok{\textquotesingle{}red\textquotesingle{}}\NormalTok{, }
                       \AttributeTok{mid =} \StringTok{\textquotesingle{}white\textquotesingle{}}\NormalTok{, }
                       \AttributeTok{midpoint =} \FloatTok{0.5}\NormalTok{) }
\end{Highlighting}
\end{Shaded}

\includegraphics{Drake_Zhou_HW_11_files/figure-latex/solution_04f_a-1.pdf}

Visualize the predicted probability surface based on the random forest
grid.

\begin{Shaded}
\begin{Highlighting}[]
\DocumentationTok{\#\#\#}
\NormalTok{viz\_input\_grid }\SpecialCharTok{\%\textgreater{}\%}
  \FunctionTok{bind\_cols}\NormalTok{(pred\_prob\_rf\_on\_rf\_grid) }\SpecialCharTok{\%\textgreater{}\%}
  \FunctionTok{ggplot}\NormalTok{(}\AttributeTok{mapping =} \FunctionTok{aes}\NormalTok{(}\AttributeTok{x =}\NormalTok{ V50, }\AttributeTok{y =}\NormalTok{ V57))}\SpecialCharTok{+}
  \FunctionTok{geom\_raster}\NormalTok{(}\AttributeTok{mapping =} \FunctionTok{aes}\NormalTok{(}\AttributeTok{fill =}\NormalTok{ M))}\SpecialCharTok{+}
  \FunctionTok{facet\_grid}\NormalTok{(}\AttributeTok{rows =} \FunctionTok{vars}\NormalTok{(V31), }\AttributeTok{cols =} \FunctionTok{vars}\NormalTok{(V22))}\SpecialCharTok{+}
  \FunctionTok{scale\_fill\_gradient2}\NormalTok{(}\AttributeTok{limits =} \FunctionTok{c}\NormalTok{(}\DecValTok{0}\NormalTok{,}\DecValTok{1}\NormalTok{), }
                       \AttributeTok{low =} \StringTok{\textquotesingle{}blue\textquotesingle{}}\NormalTok{, }
                       \AttributeTok{high =} \StringTok{\textquotesingle{}red\textquotesingle{}}\NormalTok{, }
                       \AttributeTok{mid =} \StringTok{\textquotesingle{}white\textquotesingle{}}\NormalTok{, }
                       \AttributeTok{midpoint =} \FloatTok{0.5}\NormalTok{)}
\end{Highlighting}
\end{Shaded}

\includegraphics{Drake_Zhou_HW_11_files/figure-latex/solution_04f_b-1.pdf}

\hypertarget{problem-05}{%
\subsection{Problem 05}\label{problem-05}}

Now that you have trained and tuned multiple models of varying
complexity, it's time to identify the best performing model.

\hypertarget{a-4}{%
\subsubsection{5a)}\label{a-4}}

The resampling results are compiled together using the
\texttt{resamples()} function. You must complete the first code chunk
below by assigning the model object to the corresponding name in the
list. For example, you must set the \texttt{fit\_glm\_sonar} object to
the \texttt{GLM} variable in the list.

\textbf{Complete the first code chunk below, by assigning the
\texttt{caret} trained model objects to their appropriate variables in
the list.}

\textbf{The results are then plotted for you using \texttt{dotplot()}
which model is the best?}

\hypertarget{solution-23}{%
\paragraph{SOLUTION}\label{solution-23}}

According to the figure, RF is the best model.

\begin{Shaded}
\begin{Highlighting}[]
\NormalTok{sonar\_roc\_compare }\OtherTok{\textless{}{-}} \FunctionTok{resamples}\NormalTok{(}\FunctionTok{list}\NormalTok{(}\AttributeTok{GLM =}\NormalTok{ fit\_glm\_sonar,}
                                    \AttributeTok{GLMNET =}\NormalTok{ fit\_glmnet\_sonar,}
                                    \AttributeTok{NNET =}\NormalTok{ fit\_nnet\_sonar,}
                                    \AttributeTok{RF =}\NormalTok{ fit\_rf\_sonar,}
                                    \AttributeTok{XGB =}\NormalTok{ fit\_xgb\_sonar))}

\FunctionTok{dotplot}\NormalTok{(sonar\_roc\_compare)}
\end{Highlighting}
\end{Shaded}

\includegraphics{Drake_Zhou_HW_11_files/figure-latex/solution_05a-1.pdf}

\hypertarget{b-4}{%
\subsubsection{5b)}\label{b-4}}

The code chunk below is completed for you. It extracts the hold-out set
predictions associated with the best tuning parameter values for 4 of
the models you trained, the logistic regression model, the elastic net,
the neural network, and the random forest.

\begin{Shaded}
\begin{Highlighting}[]
\NormalTok{model\_pred\_results }\OtherTok{\textless{}{-}}\NormalTok{ fit\_rf\_sonar}\SpecialCharTok{$}\NormalTok{pred }\SpecialCharTok{\%\textgreater{}\%}\NormalTok{ tibble}\SpecialCharTok{::}\FunctionTok{as\_tibble}\NormalTok{() }\SpecialCharTok{\%\textgreater{}\%} 
  \FunctionTok{filter}\NormalTok{(mtry }\SpecialCharTok{==}\NormalTok{ fit\_rf\_sonar}\SpecialCharTok{$}\NormalTok{bestTune}\SpecialCharTok{$}\NormalTok{mtry) }\SpecialCharTok{\%\textgreater{}\%} 
  \FunctionTok{select}\NormalTok{(pred, obs, M, R, rowIndex, Resample) }\SpecialCharTok{\%\textgreater{}\%} 
  \FunctionTok{mutate}\NormalTok{(}\AttributeTok{model\_name =} \StringTok{"RF"}\NormalTok{) }\SpecialCharTok{\%\textgreater{}\%} 
  \FunctionTok{bind\_rows}\NormalTok{(fit\_glm\_sonar}\SpecialCharTok{$}\NormalTok{pred }\SpecialCharTok{\%\textgreater{}\%}\NormalTok{ tibble}\SpecialCharTok{::}\FunctionTok{as\_tibble}\NormalTok{() }\SpecialCharTok{\%\textgreater{}\%} 
  \FunctionTok{select}\NormalTok{(pred, obs, M, R, rowIndex, Resample) }\SpecialCharTok{\%\textgreater{}\%} 
  \FunctionTok{mutate}\NormalTok{(}\AttributeTok{model\_name =} \StringTok{"GLM"}\NormalTok{)) }\SpecialCharTok{\%\textgreater{}\%} 
  \FunctionTok{bind\_rows}\NormalTok{(fit\_glmnet\_sonar}\SpecialCharTok{$}\NormalTok{pred }\SpecialCharTok{\%\textgreater{}\%}\NormalTok{ tibble}\SpecialCharTok{::}\FunctionTok{as\_tibble}\NormalTok{() }\SpecialCharTok{\%\textgreater{}\%} 
              \FunctionTok{filter}\NormalTok{(alpha }\SpecialCharTok{==}\NormalTok{ fit\_glmnet\_sonar}\SpecialCharTok{$}\NormalTok{bestTune}\SpecialCharTok{$}\NormalTok{alpha,}
\NormalTok{                     lambda }\SpecialCharTok{==}\NormalTok{ fit\_glmnet\_sonar}\SpecialCharTok{$}\NormalTok{bestTune}\SpecialCharTok{$}\NormalTok{lambda) }\SpecialCharTok{\%\textgreater{}\%} 
              \FunctionTok{select}\NormalTok{(pred, obs, M, R, rowIndex, Resample) }\SpecialCharTok{\%\textgreater{}\%} 
              \FunctionTok{mutate}\NormalTok{(}\AttributeTok{model\_name =} \StringTok{"GLMNET"}\NormalTok{)) }\SpecialCharTok{\%\textgreater{}\%} 
  \FunctionTok{bind\_rows}\NormalTok{(fit\_nnet\_sonar}\SpecialCharTok{$}\NormalTok{pred }\SpecialCharTok{\%\textgreater{}\%}\NormalTok{ tibble}\SpecialCharTok{::}\FunctionTok{as\_tibble}\NormalTok{() }\SpecialCharTok{\%\textgreater{}\%} 
              \FunctionTok{filter}\NormalTok{(size }\SpecialCharTok{==}\NormalTok{ fit\_nnet\_sonar}\SpecialCharTok{$}\NormalTok{bestTune}\SpecialCharTok{$}\NormalTok{size,}
\NormalTok{                     decay }\SpecialCharTok{==}\NormalTok{ fit\_nnet\_sonar}\SpecialCharTok{$}\NormalTok{bestTune}\SpecialCharTok{$}\NormalTok{decay) }\SpecialCharTok{\%\textgreater{}\%} 
              \FunctionTok{select}\NormalTok{(pred, obs, M, R, rowIndex, Resample) }\SpecialCharTok{\%\textgreater{}\%} 
              \FunctionTok{mutate}\NormalTok{(}\AttributeTok{model\_name =} \StringTok{"NNET"}\NormalTok{))}
\end{Highlighting}
\end{Shaded}

The first few rows of the \texttt{model\_pred\_results} object are
displayed for you in the code chunk below.

\begin{Shaded}
\begin{Highlighting}[]
\NormalTok{model\_pred\_results }\SpecialCharTok{\%\textgreater{}\%} \FunctionTok{head}\NormalTok{()}
\end{Highlighting}
\end{Shaded}

\begin{verbatim}
## # A tibble: 6 x 7
##   pred  obs       M     R rowIndex Resample model_name
##   <fct> <fct> <dbl> <dbl>    <int> <chr>    <chr>     
## 1 M     R     0.624 0.376        3 Fold1    RF        
## 2 R     R     0.494 0.506        5 Fold1    RF        
## 3 R     R     0.294 0.706       10 Fold1    RF        
## 4 R     R     0.104 0.896       11 Fold1    RF        
## 5 R     R     0.498 0.502       16 Fold1    RF        
## 6 M     R     0.63  0.37        20 Fold1    RF
\end{verbatim}

You must use the \texttt{model\_pred\_results} to plot the ROC curves
associated with the best tuning parameter results. You will use the
\texttt{roc\_curve()} function from the \texttt{yardstick} package to
create the ROC curve, just as you did in an earlier assignment. You
should have \texttt{yardstick} installed already. If you do not please
download and install \texttt{yardstick} before running the code chunk
below.

\begin{Shaded}
\begin{Highlighting}[]
\FunctionTok{library}\NormalTok{(yardstick)}
\end{Highlighting}
\end{Shaded}

\begin{verbatim}
## For binary classification, the first factor level is assumed to be the event.
## Use the argument `event_level = "second"` to alter this as needed.
\end{verbatim}

\begin{verbatim}
## 
## Attaching package: 'yardstick'
\end{verbatim}

\begin{verbatim}
## The following objects are masked from 'package:caret':
## 
##     precision, recall, sensitivity, specificity
\end{verbatim}

\begin{verbatim}
## The following object is masked from 'package:readr':
## 
##     spec
\end{verbatim}

\textbf{Create the cross-validation averaged ROC curve for each of the
models you trained. To do so, pipe the \texttt{model\_pred\_results}
objects into \texttt{group\_by()} and specify the grouping variable as
\texttt{model\_name}. Pipe the result to \texttt{roc\_curve()} function
with \texttt{obs} as the first argument and \texttt{M} as the second
argument. Pipe the result to the \texttt{autoplot()} function.}

\textbf{Are your generated ROC curves in aggreement with your
cross-validation summary stats displayed before?}

\hypertarget{solution-24}{%
\paragraph{SOLUTION}\label{solution-24}}

What do you think?

Yes, they correspond to each other, RF is still the best model.

\begin{Shaded}
\begin{Highlighting}[]
\NormalTok{model\_pred\_results }\SpecialCharTok{\%\textgreater{}\%}
  \FunctionTok{group\_by}\NormalTok{(model\_name) }\SpecialCharTok{\%\textgreater{}\%}
  \FunctionTok{roc\_curve}\NormalTok{(obs, M) }\SpecialCharTok{\%\textgreater{}\%}
  \FunctionTok{autoplot}\NormalTok{()}
\end{Highlighting}
\end{Shaded}

\includegraphics{Drake_Zhou_HW_11_files/figure-latex/solution_05b-1.pdf}

\hypertarget{c-4}{%
\subsubsection{5c)}\label{c-4}}

It can be useful to get a sense of the variability in the ROC curve,
based on the variation in performance across resample folds. You will
use the \texttt{roc\_curve()} function again, but this time you cannot
use \texttt{autoplot()} to create the ROC curve. You will need to use
the \texttt{geom\_path()} function from ggplot2, to create the figure
manually.

\textbf{Pipe the \texttt{model\_pred\_results} object to
\texttt{group\_by()} and specify the grouping variables to be
\texttt{model\_name} and \texttt{Resample}. Pipe the result to
\texttt{roc\_curve()} with the first argument assigned as \texttt{obs}
and the second argument assigned as \texttt{M}. Pipe the result to
\texttt{ggplot()} where you map the \texttt{x} aesthetic to
\texttt{1\ -\ specificity} and the \texttt{y} aesthetic to
\texttt{sensitivity}. Add in the \texttt{geom\_path()} layer where you
map the group \texttt{color} aesthetic to \texttt{Resample}. Use the
\texttt{facet\_wrap()} function to create separate facets for the models
that you trained. However, before your \texttt{facet\_wrap()} function
include \texttt{geom\_abline()} with \texttt{slope\ =\ 1} and
\texttt{intercept\ =\ 0} and
\texttt{linetype\ =\ \textquotesingle{}dotted\textquotesingle{}}. Also
include the \texttt{coord\_equa()} function. The line and equal
coordinates will help make the ROC curve graphic easier to read.}

\hypertarget{solution-25}{%
\paragraph{SOLUTION}\label{solution-25}}

\begin{Shaded}
\begin{Highlighting}[]
\NormalTok{model\_pred\_results }\SpecialCharTok{\%\textgreater{}\%}
  \FunctionTok{group\_by}\NormalTok{(model\_name,Resample) }\SpecialCharTok{\%\textgreater{}\%}
  \FunctionTok{roc\_curve}\NormalTok{(obs, M) }\SpecialCharTok{\%\textgreater{}\%}
  \FunctionTok{ggplot}\NormalTok{(}\FunctionTok{aes}\NormalTok{(}\AttributeTok{x =} \DecValTok{1} \SpecialCharTok{{-}}\NormalTok{ specificity, }\AttributeTok{y =}\NormalTok{ sensitivity))}\SpecialCharTok{+}
  \FunctionTok{geom\_path}\NormalTok{(}\FunctionTok{aes}\NormalTok{(}\AttributeTok{color =}\NormalTok{ Resample)) }\SpecialCharTok{+} 
  \FunctionTok{geom\_abline}\NormalTok{(}\AttributeTok{slope =} \DecValTok{1}\NormalTok{, }\AttributeTok{intercept =} \DecValTok{0}\NormalTok{, }\AttributeTok{linetype =} \StringTok{\textquotesingle{}dotted\textquotesingle{}}\NormalTok{) }\SpecialCharTok{+} 
  \FunctionTok{coord\_equal}\NormalTok{() }\SpecialCharTok{+}
  \FunctionTok{facet\_wrap}\NormalTok{( }\SpecialCharTok{\textasciitilde{}}\NormalTok{ model\_name)}
\end{Highlighting}
\end{Shaded}

\includegraphics{Drake_Zhou_HW_11_files/figure-latex/solution_05c-1.pdf}

\end{document}
